\documentclass[UTF8]{ctexart}
\usepackage{amsmath}
\usepackage{amssymb}
\usepackage{booktabs}
\usepackage{background}
\usepackage{enumitem}
\usepackage{float}
\usepackage{fontspec}
\usepackage{fourier}
\usepackage{geometry}
\usepackage{longtable}
\usepackage{xcolor}

\geometry{a5paper, top=0.1cm, left=1cm, right=1cm, bottom=0.8cm, footskip=0.1cm}
\setCJKmainfont[BoldFont={汉仪文黑-85W},ItalicFont={方正苏新诗柳楷简体}]{汉仪文黑-55W}
\setfontfamily\Issue{Century Schoolbook}
\newCJKfontfamily\TitleFont{思源宋体 CN Heavy}
\newfontfamily\timesnewroman{Times New Roman}
%\reversemarginpar

%\CTEXsetup[format = {\centering\bfseries\large}, beforeskip = 3pt, afterskip = 3pt]{section}
\CTEXsetup[format = {\color{cyan!50!black}\bfseries\large}]{subsection}

\colorlet{darkcyan}{cyan!50!black}
\newcommand\Black[1]{\textcolor[gray]{0.3}{#1}}
\newcommand\Brown[1]{\textcolor[HTML]{998A4E}{#1}}
\newcommand\Emph[1]{\colorbox{green!10}{\textcolor{green!30!black}{#1}}}
\newcommand\Notes[1]{\textcolor{yellow!50!black}{\small #1}}
\newcommand\Example[1]{\textcolor{cyan!70!black}{\small #1}}

\newcommand\x{\boldsymbol{x}}
\newcommand\A{\boldsymbol{A}}
\newcommand\W{\boldsymbol{W}}

\newcommand\IssueNumber{18}
\newcommand\Date{2024-3-6}
%\newcommand\Contributer{@金光日}
\newcommand\Subject{矩阵与数值分析}


\begin{document}
\backgroundsetup{contents=\includegraphics{上半示例.png}, center, scale=1, angle=0, opacity=1}
\BgThispage
\begin{center}
{\scriptsize\Issue \textcolor[HTML]{C8BA83}{WEEKLY TIPS}}

{\Huge\bfseries\TitleFont \Black{知\ 识\ 小\ 料}}

\vspace{-0.1cm}
{\footnotesize \Brown{「电计 2203 班」周常规知识整理共享}}
\end{center}

\vspace{-0.5cm}

\begin{figure}[H]
\hspace{1cm}
\begin{minipage}[t]{0.3\textwidth}
\centering
    \Brown{ISSUE.}

    \vspace{-0.6cm}
    \Huge \Issue\slshape\bfseries\Black{\IssueNumber}
\end{minipage}
\hfill
\begin{minipage}[t]{0.35\textwidth}
\centering
    \Brown{日期:\Date} \\
%\vspace{-0.1cm}
%    \Brown{贡献者:\Contributer} \\
\vspace{-0.1cm}
    \Brown{学科:\Subject} \\
\end{minipage}
\hspace{0.8cm}
\end{figure}

{\color{cyan!50!black}
求以下范数:

\begin{enumerate}[itemsep=0pt,parsep=0pt]
    \item 求向量 $\x = [1,-2,3,-4]^{\mathrm{T}}$ 的 1-范数、2-范数和无穷范数;
    \item 求矩阵 $\A = \begin{bmatrix}
                      1 & 0 & -1 & 0 \\
                      0 & 2 & 0 & 5\\
                      0 & -1 & -3 & 0\\
                    \end{bmatrix}$ 的 $m_1$-范数和 F-范数(Frobenius范数);
    \item 求向量 $\x = [1,-2,3]^{\mathrm{T}}$ 在权矩阵 $\W = \begin{bmatrix}
                                                        -2 & 0 & 0 \\
                                                        0 & 3 & 0 \\
                                                        0 & 1 & 1 \\
                                                      \end{bmatrix}$ 下的加权 $p$-范数。
\end{enumerate}

}

本题只需要使用范数的定义公式计算即可。

【第 1 题】向量 $\x = [x_1,x_2,\dots,x_n]^{\mathrm{T}}$ 的各种范数即为(注意\textbf{粗体}向量 $\x$ 和非粗体分量 $x_i$ 的显示区别):
\begin{itemize}
  \item 1-范数:$\|\x\|_1 = \sum\limits_{i=1}^n |x_i| $ \textcolor{cyan}{$= 1+|-2|+3+|-4|=10$}
  \item 2-范数:$\|\x\|_2 = \sqrt{\sum\limits_{i=1}^n |x_i|^2} $\textcolor{cyan}{$= \sqrt{1^2 + |-2|^2 + 3^2 + |-4|^2} = \sqrt{30}$}
  \item 无穷范数:$\|\x\|_\infty = \max\limits_{1\leqslant i\leqslant n} |x_i| $\textcolor{cyan}{$= \max\{1,|-2|,3,|-4|\} = 4$}
  \item $p$-范数:$\|\x\|_p = \left(\sum\limits_{i=1}^n |x_i|^p\right)^{\frac{1}{p}}$——也就是上面三者的一般化形式
\end{itemize}

【第 2 题】$m$ 行 $n$ 列矩阵 $\A$ 的各种范数即为
\begin{itemize}
  \item $m_1$-范数:$\|\A\|_{m_1} = \sum\limits_{i=1}^m \sum\limits_{j=1}^n  |a_{ij}| $ \textcolor{cyan}{$=1 + |-1| + 2 + 5 + |-1| + |-3| = 13$}
  \item F-范数:$\|\A\|_{\mathrm{F}} = \sqrt{\sum\limits_{i=1}^m \sum\limits_{j=1}^n  |a_{ij}|^2} $ \textcolor{cyan}{$= \sqrt{1^2 + |-1|^2 + 2^2 + 5^2 + |-1|^2 + |-3|^2} = \sqrt{41}$}
\end{itemize}

\newpage
\backgroundsetup{contents=\includegraphics{下半示例.png}, center, scale=1, angle=0, opacity=1}
\BgThispage
【第 3 题】加权范数的定义是 $\|\x\|_{\W} = \|\W\x\|$。由于是向量的 $p$-范数,因此要计算的是 $\|\W\x\|_p$。

\begin{align*}
  \|\W\x\|_p &= \left\| \begin{bmatrix}
                          -2 & 0 & 0 \\
                         0 & 3 & 0 \\
                         0 & 1 & 1 \\
                        \end{bmatrix}\begin{bmatrix}
                                       1 \\ -2 \\ 3 \\
                        \end{bmatrix} \right\|_p  \\
    &= \left\|\begin{bmatrix}
         -2 \\ -6 \\ 1 \\
       \end{bmatrix}\right\|_p \\
    &= (|-2|^p + |-6|^p + 1^p)^{\frac{1}{p}} \\
    &= (2^p + 6^p + 1)^{\frac1p} \quad (1\leqslant p < \infty)\\
\end{align*}

\textcolor{cyan!80!black}{【结论】
\begin{enumerate}
    \item $\|\x\|_1 = 10$,$\|\x\|_2 = \sqrt{30}$,$\|\x\|_\infty = 4$;
    \item $\|\A\|_{m_1} = 13$,$\|\A\|_{\mathrm{F}} = \sqrt{41}$;
    \item $\|\W\x\|_p = (2^p + 6^p + 1)^{\frac1p}$。
\end{enumerate}
}

\textcolor{cyan!80!black}{【点评】本题看着很唬人,其实就是套公式算就可以了。据说《矩阵与数值分析》的考试题大多都不难,主要考察基础概念。}

\end{document} 