\documentclass[UTF8]{ctexart}
\usepackage{amsmath}
\usepackage{amssymb}
\usepackage{background}
\usepackage{booktabs}
\usepackage{caption}
\usepackage{enumitem}
\usepackage{extarrows} %用于在长等号上方说明文字
\usepackage{fourier}
\usepackage{graphicx}
\usepackage{geometry}
\usepackage{float}
\usepackage{fontspec}
%\usepackage{mathptmx} %mathptmx和times结合使得公式使用times new roman字体
\usepackage{pifont}
\usepackage{tikz}
%\usepackage{times}
\usepackage{xcolor}
\usetikzlibrary{positioning, arrows.meta, shapes.misc}



\geometry{a5paper, top=0.1cm, left=1cm, right=1cm, bottom=1cm, footskip=0.3cm, marginparsep=0.1cm}
\setCJKmainfont[BoldFont={汉仪文黑-85W},ItalicFont={方正苏新诗柳楷简体}]{汉仪文黑-55W}
\setfontfamily\Issue{Century Schoolbook}
\newCJKfontfamily\TitleFont{思源宋体 CN Heavy}
\newfontfamily\timesnewroman{Times New Roman}
\setmainfont{Times New Roman}
\captionsetup{font=small, labelfont=bf}
\colorlet{backcolor}{yellow!80!gray!20!} %若只有两页以内,可删去
\pagecolor{backcolor}

\CTEXsetup[format = {\centering\bfseries\normalsize}, beforeskip = 0pt, afterskip = 0pt]{section}

\newcommand\Black[1]{\textcolor[gray]{0.3}{#1}}
\newcommand\Brown[1]{\textcolor[HTML]{998A4E}{#1}}
\newcommand\Emph[1]{\colorbox{red!20!}{\textcolor{red!80!black}{#1}}}
\newcommand\Correct[1]{\colorbox{green!20}{\textcolor{green!50!black}{#1}}}
\newcommand\Mathemph[1]{\text{\textcolor{green!60!black}{$#1$}}}
\newcommand\Concept[1]{\colorbox{cyan!10!white}{\textcolor{cyan!40!black}{#1}}}
\newcommand\Notes[1]{\textcolor{yellow!50!black}{\small #1}}
\newcommand\Example[1]{\textcolor{cyan!70!black}{\small #1}}
\newcommand\relation[2]{\langle #1,#2 \rangle}
\newcommand\pos[1]{\marginpar{\footnotesize\ttfamily\textcolor{yellow!50!black}{\hfill #1}}}


\newcommand\ddd{\mathrm{d}}
\newcommand\mypi{\text{\timesnewroman π}}

%这4个信息随“刊号”更新
\newcommand\IssueNumber{09}
\newcommand\Date{2023-11-15}
%\newcommand\Contributer{@金光日}
\newcommand\Subject{大学物理A2}


\begin{document}
\backgroundsetup{contents=\includegraphics{示例.png}, center, scale=1, angle=0, opacity=1}
\BgThispage
\begin{center}
{\scriptsize\Issue \textcolor[HTML]{C8BA83}{WEEKLY TIPS}}

{\Huge\bfseries\TitleFont \Black{知\ 识\ 小\ 料}}

\vspace{-0.1cm}
{\footnotesize \Brown{「电计 2203 班」周常规知识整理共享}}
\end{center}

\vspace{-0.5cm}

\begin{figure}[H]
\hspace{1cm}
\begin{minipage}[t]{0.3\textwidth}
\centering
    \Brown{ISSUE.}

    \vspace{-0.6cm}
    \Huge \Issue\slshape\bfseries\Black{\IssueNumber}
\end{minipage}
\hfill
\begin{minipage}[t]{0.3\textwidth}
\centering
    \Brown{日期:\Date} \\
%\vspace{-0.1cm}
%    \Brown{贡献者:\Contributer} \\
\vspace{-0.1cm}
    \Brown{学科:\Subject} \\
\end{minipage}
\vspace{-0.2cm}
\end{figure}

{\color{cyan!50!black}
回答以下几个有关位移电流的问题:
\begin{enumerate}[itemsep=0pt,parsep=0pt]
    \item 位移电流是由变化的\underline{\hspace{2cm}}产生的。(电场/磁场)
    \item 均匀电场被限制在半径为 $R$ 的圆形电容器内,以 $\dfrac{\ddd D}{\ddd t}$ 的电位移均匀增加,则在电容器极板间距离圆心 $r=0.5R$ 处点的位移电流的密度值为\underline{\hspace{2cm}},$r=2R$ 处的值为 \underline{\hspace{2cm}}。
    \item 一平行板电容器的两极板均为半径为 $R$ 的圆形导体片,在充电时,板间电场强度随时间的变化率为 $\dfrac{\ddd E}{\ddd t}$,若忽略边缘效应,则两极板间的位移电流大小为\underline{\hspace{2cm}}。
\end{enumerate}
}

有几个比较容易混淆的物理量,先在此列出:
\begin{itemize}[itemsep=0pt]
    \item 位移电流:$\displaystyle I_D = \oiint_S \boldsymbol{J}_D\cdot\ddd\boldsymbol{S} =  \oiint_S \dfrac{\ddd\boldsymbol{D}}{\ddd t}\cdot \ddd\boldsymbol{S} = \dfrac{\ddd\varPhi_D}{\ddd t}$
    \item 位移电流密度:$\boldsymbol{J}_D = \dfrac{\ddd\boldsymbol{D}}{\ddd t}$\textcolor{cyan}{(有时写成偏导数形式 $\dfrac{\partial\boldsymbol{D}}{\partial t}$)}
    \item 电位移通量:$\varPhi_D = \boldsymbol{D}\cdot\boldsymbol{S} = \sigma S = Q$
    \item 电位移矢量:$\boldsymbol{D} = \varepsilon_0\boldsymbol{E}$\textcolor{cyan}{(当有其他电介质 $\varepsilon_r$ 时则 $\boldsymbol{D} = \varepsilon_0\varepsilon_r\boldsymbol{E}$)}
\end{itemize}

\begin{enumerate}[itemsep=0pt]
  \item 位移电流能产生磁场,应该由变化的电场产生。
  \item $r=0.5R$这一点在极板间,因此位移电流的密度就是 $\dfrac{\ddd D}{\ddd t}$;$r=2R$这一点超出了极板范围,因此位移电流密度为 0。
  \item 位移电流有两种推导方式:$\displaystyle I_D = \oiint_S \dfrac{\ddd\boldsymbol{D}}{\ddd t}\cdot \ddd\boldsymbol{S} \xlongequal{\text{积分}} \dfrac{\ddd D}{\ddd t}\cdot \mypi R^2 \xlongequal{D=\varepsilon_0 E} \varepsilon_0\cdot \dfrac{\ddd E}{\ddd t}\cdot \mypi R^2$,或者说 $I_D = \dfrac{\ddd\varPhi_D}{\ddd t} \xlongequal{\varPhi_D = D S} \dfrac{\ddd D}{\ddd t}\cdot \mypi R^2 \xlongequal{D=\varepsilon_0 E} \varepsilon_0 \mypi R^2\dfrac{\ddd E}{\ddd t}$。
\end{enumerate}

\textcolor{cyan!80!black}{【结论】1.电场\quad 2.$\dfrac{\ddd D}{\ddd t}$;$0$\quad 3.$\varepsilon_0\mypi R^2 \dfrac{\ddd E}{\ddd t}$}

\vspace{0.3cm}

\textcolor{cyan!80!black}{【点评】本题涉及到位移电流的知识,一般考填空题,难度不高,但相对冷门,因此本次「知识小料」对这个知识点进行了复习。}


\end{document} 