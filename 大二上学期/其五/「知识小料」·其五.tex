\documentclass[UTF8]{ctexart}
\usepackage{amsmath}
\usepackage{amssymb}
\usepackage{background}
\usepackage{caption, subcaption}
\usepackage{circuitikz}
\usepackage{enumitem}
\usepackage{float}
\usepackage{fourier}
\usepackage{fontspec}
\usepackage{geometry}
\usepackage{pifont}
\usepackage{tikz}
\usepackage{ulem}
\usepackage{xcolor}
\usetikzlibrary{arrows.meta}



%\backgroundsetup{contents=\includegraphics{示例.png}, center, scale=1, angle=0, opacity=1}
\geometry{a5paper, top=0.1cm, left=1cm, right=1cm, bottom=1.1cm}
\setCJKmainfont[BoldFont={汉仪文黑-85W},ItalicFont={方正苏新诗柳楷简体}]{汉仪文黑-55W}
\setfontfamily\Issue{Century Schoolbook}
\newCJKfontfamily\TitleFont{思源宋体 CN Heavy}
\newfontfamily\timesnewroman{Times New Roman}
\captionsetup{font=small, labelfont=bf}


\newcommand\Black[1]{\textcolor[gray]{0.3}{#1}}
\newcommand\Brown[1]{\textcolor[HTML]{998A4E}{#1}}
\newcommand\Emph[1]{\colorbox{red!20!}{\textcolor{red!80!black}{#1}}}
\newcommand\Correct[1]{\colorbox{green!20}{\textcolor{green!50!black}{#1}}}
\newcommand\mypi{\text{\timesnewroman π}}
%这4个信息随“刊号”更新
\newcommand\IssueNumber{05}
\newcommand\Date{2023-10-16}
%\newcommand\Contributer{@金光日}
\newcommand\Subject{积分变换与场论}
\renewcommand\thefootnote{\ding{\numexpr171 + \value{footnote}}}
\newcommand\xb[1]{_\mathrm{#1}}
\newcommand\e{\mathrm{e}}
\newcommand\ddd{\mathrm{d}}

\ctikzset{resistors/scale=0.6, % smaller R
    capacitors/scale=0.7, % even smaller C
    diodes/scale=0.6, % small diodes
    transistors/scale=1.3,
    sources/scale=0.6,
}


\begin{document}
\backgroundsetup{contents=\includegraphics{上半示例.png}, center, scale=1, angle=0, opacity=1}
\BgThispage
\begin{center}
{\scriptsize\Issue \textcolor[HTML]{C8BA83}{WEEKLY TIPS}}

{\Huge\bfseries\TitleFont \Black{知\ 识\ 小\ 料}}

\vspace{-0.1cm}
{\footnotesize \Brown{「电计 2203 班」周常规知识整理共享}}
\end{center}

\vspace{-0.5cm}

\begin{figure}[H]
\hspace{1cm}
\begin{minipage}[t]{0.3\textwidth}
\centering
    \Brown{ISSUE.}

    \vspace{-0.6cm}
    \Huge \Issue\slshape\bfseries\Black{\IssueNumber}
\end{minipage}
\hfill
\begin{minipage}[t]{0.3\textwidth}
\centering
    \Brown{日期:\Date} \\
%\vspace{-0.1cm}
%    \Brown{贡献者:\Contributer} \\
\vspace{-0.1cm}
    \Brown{学科:\Subject} \\
\end{minipage}
\hspace{0.8cm}
\end{figure}

%此处以后填写正文
{\color{cyan!50!black}
利用拉普拉斯变换,求方程 $\left\{\begin{aligned} &y''' + 3y'' + 3y' + y = \e^t \\ &y(0)=y'(0)=y''(0)=0\end{aligned}\right.$ 的解。
}

先两边取拉普拉斯变换,由
\begin{equation*}
\begin{aligned}
    y(t) &\longrightarrow Y(s) \\
    y'(t) &\longrightarrow sY(s)-y(0) \\
    y''(t) &\longrightarrow s^2Y(s) - sy(0) - y'(0) \\
    y'''(t) &\longrightarrow s^3Y(s) - s^2y(0) - sy'(0) - y''(0) \\
\end{aligned}
\end{equation*}
并将初值条件 $y(0)=y'(0)=y''(0)=0$ 代入,得到
\begin{equation*}
    s^3Y(s) + 3s^2Y(s) + 3sY(s) + Y(s) = \mathcal{L}(\e^t) = \dfrac{1}{s-1} 
\end{equation*}
解得
\begin{equation}\label{eq:res}
    Y(s) = \dfrac{1}{(s-1)(s+1)^3}
\end{equation}

接下来要对上式取逆变换了。有多种方法可以实现逆变换,这里提供两种方法。

【方法一】\Emph{卷积定理}。使用 $f(t) * g(t) \longrightarrow F(s)\cdot G(s)$。由于变换后的结果是 $\dfrac{1}{s-1}\cdot \dfrac{1}{(s+1)^3}$,故需要对这两个乘积式进行逆变换。
\begin{equation*}
\begin{aligned}
    \mathcal{L}^{-1}\left[\dfrac{1}{s-1}\right] &= \e^t \\
    \mathcal{L}^{-1}\left[\dfrac{1}{(s+1)^3}\right] &= \e^{-t} \mathcal{L}^{-1}\left[\dfrac{1}{s^3}\right] = \e^{-t}\cdot \dfrac{1}{2}t^2 \text{\textcolor{cyan}{(逆用位移性质)}}\\
\end{aligned}
\end{equation*}
于是有 $y(t) = \e^t * \dfrac{1}{2}\e^{-t}t^2$。用卷积的定义展开该式得到
\begin{equation*}
\begin{aligned}
    y(t) &= \int_0^{t} \dfrac12\e^{-\tau}\tau ^2 \cdot \e^{t-\tau}\ddd\tau \\
    &= \dfrac12\e^t \int_0^{t} \tau^2\cdot \e^{-2\tau} \ddd\tau \\ 
    &= \dfrac12\e^t \left[-\dfrac12\e^{-2\tau} \left(\tau^2 + \tau + \dfrac12 \right)\right]\Bigg|_{\tau=0}^{\tau=t} \text{\textcolor{cyan}{(分部积分)}}\\
\end{aligned}
\end{equation*}

\newpage
\backgroundsetup{contents=\includegraphics{下半示例.png}, center, scale=1, angle=0, opacity=1}
\BgThispage
\phantom{...}
\vspace{0.3cm}
\begin{equation*}
\begin{aligned}
    &= \dfrac12\e^t \cdot \left[-\dfrac12\e^{-2t}\left(t^2 + t + \dfrac12\right) + \dfrac14\right] \\
    &= -\dfrac14 t^2 \e^{-t} - \dfrac14 t\e^{-t} - \dfrac18 \e^{-t} + \dfrac18 \e^t \\
\end{aligned}
\end{equation*}

【方法二】\Emph{反演积分公式+留数法}。我们知道,反演积分公式(课本 P108)的形式为
\begin{equation}\color{blue}
    f(t) = \mathcal{L}^{-1}[F(s)] = \sum_{k=1}^{n}\mathrm{Res}[F(s)\e^{st},\ s_k], t>0 
\end{equation}
其中 $s_k$ 是 $F(s)$ 的所有孤立奇点。那么在本题中,$Y(s)$ 有两个孤立奇点——$s=1$ 和 $s=-1$,应有
\begin{equation*}
    y(t) = \mathrm{Res}\left[\dfrac{\e^{st}}{(s-1)(s+1)^3},\ 1 \right] + \mathrm{Res}\left[\dfrac{\e^{st}}{(s-1)(s+1)^3},\ -1 \right]
\end{equation*}

而 $f(z)$ 的 $m$ 级极点处留数公式是
\begin{equation}\color{blue}
    \mathrm{Res}[f(z),z_0] = \dfrac{1}{(m-1)!} \lim\limits_{z\to z_0} [(z-z_0)^m f(z)]^{(m-1)}
\end{equation}
故对于一级极点 $s=1$ 得到
\begin{equation*}
    \mathrm{Res}\left[\dfrac{\e^{st}}{(s-1)(s+1)^3},\ 1 \right] = \lim\limits_{s\to 1} \dfrac{\e^{st}}{(s+1)^3} = \dfrac18 \e^{t}
\end{equation*}
对于三级极点 $s=-1$ 得到
\begin{equation*}
    \mathrm{Res}\left[\dfrac{\e^{st}}{(s-1)(s+1)^3},\ -1 \right] = \dfrac12\lim\limits_{s\to -1} \left(\dfrac{\e^{st}}{s-1}\right)_s'' = \dfrac18\e^{-t}(-2t^2 -2t -1)
\end{equation*}
因而将两个留数的结果相加得到
\begin{equation*}
    y(t) = \dfrac18\e^t + \dfrac18\e^{-t}(-2t^2 -2t -1)
\end{equation*}

\textcolor{cyan!80!black}{【结论】$y(t) = -\dfrac14 t^2 \e^{-t} - \dfrac14 t\e^{-t} - \dfrac18 \e^{-t} + \dfrac18 \e^t$。}

\vspace{0.3cm}

\textcolor{cyan!80!black}{【点评】本题是一道通过拉普拉斯变换求解微分方程的问题,难点在于如何做拉普拉斯逆变换。通常来说,一个函数做拉普拉斯逆变换有多种方法,如卷积定理、反演积分公式、部分分式法等,对于不同的问题应作出具体分析。}

\end{document} 