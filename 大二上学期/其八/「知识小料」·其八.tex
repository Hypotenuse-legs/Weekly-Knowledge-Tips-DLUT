\documentclass[UTF8]{ctexart}
\usepackage{amsmath}
\usepackage{amssymb}
\usepackage{background}
\usepackage{booktabs}
\usepackage{caption}
\usepackage{enumitem}
\usepackage{fancyhdr}
\usepackage{float}
\usepackage{fontspec}
\usepackage{fourier}
\usepackage{graphicx}
\usepackage{geometry}
\usepackage{pifont}
\usepackage{tcolorbox}
\usepackage{tikz}
\usepackage{xcolor}
\tcbuselibrary{breakable, raster}
\usetikzlibrary{positioning, arrows.meta, shapes.misc}



\geometry{a5paper, top=0.1cm, left=1cm, right=1cm, bottom=0.4cm, footskip=0cm, marginparsep=0.1cm}
\setCJKmainfont[BoldFont={汉仪文黑-85W},ItalicFont={方正苏新诗柳楷简体}]{汉仪文黑-55W}
\setfontfamily\Issue{Century Schoolbook}
\setfontfamily\tm{Times New Roman}
\newCJKfontfamily\TitleFont{思源宋体 CN Heavy}
\captionsetup{font=small, labelfont=bf}
\colorlet{backcolor}{yellow!80!gray!20!} %若只有两页以内,可删去
\pagecolor{backcolor}
\pagestyle{fancy}
\fancyhf{}
\rfoot{\ttfamily\footnotesize{-\ \thepage\ -}}

\newtcolorbox{mybox}{colback=cyan!10, colframe=cyan!50!black, boxrule=0.5pt, breakable}
\CTEXsetup[format = {\centering\bfseries\large}]{section}
\CTEXsetup[format = {\raggedright\bfseries\normalsize}]{subsection}
\CTEXsetup[format = {\color{cyan!50!black}\bfseries}]{paragraph}

\newcommand\Black[1]{\textcolor[gray]{0.3}{#1}}
\newcommand\Brown[1]{\textcolor[HTML]{998A4E}{#1}}


\renewcommand\arraystretch{1.7}
\newcommand\Emph[1]{\colorbox{red!10}{\textcolor{red!70!black}{#1}}}
\renewcommand\i{\mathrm{i}}
\newcommand\e{\mathrm{e}}
\renewcommand\d{\mathrm{d}}
\renewcommand\pi{\text{\tm π}}
\newcommand\iwt{{\mathrm{i}\omega t}}
\newcommand\F{\mathcal{F}}
\renewcommand\L{\mathcal{L}}
\newcommand\w{\omega}
\newcommand\fbh{\xrightarrow{\quad \mathcal{F} \quad}} %傅里叶变换
\newcommand\lbh{\xrightarrow{\quad \mathcal{L} \quad}} %拉普拉斯变换
\renewcommand\Re{\operatorname{Re}}
\newcommand\A{\boldsymbol{A}}
\newcommand\ii{\boldsymbol{i}}
\newcommand\jj{\boldsymbol{j}}
\newcommand\kk{\boldsymbol{k}}
\newcommand\grad{\mathbf{grad}}
\renewcommand\div{\operatorname{div}} %div原指除号,今借以表示散度
\newcommand\rot{\mathbf{rot}}


%这4个信息随“刊号”更新
\newcommand\IssueNumber{08}
\newcommand\Date{2023-11-7}
%\newcommand\Contributer{@金光日}
\newcommand\Subject{积分变换与场论}


\begin{document}
\backgroundsetup{contents=\includegraphics{上半示例.png}, center, scale=1, angle=0, opacity=1}
\BgThispage
\begin{center}
{\scriptsize\Issue \textcolor[HTML]{C8BA83}{WEEKLY TIPS}}

{\Huge\bfseries\TitleFont \Black{知\ 识\ 小\ 料}}

\vspace{-0.1cm}
{\footnotesize \Brown{「电计 2203 班」周常规知识整理共享}}
\end{center}

\vspace{-0.5cm}

\begin{figure}[H]
\hspace{1cm}
\begin{minipage}[t]{0.3\textwidth}
\centering
    \Brown{ISSUE.}

    \vspace{-0.6cm}
    \Huge \Issue\slshape\bfseries\Black{\IssueNumber}
\end{minipage}
\hfill
\begin{minipage}[t]{0.3\textwidth}
\centering
    \Brown{日期:\Date} \\
%\vspace{-0.1cm}
%    \Brown{贡献者:\Contributer} \\
\vspace{-0.1cm}
    \Brown{学科:\Subject} \\
\end{minipage}
\vspace{-0.2cm}
\end{figure}

\begin{center}\color{cyan!50!black}
本文档将用于对积分变换作出简明复习。
\begin{equation*}
    \int_0^{+\infty} \dfrac{\sin x}{x}\d x = \dfrac{\pi}{2} \qquad \displaystyle \int_0^{+\infty}  \e^{-x^2}\d x = \dfrac{\sqrt{\pi}}{2}
\end{equation*}
\end{center}

\section{傅里叶变换}
\subsection{傅里叶级数}
周期为 $T$ 的函数 $f_T(t)$ 的傅里叶级数展开为 $\displaystyle f_T(t) = \sum\limits_{n=-\infty}^{+\infty} c_n\e^{\i\omega_n t}$

\subsection{傅里叶变换·概念}
\paragraph{傅里叶积分定理} $f(t)$ 需在 $(-\infty, +\infty)$ 满足:
\begin{itemize}[itemsep=0pt, parsep=0pt]
    \item 在任意有限区间内满足狄利克雷条件:
    \begin{itemize}[itemsep=0pt, parsep=0pt]
        \item 连续,或只有有限个第一类间断点;
        \item 只有有限个极值点;
    \end{itemize}
    \item $f(t)$ 在 $(-\infty, +\infty)$ 绝对可积。
\end{itemize}
\begin{equation}\label{eq:f-intergral}
    \dfrac{1}{2\pi} \int_{-\infty}^{+\infty}
    \left[
        \int_{-\infty}^{+\infty} f(t)\e^{-\iwt}\d t
    \right] \cdot\e^{\iwt}\d \omega
\end{equation}
另外,$f(t)$ 在每个点收敛于 $\frac12[f(t-0) + f(t+0)]$。

\paragraph{傅里叶变换}
以下是傅里叶变换的定义。

\begin{table}[H]
    \centering
    \renewcommand\arraystretch{1.6}
    \begin{tabular}{|c|c|c|c|}
    \hline
    表达式 & 正变换 & 逆变换 \\[-5pt]
    \hline
    原始变换 & $\displaystyle F(\omega) = \int_{-\infty}^{+\infty} f(t)\e^{-\iwt}\d t$
    & $\displaystyle f(t) = \dfrac{1}{2\pi} \int_{-\infty}^{+\infty}  F(\omega)\e^{\iwt}\d \omega$ \\
    \hline
    正弦变换 & $\displaystyle F(\omega) = \int_0^{+\infty} f(t)\sin\omega t\ \d t$
    & $\displaystyle f(t) = \dfrac{2}{\pi} \int_0^{+\infty} F(\omega) \sin\omega t\ \d\omega$ \\
    \hline
    余弦变换 & $\displaystyle F(\omega) = \int_0^{+\infty} f(t)\cos\omega t\ \d t$
    & $\displaystyle f(t) = \dfrac{2}{\pi} \int_0^{+\infty} F(\omega) \cos\omega t\ \d\omega$ \\
    \hline
    \end{tabular}
    \caption{傅里叶变换式}\label{tab:f-transform}
\end{table}

\newpage
\backgroundsetup{contents=\includegraphics{空白示例.png}, center, scale=1, angle=0, opacity=1}
\BgThispage

\paragraph{常见函数的傅里叶变换}
以下是常见函数的傅里叶变换列表。

\begin{tcbraster}[raster columns=2, raster height=14cm]
\begin{mybox}
    \begin{equation*}\begin{aligned}
        \delta(t) & \fbh  1 \\
        \delta(t-t_0) & \fbh  \e^{-\i\w t_0} \\
        1 &\fbh  2\pi\delta(\w) \\
        \e^{\i\w_0 t} &\fbh 2\pi\delta(\w - \w_0) \\
        H(t) & \fbh  \dfrac{1}{\i\w} + \pi \delta(\omega) \\
        H(t-t_0) &\fbh  \dfrac1{\i\w}\e^{-\i\w t_0} + \pi\delta(\omega) \\
    \end{aligned}\end{equation*}
\end{mybox}
\begin{tcolorbox}[colback=cyan!10, colframe=cyan!50!black, boxrule=0.5pt, left=0pt]
    \begin{equation*}\begin{aligned}
        \sin\w_0 t &\fbh \i\pi[\delta(\w+\w_0) - \delta(\w-\w_0)] \\
        \cos\w_0 t &\fbh \pi[\delta(\w+\w_0) + \delta(\w-\w_0)] \\
    \end{aligned}\end{equation*}
    \begin{equation*}\begin{aligned}
        \e^{-\beta t}(t>0) &\fbh \dfrac{1}{\beta + \i\w} \\
        \e^{-\beta |t|} &\fbh \dfrac{2\beta}{\beta^2 + \w^2} \\
        A\e^{-\beta t^2} &\fbh \sqrt{\dfrac{\pi}{\beta}}\cdot A\e^{-\frac{\w^2}{4\beta}} \\
    \end{aligned}\end{equation*}
(上述式中 $A,\beta>0$)
\end{tcolorbox}
\end{tcbraster}

%\begin{mybox}
%第一类:与单位阶跃函数与单位冲激函数有关
%\begin{align*}
%    \delta(t) & \fbh  1 \\
%    \delta(t-t_0) & \fbh  \e^{-\i\w t_0} \\
%    1 &\fbh  2\pi\delta(\w) \\
%    \e^{\i\w_0 t} &\fbh 2\pi\delta(\w - \w_0) \\
%    H(t) & \fbh  \dfrac{1}{\i\w} + \pi \delta(\omega) \\
%    H(t-t_0) &\fbh  \dfrac1{\i\w}\e^{-\i\w t_0} + \pi\delta(\omega) \\
%\end{align*}
%第二类:三角函数与指数类递减函数
%\begin{align*}
%    \sin\w_0 t &\fbh \i\pi[\delta(\w+\w_0) - \delta(\w-\w_0)] \\
%    \cos\w_0 t &\fbh \pi[\delta(\w+\w_0) + \delta(\w-\w_0)] \\
%    \e^{-\beta t}(t>0) &\fbh \dfrac{1}{\beta + \i\w} \\
%    \e^{-\beta |t|} &\fbh \dfrac{2\beta}{\beta^2 + \w^2} \\
%    A\e^{-\beta t^2} &\fbh \sqrt{\dfrac{\pi}{\beta}}\cdot A\e^{-\frac{\w^2}{4\beta}} \\
%\end{align*}
%\end{mybox}

\subsection{狄拉克函数}
狄拉克函数的特征:$\displaystyle\int_{-\infty}^{+\infty} \delta(t-t_0) \d t = 1$。

狄拉克函数的性质:
\begin{itemize}
    \item 筛选性:$\displaystyle \int_{-\infty}^{+\infty} \delta(t-t_0) f(t)\d t = f(t_0)$
    \item 偶函数:$\delta(t) = \delta(-t)$
    \item 与单位阶跃函数联系:$\displaystyle \int_{-\infty}^t  \delta(\tau)\d\tau = H(t)$,$\dfrac{\d H(t)}{\d t} = \delta (t) $
    \item 尺度变换:$\delta (bt-a) = \dfrac1b \delta \left(t-\dfrac ab\right)$
    \item 乘以时间函数的性质:$\varphi (t) \delta (t-a) = \varphi(a)\delta (t-a)$
    \item 求导:$\displaystyle \int_{-\infty}^{+\infty}  \delta'(t-t_0) f(t)\d t = -f'(t_0)$
\end{itemize}

\subsection{傅里叶变换·性质}
以下是傅里叶变换的性质,共 8 条。
\begin{mybox}
\begin{center}
\renewcommand\arraystretch{1.2}
    \begin{tabular}{|c|c|c|}
    \hline
    序号 & 性质名 & 公式表述 \\
    \hline
    1 & 线性性 & 略 \\
    \hline
    2 & 位移性 & $\begin{aligned}
        f(t+t_0) & \fbh \e^{\i\w t_0}F(\w) \\
        \e^{-\i\w_0 t}f(t) &\fbh F(\w+\w_0) \\
    \end{aligned}$ \\[10pt]
    \hline
    3 & 相似性 & $f(at) \fbh \dfrac{1}{|a|} F\left(\dfrac{\omega}{a}\right)  $\\[8pt]
    \hline
    4 & 位移相似性 & $f(at-b)\fbh \dfrac1{|a|} \e^{-\i\frac ba\w} F\left(\dfrac{\omega}{a}\right)$\\[8pt]
    \hline
    5 & 翻转性 & $f(-t) \fbh F(-\w)$\\
    \hline
    6 & 对称性 & $F(t) \fbh 2\pi f(-\w)$\\
    \hline
    \end{tabular}
\end{center}

\begin{enumerate}[itemsep=0pt]
  \item[7.] 微分性:若 $f(t)$ 在 $(-\infty,+\infty)$ 满足狄利克雷条件,且\Emph{$\lim\limits_{|t|\to +\infty} f(t)=0$},则
  \begin{equation*}\begin{aligned}
    f'(t) &\fbh \i\w F(\w) \\
    tf(t) &\fbh \i F'(\w) \\
  \end{aligned}\end{equation*}
  \item[8.] 积分性:一般地,在任何条件下,均有
  \begin{equation*}
        \int_{-\infty}^t f(\tau)\d\tau \fbh \dfrac1{\i\w}F(\w) + \pi F(0)\delta(\w)
  \end{equation*}
  如果还有 \Emph{$\displaystyle\lim\limits_{t\to+\infty}  \int_{-\infty}^t  f(\tau)\d\tau = 0$(错题)},则
  \begin{equation*}
        \int_{-\infty}^t f(\tau)\d\tau \fbh \dfrac1{\i\w}F(\w)
  \end{equation*}
\end{enumerate}

\end{mybox}

帕塞瓦尔等式(能量积分):$\displaystyle \int_{-\infty}^{+\infty}  [f(t)]^2\d t = \dfrac1{2\pi} \int_{-\infty}^{+\infty}  |F(\w)|^2 \d\w$。

\subsection{傅里叶变换·卷积}
傅里叶卷积的定义:$\displaystyle f(t) * g(t) = \int_{-\infty}^{+\infty} f(\tau) g(t-\tau)\d\tau$。

以下是傅里叶卷积的性质,$f=f(t)$,$g=g(t)$(简写)。
\begin{mybox}
\renewcommand\arraystretch{1.2}
\begin{center}
    \begin{tabular}{|c|c|c|}
    \hline
    序号 & 性质名 & 公式表述 \\
    \hline
    1 & 交换律 & $f*g=g*f$ \\
    \hline
    2 & 结合律 & $(f*g)*h = f*(g*h)$\\
    \hline
    3 & 分配律 & $f*(g+h)=f*g+f*h$\\
    \hline
    4 & 卷积不等式 & $|f*g|\leqslant |f|*|g|$\\
    \hline
    5 & $\delta(t)$ 卷积不变性 & $f(t) * \delta(t-t_0) = f(t-t_0)$\\
    \hline
    6 & $H(t)$ 卷积 & $\displaystyle f(t) * H(t) = \int_{-\infty}^t  f(\tau)\d \tau$\\[8pt]
    \hline
    7 & 卷积的导数 & $(f*g)' = f'*g = f*g'$ \\
    \hline
    \end{tabular}
\end{center}
\end{mybox}

\paragraph{卷积定理} 若 $f(t),g(t)$ 均满足傅里叶积分定理的条件,则
\begin{align}
    f(t) * g(t) &\fbh F(\w) \cdot G(\w) \\
    2\pi f(t) \cdot g(t) &\fbh F(\w) * G(\w)
\end{align}

\subsection{傅里叶变换的应用}
如求解微分方程、积分方程等。

\section{拉普拉斯变换}
\subsection{拉普拉斯变换·概念}
\paragraph{定义} $s$ 是一个复参量,则
\begin{equation}
    F(s) = \L[f(t)] = \int_0^{+\infty} f(t)\e^{-st}\d t ,\qquad \Re(s)>0
\end{equation}

\paragraph{存在定理}
若 $f(t)$ 满足

\begin{itemize}[itemsep=0pt]
    \item 在 $t\geqslant 0$ 的任一有限区间上分段连续;
    \item 存在 $M$ 和增长指数 $c$ 使得 $|f(t)|\leqslant M\e^{ct}$,即增长是可控的。
\end{itemize}
则 $\L[f(t)]$ 在半平面 $\Re(s)>c$ 上一定存在且解析,积分 $\displaystyle \int_0^{+\infty} f(t)\e^{-st}\d t$ 在 $\Re(s)>c$ 上绝对收敛、一致收敛。

\paragraph{常见函数的拉普拉斯变换}
以下是常见函数的拉普拉斯变换列表。

\begin{mybox}
\begin{equation*}\begin{aligned}
    1,\  H(t) &\lbh \dfrac1s & \Re(s)>0 \\
    \e^{kt} &\lbh \dfrac{1}{s-k} & \Re(s)>k \\
    \delta(t) &\lbh 1 & \\
    \sin kt &\lbh \dfrac{k}{s^2+k^2} & \Re(s)>0 \\
    \cos kt &\lbh \dfrac{s}{s^2+k^2} & \Re(s)>0 \\
    t^k & \lbh \dfrac{k!}{s^{k+1}} & k\in\mathbf{N}^* \\
\end{aligned}\end{equation*}
最后一式亦可推广为 $t^k \lbh \dfrac{\varGamma(k+1)}{s^{k+1}}$($k>-1$)。
\end{mybox}

\paragraph{周期函数的拉普拉斯变换} 设 $T$ 是 $f(t)$ 的周期,且 $f(t)$ 在一周期内分段连续,$t>0$,则
\begin{equation*}
    \L[f(t)] = \dfrac{1}{1-\e^{-Ts}} \int_0^T f(t)\e^{-st}\d t, \qquad \Re(s)>0
\end{equation*}

\subsection{拉普拉斯变换·性质}
以下是拉普拉斯变换的性质,共 9 条。
\begin{mybox}
\begin{center}
\renewcommand\arraystretch{1.2}
    \begin{tabular}{|c|c|c|}
    \hline
    序号 & 性质名 & 公式表述 \\
    \hline
    1 & 线性性 & 略 \\
    \hline
    2 & 位移性 & $\begin{aligned}
        f(t-\tau)\cdot H(t-\tau) & \lbh \e^{-s\tau} F(s) &\Re(s)>c \\
        \e^{at}f(t) &\lbh F(s-a) &\Re(s-a)>c \\
    \end{aligned}$ \\[10pt]
    \hline
    3 & 相似性 & $f(at) \lbh \dfrac{1}{|a|} F\left(\dfrac sa\right)  $\\[8pt]
    \hline
    4 & 位移相似性 & $f(at-b)\fbh \dfrac1{|a|} \e^{-\frac ba s} F\left(\dfrac sa\right)$\\[8pt]
    \hline
    \end{tabular}
\end{center}

\begin{enumerate}[itemsep=0pt]
\setcounter{enumi}{4} %前面4条性质在上面的表格
    \item 微分性:若 $f(t)$ 满足拉普拉斯变换存在定理,$\Re(s)>c$(几乎等同于无条件),则有
    \begin{equation*}\begin{aligned}
        f'(t) &\lbh sF(s) - f(0) \\
        f''(t) &\lbh s^2F(s) - sf(0) - f'(0) \\
        f^{(n)}(t) &\lbh s^nF(s) - s^{n-1}\cdot f(0) -\cdots - s\cdot f^{(n-2)}(0) - f^{(n-1)}(0) \\
        tf(t) &\lbh -F'(s) \\
        t^n f(t) &\lbh (-1)^n\cdot F^{(n)}(s) \\
    \end{aligned}\end{equation*}

    \item 积分性:若 $f(t)$ 满足拉普拉斯变换存在定理,$\Re(s)>c$(几乎等同于无条件),则有
    \begin{equation*}\begin{aligned}
        \int_0^t f(\tau)\d\tau &\lbh \dfrac1s F(s) \\
        \dfrac1t f(t) &\lbh \int_s^\infty F(\xi)\d\xi \\
    \end{aligned}\end{equation*}

    \item 「双积分相等」:
    \begin{equation*}
        \int_0^\infty F(s)\d s = \int_0^{+\infty} \dfrac{f(t)}{t}\d t,\quad \Re(s)\geqslant 0
    \end{equation*}

    \item 陶伯定理·初值:若 $f(t)$ 的拉普拉斯变换存在,且 $t\to 0$ 时 $f(t)$ 及其导数存在,则初值 $f(0)$ 满足
    \begin{equation*}
        \lim\limits_{s\to\infty} sF(s) = f(0)
    \end{equation*}

    \item 陶伯定理·终值:若 $f(t)$ 的\underline{所有奇点都在复平面左半边},且有关极限存在,则终值 $f(\infty)$ 满足
    \begin{equation*}
        \lim\limits_{s\to 0} sF(s) = \lim\limits_{t\to\infty} f(t)
    \end{equation*}
\end{enumerate}

\end{mybox}

\subsection{拉普拉斯卷积}
\paragraph{拉普拉斯卷积}
\begin{equation*}
    f(t)*g(t) = \int_0^t f(\tau)g(t-\tau)\d\tau
\end{equation*}
性质:交换律、结合律、分配律、卷积不等式、数乘、微分性质、积分性质。

\paragraph{卷积定理}  若 $f(t),g(t)$ 满足拉普拉斯变换存在定理,则
\begin{equation}
    f(t)*g(t)\lbh F(\w)\cdot G(\w)
\end{equation}

\subsection{拉普拉斯逆变换}
\paragraph{反演积分公式} 若 $s_1,s_2,\dots,s_n$ 是 $F(s)$ 的所有奇点,且
\begin{itemize}[itemsep=0pt, parsep=0pt]
  \item 奇点都是孤立奇点,存在 $\beta$ 使得这些奇点都在 $\Re(s)<\beta$ 的范围内;
  \item 当 $s\to\infty$ 时,$F(s)\to 0$。
\end{itemize}
则 $F(s)$ 的拉普拉斯逆变换为
\begin{equation}
\begin{aligned}
    f(t) = \L^{-1}[F(s)] &= \dfrac{1}{2\pi\i}\int_{\beta -\i\infty}^{\beta + \i\infty} F(s)\e^{st}\d s \\
    &= \sum\limits_{k=1}^n \operatorname{Res}[F(s)\e^{st},\  s_k],\quad t>0
\end{aligned}
\end{equation}

如果 $F(s) = \dfrac{P(s)}{Q(s)}$,且 $s_k$ 为一级极点,则 $\operatorname{Res}[F(s)\e^{st},\ s_k] = \dfrac{P(s_k)}{Q'(s_k)}\e^{s_k t}$。

另外,在复变函数中,若 $z_0$ 是 $f(z)$ 的 $m$ 级极点,则其留数为
\begin{equation*}
    \mathrm{Res}[f(z),z_0] = \dfrac{1}{(m-1)!} \lim\limits_{z\to z_0} \Big[(z-z_0)^m f(z)\Big]^{(m-1)}
\end{equation*}

\paragraph{拉普拉斯逆变换方法} 以下若干种方法可用于求解拉普拉斯逆变换:
\begin{itemize}[itemsep=0pt,parsep=0pt]
    \item 反演积分公式+留数;
    \item 常见函数的拉普拉斯变换正反推;
    \item 拉普拉斯变换的性质;
    \item 部分分式法(对有理函数,写成若干个易于逆变换的函数相加的形式);
    \item 卷积定理。
\end{itemize}

\subsection{拉普拉斯变换的应用}
一个是计算广义积分,使用积分变元换序得到某些积分的值(即使并不严格)
\begin{equation*}
    \L\left[\int_a^b f(t,x)\d x\right] = \int_a^b \L[f(t,x)]\d x
\end{equation*}

另外就是求解微分、积分方程。其中解偏微分方程时,通常对 $t$ 取拉普拉斯变换,$x$ 保持不变,例如
\begin{equation*}
\begin{aligned}
    &u(t,x)\lbh U(s,x),\quad u_{xx}'' (t,x)\lbh U_{xx}'' (s,x), \\
    &u_{tt}''(t,x) \lbh s^2 U(s,x) - s u(0,x) - u_t' (0,x)\\
\end{aligned}
\end{equation*}
式中 $u_{xx}''(t,x)$ 表示对二元函数 $u(t,x)$ 连续对 $x$ 求二阶偏导数,即 $\dfrac{\partial ^2 u}{\partial x^2}$,其余量含义类推。

\section{场论}
\subsection{向量值函数} 设向量值函数 $\A(t) = P(t)\ii + Q(t)\jj + R(t)\kk$,简写为 $\A=[P,Q,R]$,则 $\A$ 的导数与积分是
\begin{equation*}
\begin{aligned}
    \A' &= [P', Q', R'] \\
    \d\A &= \A'(t) \d t \\
    \int \A\d t &= \ii\int P\d t + \jj\int Q\d t + \kk\int R\d t \\
\end{aligned}
\end{equation*}

\subsection{等值面与矢量线}
\paragraph{等值面} 是数量场 $u(x,y,z)$ 取同值的点构成的曲面,方程为 $u(x,y,z)=C$。

\paragraph{矢量线} 是矢量场 $\A=[P,Q,R]$ 每一点的切线,方程为
\begin{equation*}
    \dfrac{\d x}{P} = \dfrac{\d y}{Q} = \dfrac{\d z}{R}
\end{equation*}

\subsection{方向导数·梯度·散度·旋度}
\paragraph{方向导数} 是数量场 $u(x,y,z)$ 沿着方向 $\boldsymbol{l}$ 的导数,其中 $\boldsymbol{e}_l = \dfrac{\boldsymbol{l}}{||\boldsymbol{l}||}=[\cos\alpha,\cos\beta,\cos\gamma]$ 为单位化的方向向量。
\begin{equation*}
    \dfrac{\partial u}{\partial l} = \dfrac{\partial u}{\partial x}\cos\alpha + \dfrac{\partial u}{\partial y}\cos\beta + \dfrac{\partial u}{\partial z}\cos\gamma
\end{equation*}
或简写为梯度与单位化方向向量的点积:
\begin{equation*}
    \dfrac{\partial u}{\partial l} = \grad u\cdot \boldsymbol{e}_l
\end{equation*}
当梯度方向与已知方向同向,则方向导数最大;当反向,则方向导数最小,最大最小值都是梯度的绝对值取正/负号。

\paragraph{雅可比矩阵}
矢量场 $\A = [P,Q,R]$ 的雅可比矩阵形式为
\begin{equation*}\renewcommand\arraystretch{1.2}
    \mathrm{D}\A = \begin{vmatrix}
        \frac{\partial P}{\partial x} & \frac{\partial P}{\partial y} & \frac{\partial P}{\partial z}\\
        \frac{\partial Q}{\partial x} & \frac{\partial Q}{\partial y} & \frac{\partial Q}{\partial z}\\
        \frac{\partial R}{\partial x} & \frac{\partial R}{\partial y} & \frac{\partial R}{\partial z}\\
                      \end{vmatrix} = \begin{vmatrix}
        P_x' & P_y' & P_z' \\
        Q_x' & Q_y' & Q_z' \\
        R_x' & R_y' & R_z' \\
                                      \end{vmatrix}
\end{equation*}
可以从雅可比矩阵中观察出散度和旋度的值。

\paragraph{梯度·散度·旋度} 
引入哈密顿算子 $\nabla = \dfrac{\partial}{\partial x}\ii + \dfrac{\partial}{\partial y}\jj + \dfrac{\partial}{\partial z}\kk$:
\begin{mybox}
\begin{center}
\begin{tabular}{|c|c|c|c|c|}
    \hline
    种类 & 记号 & 空间表达式 & 平面表达式 & 结果 \\
    \hline
    梯度 & $\grad u$、$\nabla u$ & $\left[\dfrac{\partial u}{\partial x},\dfrac{\partial u}{\partial y}, \dfrac{\partial u}{\partial z}\right]$ & $\left[\dfrac{\partial u}{\partial x}, \dfrac{\partial u}{\partial y}\right]$ & 向量\\
    \hline
    散度 & $\div \A$、$\nabla\cdot\A$ & $\dfrac{\partial P}{\partial x} + \dfrac{\partial Q}{\partial y} + \dfrac{\partial R}{\partial z}$ & $\dfrac{\partial P}{\partial x}+\dfrac{\partial Q}{\partial y}$ &数量 \\
    \hline
    旋度 & $\rot \A$、$\nabla\times\A$ &{\renewcommand\arraystretch{1.2} $\begin{vmatrix}
          \ii & \jj & \kk \\
          \dfrac{\partial }{\partial x} & \dfrac{\partial }{\partial y} & \dfrac{\partial }{\partial z} \\
          P & Q & R \\
                                       \end{vmatrix}$} 
          & $\dfrac{\partial Q}{\partial x}-\dfrac{\partial P}{\partial y}$ & 向量 \\[8pt]
    \hline
\end{tabular}
\end{center}
\end{mybox}
另有一些关系,如 $\rot(\grad u)=\mathbf{0}$,$\div(\rot \A)=0$ 等。

\subsection{Gauss公式与Stokes公式}
\paragraph{Gauss公式} 设 $S$ 为分片光滑的闭曲面,$V$ 是 $S$ 所围成的\underline{单连通闭区域},函数 $P(x,y,z),Q(x,y,z),R(x,y,z)$(简写为 $P,Q,R$)在 $V$ 上有一阶连续偏导数,则
\begin{equation*}
    \oiint\limits_{S} P\d y\d z + Q\d z\d x + R\d x\d y = \iiint\limits_{V} \left(\dfrac{\partial P}{\partial x} + \dfrac{\partial Q}{\partial y} + \dfrac{\partial R}{\partial z}\right)\d V
\end{equation*}
由 $\d\boldsymbol{S} = [\d y\d z, \d z\d x, \d x\d y]$ 以及散度公式,记 $\A=[P,Q,R]$,可得更简洁的 Gauss 公式:
\begin{equation*}
    \oiint\limits_{S} \A\cdot \d\boldsymbol{S} = \iiint\limits_{V} \div \A\cdot  \d V
\end{equation*}
(注:$\A\cdot \d\boldsymbol{S}$ 中间是点积,$\div \A\cdot  \d V$ 中间是数量相乘)

\paragraph{Stokes 公式} 设 $L$ 为分段光滑的闭曲线,$S$ 是以 $L$ 为边界的有向曲面,函数 $P,Q,R$ 有一阶连续偏导数,则
\begin{equation*}\renewcommand\arraystretch{1.1}
    \oint\limits_{L} P\d x+Q\d y+R\d z = \iint\limits_{S} \begin{vmatrix}
        \d y\d z & \d z\d x & \d x\d y \\
        \dfrac{\partial}{\partial x} & \dfrac{\partial}{\partial y} & \dfrac{\partial}{\partial z} \\
        P & Q & R \\
                                                          \end{vmatrix}
\end{equation*}
由 $\d\boldsymbol{S}$ 的表达式及旋度公式,记 $\A=[P,Q,R]$,可得更简洁的Stokes公式:
\begin{equation*}
    \oint\limits_{L} \A\cdot\d \boldsymbol{l} = \iint\limits_{S} \rot\A\cdot \d\boldsymbol{S}
\end{equation*}


\subsection{典型场·势函数·力函数}
\paragraph{典型场} 在单连通域内:
\begin{itemize}[itemsep=0pt, parsep=0pt]
  \item 无旋场、有势场、保守场:$\rot\A=\mathbf{0}$;存在数量值势函数 $v$ 使得 $\A = -\grad v$。(如电场)
  \item 无源场、管形场:$\div \A=0$(如磁场)
  \item 调和场:既无源也无旋,$\div \A=0 \wedge \rot\A=\mathbf{0}$
\end{itemize}

\paragraph{势函数} 在单连通域内,若 $\A=[P,Q,R]$ 是有势场/无旋场,则势函数 $v$ 满足 $-\A = \grad v$,即可用偏积分法求出势函数 $v$:
\begin{equation}
    \dfrac{\partial v}{\partial x}=-P,\quad \dfrac{\partial v}{\partial y}=-Q, \quad \dfrac{\partial v}{\partial z}=-R
\end{equation} 
也可以用曲线积分与路径的无关性,选取点 $M_0(x_0,y_0,z_0)$ 至 $M(x,y,z)$ 的路径积分,也可求势函数。其中起始点 $M_0$ 通常取作 $(0,0,0)$。
\begin{equation*}
    v(x,y,z) = -\int_{x_0}^x P(x,y_0,z_0)\d x - \int_{y_0}^y Q(x,y,z_0)\d y - \int_{z_0}^z R(x,y,z)\d z
\end{equation*}

\paragraph{力函数} 若 $\A=[P,Q]$ 是平面调和场,则 $\div\A=0$,即 $\dfrac{\partial P}{\partial x} + \dfrac{\partial Q}{\partial y} = 0$,即可引入新的有势场\underline{$\boldsymbol B = [-Q,P]$},且存在力函数 $u$ 使得 $\boldsymbol B=\grad u$,即有
\begin{equation}
    \dfrac{\partial u}{\partial x}=-Q,\quad \dfrac{\partial u}{\partial y}=P
\end{equation}
也可选取 $(x_0,y_0)$ 至 $(x,y)$ 的路径积分来求力函数:
\begin{equation*}
    u(x,y) = \int_{x_0}^x -Q(x,y_0)\d x + \int_{y_0}^y P(x,y)\d y
\end{equation*}

\paragraph{势、力函数的关系} 事实上,势函数 $v$ 和力函数 $u$ 都是调和函数:
\begin{equation*}
    \Delta u = \dfrac{\partial^2 u}{\partial x^2} + \dfrac{\partial^2 u}{\partial y^2} = 0,\quad \Delta v = \dfrac{\partial^2 v}{\partial x^2} + \dfrac{\partial^2 v}{\partial y^2}=0
\end{equation*}
而且因为 $u,v$ 满足复变函数的「柯西—黎曼方程」:
\begin{equation*}
    \dfrac{\partial u}{\partial x} = \dfrac{\partial v}{\partial y},\quad \dfrac{\partial u}{\partial y}= -\dfrac{\partial v}{\partial x}
\end{equation*}
因此 $u,v$ 也是共轭调和函数,由它们之中的其中一个可求出另一个。

注:上式中 $\Delta=\dfrac{\partial^2}{\partial x^2} + \dfrac{\partial^2}{\partial y^2}$ 为拉普拉斯算子,$\Delta u$ 称为 $u$ 的调和量。

\newpage
\backgroundsetup{contents=\includegraphics{下半示例.png}, center, scale=1, angle=0, opacity=1}
\BgThispage
\subsection{相对冷僻的概念}
注:本节内容相对冷僻,可以视情况忽略。
\paragraph{通量、散度的原始定义} 
\begin{equation*}
    \varPhi = \iint\limits_{S} \A\cdot \d\boldsymbol{S}
\end{equation*}
\begin{equation*}
    \div \A(M) = \lim\limits_{\Delta V\to M} \dfrac{\displaystyle \oiint\limits_{\Delta S} \A(M)\cdot \d\boldsymbol{S}}{\Delta V}
\end{equation*}

\paragraph{环量密度、旋度的原始定义}
\begin{equation*}
    \mu_n = \lim\limits_{\Delta S\to M} \dfrac{\displaystyle\oint\limits_{\Delta L} \A\cdot\d\boldsymbol{l}}{\Delta S}
\end{equation*} 
\begin{equation*}
    \mu_n = \rot\A\cdot\boldsymbol{n}_0
\end{equation*}
其中,曲面 $\Delta S$ 的外法线方向为 $\boldsymbol{n}$,其单位化向量为 $\boldsymbol{n}_0$。

\paragraph{矢势量} 在面单连通域内矢量 $\A$ 为管形场的充要条件是,它是另一个矢量场 $\boldsymbol{B}$ 的旋度。$\boldsymbol{B}$ 称为矢势量。若设 $\A=[P,Q,R]$,$\boldsymbol{B}=[U,V,W]$,则
\begin{equation*}
    U=\int_{z_0}^z Q(x,y,z)\d z-\int_{y_0}^y R(x,y,z_0)\d y,\quad V=-\int_{z_0}^z P(x,y,z)\d z,\quad W=1
\end{equation*}

\end{document} 