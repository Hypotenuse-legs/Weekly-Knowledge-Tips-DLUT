\documentclass[UTF8]{ctexart}
\usepackage{amsmath}
\usepackage{amssymb}
\usepackage{background}
\usepackage{booktabs}
\usepackage{caption}
\usepackage{graphicx}
\usepackage{geometry}
\usepackage{float}
\usepackage{fontspec}
\usepackage{mathptmx} %mathptmx和times结合使得公式使用times new roman字体
\usepackage{times}
\usepackage{xcolor}


\geometry{a5paper, top=0.1cm, left=1cm, right=1cm, bottom=1cm}
\setCJKmainfont[BoldFont={汉仪文黑-85W},ItalicFont={方正苏新诗柳楷简体}]{汉仪文黑-55W}
\setfontfamily\Issue{Century Schoolbook}
\newCJKfontfamily\TitleFont{思源宋体 CN Heavy}
\newfontfamily\timesnewroman{Times New Roman}
\setmainfont{Times New Roman}
\captionsetup{font=small, labelfont=bf}

\newcommand\Black[1]{\textcolor[gray]{0.3}{#1}}
\newcommand\Brown[1]{\textcolor[HTML]{998A4E}{#1}}
\newcommand\Emph[1]{\colorbox{red!20!}{\textcolor{red!80!black}{#1}}}
\newcommand\Correct[1]{\colorbox{green!20}{\textcolor{green!50!black}{#1}}}
\newcommand\Mathemph[1]{\text{\textcolor{green!60!black}{$#1$}}}

\newcommand\h{\text{\textcolor{blue}{\,$\wedge$\,}}} %合取
\newcommand\x{\text{\textcolor{red!70!black}{\,$\vee$\,}}} %析取
\newcommand\f{\neg} %非
\newcommand\is{\Leftrightarrow\ } %等价

%这4个信息随“刊号”更新
\newcommand\IssueNumber{03}
\newcommand\Date{2023-9-25}
%\newcommand\Contributer{@金光日}
\newcommand\Subject{离散数学}


\begin{document}
\backgroundsetup{contents=\includegraphics{上半示例.png}, center, scale=1, angle=0, opacity=1}
\BgThispage
\begin{center}
{\scriptsize\Issue \textcolor[HTML]{C8BA83}{WEEKLY TIPS}}

{\Huge\bfseries\TitleFont \Black{知\ 识\ 小\ 料}}

\vspace{-0.1cm}
{\footnotesize \Brown{「电计 2203 班」周常规知识整理共享}}
\end{center}

\vspace{-0.5cm}

\begin{figure}[H]
\hspace{1cm}
\begin{minipage}[t]{0.3\textwidth}
\centering
    \Brown{ISSUE.}

    \vspace{-0.6cm}
    \Huge \Issue\slshape\bfseries\Black{\IssueNumber}
\end{minipage}
\hfill
\begin{minipage}[t]{0.3\textwidth}
\centering
    \Brown{日期:\Date} \\
%\vspace{-0.1cm}
%    \Brown{贡献者:\Contributer} \\
\vspace{-0.1cm}
    \Brown{学科:\Subject} \\
\end{minipage}
\hspace{0.8cm}
\end{figure}

%此处以后填写正文
\textcolor{cyan!50!black}{
求出 $(P \to (Q\h R))\h (\f P \to (\f Q \h \f R))$ 的主析取范式和主合取范式,并判断公式类型。
}

由于中间是“合取”,所以这里先求主合取范式。对原式作如下等值演算:
\begin{equation}
\begin{aligned}
& (P \to (Q\h R))\h (\f P \to (\f Q \h \f R)) \\ 
\iff & (\f P\x (Q\h R)) \h (P \x (\f Q\h \f R)) \\
\iff & ((\f P\x Q)\h (\f P\x R)) \h ((P\x \f Q)\h (P\x \f R)) \\ 
\iff & (\f P\x Q) \h (\f P\x R) \h (P\x \f Q) \h (P\x \f R) \text{\textcolor{cyan}{(去外层括号)}}\\
\end{aligned}
\end{equation}

到此我们得到了合取范式,现在要变为主合取范式——对每一个简单析取式,需要补进所有未出现的命题变元,以获得极大项。如第一个简单析取式 $\f P\x Q$,可以补进命题变元 $R$ 后用分配律展开之:
\begin{equation}
\begin{aligned}
    \f P\x Q & \iff \f P\x Q\x (R\h \f R) \\
    &\iff (\f P\x Q\x R)\h (\f P\x Q\x \f R) \\
\end{aligned}
\end{equation}

因此对得到的 4 个简单析取式作类似的操作,得到的极大项如下表 \ref{tab:jidaxiang} 所示。
\begin{table}[htb]
    \centering
    \begin{tabular}{ccc}
    \toprule
    $\f P\x Q$ & $\iff$ & $(\f P\x Q\x \Mathemph{R})\h (\f P\x Q\x \Mathemph{\f R})$ \\
    $\f P\x R$ & $\iff$ & $(\f P\x \Mathemph{Q}\x R)\h (\f P\x \Mathemph{\f Q}\x R)$ \\
    $ P\x \f Q$ & $\iff$ & $(P\x \f Q\x \Mathemph{R})\h (P\x \f Q\x \Mathemph{\f R})$ \\
    $ P\x \f R$ & $\iff$ & $(P\x \Mathemph{Q}\x \f R)\h (P\x \Mathemph{\f Q}\x \f R)$ \\
    \bottomrule 
    \end{tabular}
    \caption{对 4 个简单析取式展开得到的极大项表}\label{tab:jidaxiang}
\end{table}
重复出现的只计一次,得到原式的主合取范式为:
\begin{equation}\begin{split}
& (P\x Q\x \f R)\h (P\x \f Q\x R)\h (P\x \f Q\x \f R) \\
\h & (\f P\x Q\x R)\h (\f P\x Q\x \f R)\h (\f P\x \f Q\x R) \\
\iff & M_{001}\h M_{010}\h M_{011} \h M_{100} \h M_{101}\h M_{110} \\
\iff & M_1\h M_2\h M_3 \h M_4\h M_5\h M_6 \text{\quad (可写作 $\displaystyle\bigwedge\limits_{i=1}^6 M_i$)}
\end{split}\end{equation}



\newpage
\backgroundsetup{contents=\includegraphics{下半示例.png}, center, scale=1, angle=0, opacity=1}
\BgThispage
\phantom{...}
\vspace{0.5cm}

需要注意的是,在对极大项二进制编号时,原子命题记 0,否定记 1。这一点与极小项不同。

下面我们再来求主析取范式。对原式作如下等值演算:
\begin{equation}
\begin{aligned}
& (P \to (Q\h R))\h (\f P \to (\f Q \h \f R)) \\
\iff & \Mathemph{(\f P\x (Q\h R))} \h (P \x (\f Q\h \f R)) \\
\iff & \big(\Mathemph{(\f P\x (Q\h R))}\h P\big) \x \big(\Mathemph{(\f P\x (Q\h R))}\h (\f Q\h \f R)\big) \text{\textcolor{cyan}{(分配律)}}\\
\iff & \big((\f P\h P)\x (Q\h R\h P)\big) \x \big( (\f P\h \f Q\h \f R) \x (Q\h R\h \f Q\h \f R) \big) \\
\iff & (P\h Q\h R)\x (\f P\h \f Q\h \f R) \\
\iff & m_{111} \x m_{000} \\
\iff & m_7 \x m_0 \\
\end{aligned}
\end{equation}
可以看到,求主析取范式时,大量使用了分配律,而且式子结构不美观,容易出错。事实上,当我们得到主合取范式为 $M_1\h M_2\h M_3 \h M_4\h M_5\h M_6$ 之后,便可以直接推出主析取范式为 $m_0\x m_7$,因为这两种范式有「互补」关系。

最后就是判断公式类型,一般为永真式、永假式、可满足式的三者之一。本题由范式分析的结果很容易得到类型为可满足式。

\textcolor{cyan!80!black}{【结论】主析取范式为 $m_0\x m_7$,主合取范式为 $M_1\h M_2\h M_3 \h M_4\h M_5\h M_6$,公式类型为可满足式。}

\textcolor{cyan!80!black}{【点评】本题是一道求主范式的经典试题,可以由式子自身的结构选择求取某一种范式,然后通过「互补」关系直接写出另一种范式。}

\end{document}