\documentclass[UTF8]{ctexart}
\usepackage{amsmath}
\usepackage{amssymb}
\usepackage{booktabs}
\usepackage{background}
\usepackage{enumitem}
\usepackage{float}
\usepackage{fontspec}
\usepackage{fourier}
\usepackage{geometry}
\usepackage{longtable}
\usepackage{xcolor}

\geometry{a5paper, top=0.1cm, left=1cm, right=1cm, bottom=0.8cm, footskip=0.1cm}
\setCJKmainfont[BoldFont={汉仪文黑-85W},ItalicFont={方正苏新诗柳楷简体}]{汉仪文黑-55W}
\setfontfamily\Issue{Century Schoolbook}
\newCJKfontfamily\TitleFont{思源宋体 CN Heavy}
\newfontfamily\timesnewroman{Times New Roman}
%\reversemarginpar

%\CTEXsetup[format = {\centering\bfseries\large}, beforeskip = 3pt, afterskip = 3pt]{section}
\CTEXsetup[format = {\color{cyan!50!black}\bfseries\large}]{subsection}

\colorlet{darkcyan}{cyan!50!black}
\newcommand\Black[1]{\textcolor[gray]{0.3}{#1}}
\newcommand\Brown[1]{\textcolor[HTML]{998A4E}{#1}}
\newcommand\Emph[1]{\colorbox{green!10}{\textcolor{green!30!black}{#1}}}
\newcommand\Notes[1]{\textcolor{yellow!50!black}{\small #1}}
\newcommand\Example[1]{\textcolor{cyan!70!black}{\small #1}}
\renewcommand\arraystretch{1.8}

\renewcommand\pi{\text{\timesnewroman π}}
\renewcommand\d{\mathrm{d}}
\newcommand\B{\boldsymbol{B}}
\renewcommand\S{\boldsymbol{S}}
\renewcommand\l{\boldsymbol{l}}
\newcommand\M{\boldsymbol{M}}
\renewcommand\H{\boldsymbol{H}}
\newcommand\dds{\mathcal{E}} %电动势
\newcommand\delt{\text{\timesnewroman δ}}

\newcommand\IssueNumber{17}
\newcommand\Date{2024-1-10}
%\newcommand\Contributer{@金光日}
\newcommand\Subject{大学物理 A2}


\begin{document}
\backgroundsetup{contents=\includegraphics{示例.png}, center, scale=1, angle=0, opacity=1}
\BgThispage
\begin{center}
{\scriptsize\Issue \textcolor[HTML]{C8BA83}{WEEKLY TIPS}}

{\Huge\bfseries\TitleFont \Black{知\ 识\ 小\ 料}}

\vspace{-0.1cm}
{\footnotesize \Brown{「电计 2203 班」周常规知识整理共享}}
\end{center}

\vspace{-0.5cm}

\begin{figure}[H]
\hspace{1cm}
\begin{minipage}[t]{0.3\textwidth}
\centering
    \Brown{ISSUE.}

    \vspace{-0.6cm}
    \Huge \Issue\slshape\bfseries\Black{\IssueNumber}
\end{minipage}
\hfill
\begin{minipage}[t]{0.35\textwidth}
\centering
    \Brown{日期:\Date} \\
%\vspace{-0.1cm}
%    \Brown{贡献者:\Contributer} \\
\vspace{-0.1cm}
    \Brown{学科:\Subject} \\
\end{minipage}
\hspace{0.8cm}
\end{figure}

{\color{cyan!50!black}

一油轮漏出的油($n_2=1.20$)污染了海域,在海水($n_3=1.30$)表面形成了一层薄薄的油污(厚度 460nm)。
\begin{enumerate}[itemsep=0pt,parsep=0pt]
  \item 如果太阳位于海域上空,直升机驾驶员从机上向下观察,则他将观察到油层呈什么颜色?
  \item 如果潜水员位于该区域水下,则他将观察到油层呈什么颜色?
\end{enumerate}
}

\begin{figure}[htb]
    \centering
    \newcount\WL \unitlength.75pt
    \begin{picture}(460,30)(355,-10)
        \sffamily \tiny \linethickness{1.25\unitlength} \WL=380
        \multiput(380,0)(1,0){400}%
        {{\color[wave]{\the\WL}\line(0,1){30}}\global\advance\WL1}
        \linethickness{0.25\unitlength}\WL=380
        \multiput(380,0)(20,0){20}%
        {\picture(0,0)
        \line(0,-1){5} \multiput(5,0)(5,0){3}{\line(0,-1){2.5}}
        \put(0,-10){\makebox(0,0){\the\WL}}\global\advance\WL20
        \endpicture}
    \end{picture}

    \caption{可见光光谱图(数字表示波长,单位:nm)}\label{fig:guangpu}
\end{figure}

【解答】本题有 $n_1=1$(空气)、$n_2$、$n_3$ 三个折射率,推知本题考察薄膜干涉。记 $h=460\mathrm{nm}$。

\begin{enumerate}
    \item 水上观察——\textbf{反射光}干涉相长。$n_1<n_2<n_3$——无半波损失。
        \begin{equation}\label{eq:1}
            \delta = 2hn_2 = k\lambda
        \end{equation}
        解出 $\lambda = \dfrac{2hn_2}{k} = \dfrac{1104}{k}\mathrm{nm}$。取 $k=2,\lambda=\textcolor{green!70!black}{552\mathrm{nm}}$。
    \item 水下观察——\textbf{透射光}干涉相长。与反射光相反——有半波损失。
        \begin{equation}\label{eq:2}
            \delta = 2hn_2 + \dfrac{\lambda}2 =  k\lambda
        \end{equation}
        解出 $\lambda = \dfrac{2hn_2}{k-\frac12} = \dfrac{1104}{k-\frac12}\mathrm{nm}$。取 $k=2,\lambda=\textcolor{red}{736\mathrm{nm}}$;$k=3,\lambda=\textcolor{blue!90!white}{441.6\mathrm{nm}}$。
\end{enumerate}

\textcolor{cyan!80!black}{【结论】1. 绿色;2. 红色和蓝色(紫色)。}

\textcolor{cyan!80!black}{【点评】本题考察薄膜干涉,但提示相对隐晦,可能难以识别。本题的坑在于反射光与透射光是否有半波损失的问题,做题时应留意。}

\end{document}