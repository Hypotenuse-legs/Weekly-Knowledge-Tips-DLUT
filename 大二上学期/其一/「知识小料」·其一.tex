\documentclass[UTF8]{ctexart}
\usepackage{amsmath}
\usepackage{amssymb}
\usepackage{graphicx}
\usepackage{geometry}
\usepackage{float}
\usepackage{fourier}
\usepackage{fontspec}
\usepackage{background}
\usepackage{xcolor}


\backgroundsetup{contents=\includegraphics{示例.png}, center, scale=1, angle=0, opacity=1}
\geometry{a5paper, top=0.1cm, left=1cm, right=1cm, bottom=2cm}
\setCJKmainfont[BoldFont={汉仪文黑-85W},ItalicFont={方正苏新诗柳楷简体}]{汉仪文黑-55W}
\setfontfamily\Issue{Century Schoolbook}
\newCJKfontfamily\TitleFont{思源宋体 CN Heavy}
\newfontfamily\timesnewroman{Times New Roman}

\newcommand\Black[1]{\textcolor[gray]{0.3}{#1}}
\newcommand\Brown[1]{\textcolor[HTML]{998A4E}{#1}}
\newcommand\Emph[1]{\colorbox{red!20!}{\textcolor{red!80!black}{#1}}}
\newcommand\Correct[1]{\colorbox{green!20}{\textcolor{green!50!black}{#1}}}
\newcommand\mypi{\text{\timesnewroman π}}
\newcommand\ddd{\mathrm{d}}
\newcommand\iii{\mathrm{i}}
\newcommand\e{\mathrm{e}}
%这4个信息随“刊号”更新
\newcommand\IssueNumber{01}
\newcommand\Date{2023-9-11}
%\newcommand\Contributer{@金光日}
\newcommand\Subject{积分变换与场论}


\begin{document}
\begin{center}
{\scriptsize\Issue \textcolor[HTML]{C8BA83}{WEEKLY TIPS}}

{\Huge\bfseries\TitleFont \Black{知\ 识\ 小\ 料}}

\vspace{-0.1cm}
{\footnotesize \Brown{「电计 2203 班」周常规知识整理共享}}
\end{center}

\vspace{-0.5cm}

\begin{figure}[H]
\hspace{1cm}
\begin{minipage}[t]{0.3\textwidth}
\centering
    \Brown{ISSUE.}

    \vspace{-0.6cm}
    \Huge \Issue\slshape\bfseries\Black{\IssueNumber}
\end{minipage}
\hfill
\begin{minipage}[t]{0.3\textwidth}
\centering
    \Brown{日期:\Date} \\
%\vspace{-0.1cm}
%    \Brown{贡献者:\Contributer} \\
\vspace{-0.1cm}
    \Brown{学科:\Subject} \\
\end{minipage}
\hspace{0.8cm}
\end{figure}

%此处以后填写正文
\textcolor{cyan!50!black}{
求函数 $f(t) = \left\{\begin{aligned} &1-|t|, \quad & t^2<1 \\ &0, & t^2>1\end{aligned}\right. $ 的傅里叶积分。
}

\hspace{1cm}

所谓「傅里叶积分」,即是对函数作傅里叶变换再做逆变换的形式。也就是《积分变换与场论》课本的第 16 页:
\textcolor{red!80!black}{
\begin{equation}\label{eq:origin}
    \dfrac{1}{2\mypi} \int_{-\infty}^\infty \left[\int_{-\infty}^\infty f(t) \e^{-\iii \omega t}\ddd t\right]\cdot \e^{\iii \omega t}\ddd \omega
\end{equation}}

可以一步一步来,比如先做傅里叶正变换(这里写详细一些)——
\begin{equation}\label{eq:towards}
    \begin{aligned}
    F(\omega) &= \mathcal{F}[f(t)] = \int_{-\infty}^\infty f(t)\e^{-\iii\omega t}\ddd t  \\
    &= \int_{-\infty}^\infty f(t)\cdot [\cos(-\omega t)+\iii\sin(-\omega t)]\ddd t\\
    &= \int_{-\infty}^\infty f(t)\cdot \cos(\omega t)\ddd t \qquad\text{\color{cyan}(奇函数部分在对称区间积分得 0)}\\
    &= 2\int_0^\infty (1-|t|)\cdot\cos(\omega t)\ddd t \\
    &= 2\int_0^1 (1-t)\cdot\cos(\omega t)\ddd t \qquad\text{\color{cyan}(注意这里把积分上限换成了 1)}\\
    &= 2\cdot\dfrac{1}{\omega} \int_0^1 (1-t)\cdot \ddd\sin(\omega t) \qquad\text{\color{cyan}(准备分部积分)}\\
    &= \dfrac{2}{\omega}\left[(1-t)\sin(\omega t)|_{t=0}^{t=1}  -  \int_0^1 \sin(\omega t)\cdot (-1)\ddd t\right]\\
    &= \dfrac{2}{\omega} \int_0^1 \sin(\omega t)\ddd t \\
    &= \dfrac{2(1-\cos\omega)}{\omega^2} \\
    \end{aligned}
\end{equation}

然后再做逆变换——

\newpage
\phantom{...}
\vspace{3.5cm}
\begin{equation}\label{eq:backwards}
\begin{aligned}
    f(t) &= \mathcal{F}^{-1}[F(\omega)] = \dfrac{1}{2\mypi} \int_{-\infty}^\infty F(\omega) \e^{\iii\omega t}\ddd\omega \\
    &= \dfrac{1}{\mypi} \int_{-\infty}^\infty \dfrac{1-\cos\omega}{\omega^2}
    \cdot  \text{\Correct{[cos($\omega$\textit{t}) + i\ sin($\omega$\textit{t})]}}  \ddd\omega\\
    &= \dfrac{1}{\mypi} \int_{-\infty}^\infty \dfrac{(1-\cos\omega)\cdot\cos\omega t}{\omega^2}\ddd \omega \\
    &= \dfrac{2}{\mypi} \int_{0}^\infty \dfrac{(1-\cos\omega)\cdot\cos\omega t}{\omega^2}\ddd \omega
\end{aligned}
\end{equation}

算到这里,结论已经显现。

\vspace{0.7cm}
\textcolor{cyan!80!black}{【结论】$\displaystyle\dfrac{2}{\mypi} \int_{0}^\infty \dfrac{1-\cos\omega}{\omega ^ 2} \cos\omega t\,\ddd \omega$。}

\vspace{0.7cm}
\textcolor{cyan!80!black}{【点评】这是一道经典的傅里叶积分问题,求解思路比较明确,主要是求分部积分的时候容易弄错,或者不记得把积分上限换成 1,导致拿广义积分来积半天也不出结果。另外,这里用到了「奇函数在对称区间积分值为 0」的性质\Correct{简化了}计算过程。}


\end{document} 