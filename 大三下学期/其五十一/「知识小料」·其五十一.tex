\documentclass[UTF8]{ctexart}
\usepackage{amsmath}
\usepackage{amssymb}
\usepackage{booktabs}
\usepackage{background}
\usepackage{caption,subcaption}
\usepackage{enumitem}
\usepackage{fancyhdr}
\usepackage{float}
\usepackage{fontspec}
%\usepackage{fourier}
\usepackage{geometry}
\usepackage{listings}
\usepackage{pifont}
\usepackage{tasks}
\usepackage{tikz}
\usetikzlibrary{arrows.meta}
\usepackage{xcolor}

\geometry{a5paper, top=0.1cm, left=1cm, right=1cm, bottom=1cm, footskip=0.1cm}
\setCJKmainfont[BoldFont={汉仪文黑-85W},ItalicFont={方正苏新诗柳楷简体}]{汉仪文黑-55W}
\setfontfamily\Issue{Century Schoolbook}
\setfontfamily\Genshin{Genshin Teyvat Lingua Franca}
\newCJKfontfamily\TitleFont{思源宋体 CN Heavy}
\newfontfamily\timesnewroman{Times New Roman}

\pagestyle{fancy}
\fancyhf{}
\cfoot{\sffamily\footnotesize{-\ \thepage\ -}}
\CTEXsetup[format={\bfseries\large}]{paragraph}
%\CTEXsetup[format = {\centering\bfseries\large}, beforeskip = 3pt, afterskip = 3pt]{section}

\colorlet{darkcyan}{cyan!50!black}
\newcommand\Black[1]{\textcolor[gray]{0.3}{#1}}
\newcommand\Brown[1]{\textcolor[HTML]{998A4E}{#1}}
\newcommand\Emph[1]{\colorbox{green!10}{\textcolor{green!30!black}{#1}}}
\newcommand\Notes[1]{\textcolor{yellow!50!black}{\small #1}}
\newcommand\Example[1]{\textcolor{cyan!70!black}{\small #1}}
\settasks{ label=(\Alph*), item-format=\color{darkcyan}, label-format=\color{darkcyan}, label-offset=0.7em}


\lstset{
    basicstyle=\small\ttfamily, %注意行末有逗号!
    keywordstyle=\bfseries\color{blue!70!black},
    commentstyle=\color{cyan!90!black},
    stringstyle=\color{green!40!black},
    columns=flexible,
    numbers=left,
    numberstyle=\footnotesize,
    escapechar=`,
    frame=shadowbox,
    %rulesepcolor=\color{red!20!blue!20!green!20}
    backgroundcolor=\color{cyan!5!white},
    language = SQL,
    tabsize = 4,
    breaklines = true,
}

% -----------------本文档专用-----------------
\newcommand\drawrec[3]{ % 画矩形,第一个数字表示进程号,第二个数字表示开始时间,第三个数字表示结束时间
    \ifnum #1=1
        \filldraw[fill=cyan] (#2, 2.3) rectangle (#3, 2.7);
    \fi
    \ifnum #1=2
        \filldraw[fill=violet] (#2, 1.3) rectangle (#3, 1.7);
    \fi
    \ifnum #1=3
        \filldraw[fill=red] (#2, 0.3) rectangle (#3, 0.7);
    \fi
}
\newcommand\marking[2]{ %标线,第一个数字表示标记的时间,第二个数字表示标记的进程号
    \draw[dashed] (#1, 0) -- (#1, 4-#2-0.2);
    \node[below, font=\small] at (#1,0) {#1};
}
% ---------------------------------------------

\newcommand\IssueNumber{51}
\newcommand\Date{2025-4-13}
%\newcommand\Contributer{@金光日}
\newcommand\Subject{操作系统}
\newcommand\Source{2023 考研 408 真题}


\begin{document}
\backgroundsetup{contents=\includegraphics{上半示例.png}, center, scale=1, angle=0, opacity=1}
\BgThispage
\begin{center}
%{\scriptsize\Issue \textcolor[HTML]{C8BA83}{\Genshin WEEKLY TIPS}}
\phantom{...}

{\Large\textcolor{brown!40!white}{\makebox[10cm][s]{\Genshin WEEKLY KNOWLEDGE TIPS}}}

\vspace{-2em}

{\Huge\bfseries\TitleFont \Black{知\ 识\ 小\ 料}}


\vspace{-0.1cm}
{\footnotesize \Brown{「电计 2203 班」周常规知识整理共享}}
\end{center}

\vspace{-0.5cm}


\begin{figure}[H]
\hspace{1cm}
\begin{minipage}[t]{0.3\textwidth}
\centering
    \Brown{\Genshin ISSUE}

    \vspace{-0.6cm}
    \Huge \Issue\slshape\bfseries\Black{\IssueNumber}
\end{minipage}
\hfill
\begin{minipage}[t]{0.35\textwidth}
\centering
    \Brown{日期:\Date} \\
%\vspace{-0.1cm}
%    \Brown{贡献者:\Contributer} \\
\vspace{-0.1cm}
    \Brown{学科:\Subject} \\
\vspace{-0.1cm}
    \Brown{来源:\Source} \\
\end{minipage}
\hspace{0.8cm}
\end{figure}

{\color{darkcyan}
进程 P1、P2 和 P3 进入就绪队列的的时刻、优先值(越大优先权越高)以及 CPU 的执行时间如下表所示。

\begin{table}[htb]
    \color{darkcyan}
    \centering
    \begin{tabular}{cccc}
    \toprule
        进程名 & 进入就绪队列的时刻 & 优先级 & CPU 执行时间 \\
    \midrule
        P1 & 0ms & 1 & 60ms \\
        P2 & 20ms & 10 & 42ms \\
        P3 & 30ms & 100 & 13ms \\
    \bottomrule
    \end{tabular}
\end{table}

系统采用基于优先权的抢占式CPU调度算法,从0ms时刻开始进行调度,则P1、P2和P3的平均周转时间为

\begin{tasks}(4)
    \task 60ms
    \task 61ms
    \task 70ms
    \task 71ms
\end{tasks}
}

这是一道经典的调度问题。基于优先权调度,而且优先权数字大的进程会抢占CPU。调度时空图如图 \ref{fig:answer} 所示。

\begin{itemize}[itemsep=0pt,parsep=2pt]
    \item 0ms:P1 加入就绪队列。P1 一直调度,持续到 20ms;
    \item 20ms:P2 加入就绪队列,其优先级比 P1 高,因此\Emph{抢占},持续到 30ms;
    \item 30ms:P3 加入就绪队列,其优先级比 P2 高,因此\Emph{抢占},直至P3调度结束;
    \item 43ms:P3 离开就绪队列,P2 继续调度(剩余 32ms),直至P2调度结束;
    \item 75ms:P2 离开就绪队列,P1 继续调度(剩余 40ms),直至P1调度结束;
    \item 115ms:P1 离开就绪队列,调度完成。
\end{itemize}

求「周转时间」,公式是\Emph{「$\text{周转时间} = \text{完成时刻} - \text{进入时刻}$」}(如图 \ref{fig:time} 所示)。P1 的周转时间为 $115-0=115$ms,P2 的周转时间为 $75-20=55$ms,P3 的周转时间为 $43-30=13$ms。因此平均周转时间为 $(115+55+13)\div 3 = 61$ms。

\newpage
\begin{figure}[htb]
    \centering
    \begin{tikzpicture}[>=Stealth, xscale=0.1]
        \draw[->] (0,0) -- (120,0) node[above, font=\small] {$t$/ms};
        \draw (0,0) -- (0, 3) node[above, font=\small] {进程};
        \node[below left] at (0,0) {$O$};
        \node[left] at (0, 0.5) {P3};
        \node[left] at (0, 1.5) {P2};
        \node[left] at (0, 2.5) {P1};
        \drawrec{1}{0}{20};
        \drawrec{2}{20}{30};
        \drawrec{3}{30}{43};
        \drawrec{2}{43}{75};
        \drawrec{1}{75}{115};
        \marking{20}{1};
        \marking{30}{2};
        \marking{43}{2};
        \marking{75}{1};
        \marking{115}{1};
        \node[right, cyan] at (20, 2.5) {$\frac{20}{60}$};
        \node[right, violet] at (30, 1.5) {$\frac{10}{42}$};
        \node[right, red] at (43, 0.5) {\ding{51}};
        \node[right, violet] at (75, 1.5) {\ding{51}};
        \node[right, cyan] at (115, 2.5) {\ding{51}};
    \end{tikzpicture}
    \caption{调度时空图}\label{fig:answer}
\end{figure}

\begin{figure}[htb]
    \centering
    \begin{tikzpicture}[>=Stealth, xscale=0.8]
        \filldraw[draw=cyan, fill=cyan] (0,0) rectangle (3, 0.4);
        \filldraw[draw=cyan, fill=violet] (3,0) rectangle (8,0.4);
        \draw[->] (0, 1.4) node[above,font=\small] {进入(到达)} -- (0, 0.4);
        \draw[->] (3, 1.4) node[above,font=\small] {开始} -- (3, 0.4);
        \draw[->] (8, 1.4) node[above,font=\small] {完成} -- (8, 0.4);
        \draw (0,0) -- (0,-1.5);
        \draw (3,0) -- (3,-1);
        \draw (8,0) -- (8,-1.5);
        \draw[<->, violet] (3,-0.7) -- (8,-0.7) node[midway,above,font=\small] {服务时间};
        \draw[<->, blue] (0, -1.3) -- (8, -1.3) node[midway,above,font=\small] {周转时间};
        
        \node[right,font=\small] at (9,0) {$\text{带权周转} = \dfrac{\text{\textcolor{blue}{周转时间}}}{\text{\textcolor{violet}{服务时间}}} \geqslant 1$};
    \end{tikzpicture}
    \caption{周转时间图解}\label{fig:time}
\end{figure}

\backgroundsetup{contents=\includegraphics{下半示例.png}, center, scale=1, angle=0, opacity=1}
\BgThispage
\vspace{1em}

{\color{cyan!80!black}
【结论】B

【点评】这是一道调度算法的选择题。常见的调度算法包括「先来先服务」FCFS、「短作业优先」SJF、「时间片轮转」RR、「优先级调度」等。画出时空图,明确完成时间、周转时间、带权周转时间等指标的求法,是解决此题的关键。
}



\end{document}