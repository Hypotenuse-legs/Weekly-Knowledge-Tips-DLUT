\documentclass[UTF8]{ctexart}
\usepackage{amsmath}
\usepackage{amssymb}
\usepackage{booktabs}
\usepackage{background}
\usepackage{caption,subcaption}
\usepackage{enumitem}
\usepackage{fancyhdr}
\usepackage{float}
\usepackage{fontspec}
%\usepackage{fourier}
\usepackage{geometry}
\usepackage{listings}
\usepackage{makecell}
\usepackage{tikz}
\usepackage{ulem}
\usetikzlibrary{arrows.meta}
\usepackage{xcolor}

\geometry{a5paper, top=0.1cm, left=1cm, right=1cm, bottom=1cm, footskip=0.1cm}
\setCJKmainfont[BoldFont={汉仪文黑-85W},ItalicFont={方正苏新诗柳楷简体}]{汉仪文黑-55W}
\setfontfamily\Issue{Century Schoolbook}
\setfontfamily\Genshin{Genshin Teyvat Lingua Franca}
\newCJKfontfamily\TitleFont{思源宋体 CN Heavy}
\newfontfamily\timesnewroman{Times New Roman}
%\reversemarginpar

\pagestyle{fancy}
\fancyhf{}
\cfoot{\sffamily\footnotesize{-\ \thepage\ -}}
\CTEXsetup[format={\bfseries\large}]{paragraph}
%\CTEXsetup[format = {\centering\bfseries\large}, beforeskip = 3pt, afterskip = 3pt]{section}

\colorlet{darkcyan}{cyan!50!black}
\newcommand\Black[1]{\textcolor[gray]{0.3}{#1}}
\newcommand\Brown[1]{\textcolor[HTML]{998A4E}{#1}}
\newcommand\Emph[1]{\colorbox{green!10}{\textcolor{green!30!black}{#1}}}
\newcommand\Notes[1]{\textcolor{yellow!50!black}{\small #1}}
\newcommand\Example[1]{\textcolor{cyan!70!black}{\small #1}}

\renewcommand\d{\mathrm{d}}

\lstset{
    basicstyle=\small\ttfamily, %注意行末有逗号!
    keywordstyle=\bfseries\color{blue!70!black},
    commentstyle=\color{cyan!90!black},
    stringstyle=\color{green!40!black},
    columns=flexible,
    numbers=left,
    numberstyle=\footnotesize,
    escapechar=`,
    frame=shadowbox,
    %rulesepcolor=\color{red!20!blue!20!green!20}
    backgroundcolor=\color{cyan!5!white},
    language = SQL,
    tabsize = 4,
    breaklines = true,
}

\newcommand\IssueNumber{50}
\newcommand\Date{2025-3-20}
%\newcommand\Contributer{@金光日}
\newcommand\Subject{数据库系统原理}


\begin{document}
\backgroundsetup{contents=\includegraphics{上半示例.png}, center, scale=1, angle=0, opacity=1}
\BgThispage
\begin{center}
%{\scriptsize\Issue \textcolor[HTML]{C8BA83}{\Genshin WEEKLY TIPS}}
\phantom{...}

{\Large\textcolor{brown!40!white}{\makebox[10cm][s]{\Genshin WEEKLY KNOWLEDGE TIPS}}}

\vspace{-2em}

{\Huge\bfseries\TitleFont \Black{知\ 识\ 小\ 料}}


\vspace{-0.1cm}
{\footnotesize \Brown{「电计 2203 班」周常规知识整理共享}}
\end{center}

\vspace{-0.5cm}


\begin{figure}[H]
\hspace{1cm}
\begin{minipage}[t]{0.3\textwidth}
\centering
    \Brown{\Genshin ISSUE}

    \vspace{-0.6cm}
    \Huge \Issue\slshape\bfseries\Black{\IssueNumber}
\end{minipage}
\hfill
\begin{minipage}[t]{0.35\textwidth}
\centering
    \Brown{日期:\Date} \\
%\vspace{-0.1cm}
%    \Brown{贡献者:\Contributer} \\
\vspace{-0.1cm}
    \Brown{学科:\Subject} \\
\end{minipage}
\hspace{0.8cm}
\end{figure}

{\color{darkcyan}
设学生—课程数据库中包含三个表:
\begin{itemize}[itemsep=0pt,parsep=0pt]
    \item 学生表:$\mathrm{Student(Sno, Sname, Ssex, Sage, Sdept)}$
    \item 课程表:$\mathrm{Course(Cno, Cname, Ccredit)}$
    \item 选课表:$\mathrm{SC(Sno, Cno, Grade)}$
\end{itemize}
完成下面两个操作:
\begin{enumerate}
    \item 查询选修了课程名为「数据库系统原理」的学生姓名和课程成绩。使用关系代数、ALPHA 语言、QBE 语言、SQL 语言完成操作。(16分)
    \item 查询选修了全部课程的学生学号和姓名。使用关系代数、ALPHA 语言、SQL 语言完成操作。(12分)
\end{enumerate}
}

【第 1 问】
\begin{itemize}
    \item 用关系代数查找。课程名称位于 Course 表,学生姓名位于 Student 表,课程成绩位于 SC 表,所以应该做出\Emph{三表连接}。比较简单的一种操作是:
    \begin{equation*}
        \mathrm{\pi_{Sname,Grade}(\sigma_{Cname= "\text{数据库系统原理}"}(Student \bowtie Course \bowtie SC))}
    \end{equation*}

    也可以先在内部做出选择,再在外部连接:
    \begin{equation*}
        \mathrm{\pi_{Sname,Grade}(Student \bowtie \sigma_{Cname= "\text{数据库系统原理}"}(Course) \bowtie SC) }
    \end{equation*}
    
    \item 用 ALPHA 语言查找。限定的是课程名,可以对课程表定义 \textsc{Range} 语句,采用课程实例 (CX) 将三表连接。
    \begin{table}[htb]
    \newcommand\SC{\textcolor{blue}{SC}}
    \newcommand\Student{\textcolor{green!40!black}{Student}}
    \newcommand\CX{\textcolor{red!70!black}{CX}}    
        \centering
        \begin{tabular}{llll}
        \textsc{Range} & Course & \CX & \\
        \multicolumn{3}{l}{\textsc{Get}\  W (\Student.Sname, \SC.Grade):\ } &  $\exists$ \CX( \Student.Sno = \SC.Sno $\wedge$\\
        \multicolumn{4}{l}{ \CX.Cno=\SC.Cno $\wedge$ \CX.Cname="数据库系统原理")} \\
        \end{tabular}
    \end{table}
    
    \item 用 QBE 语言查找。注意实例用下划线标出,所有属性要写全,用于连接的示例(\uline{001},\uline{2})要彼此一致,用于打印的项目前面应加上「P.」。
    
    \begin{table}[H]
    \centering
    \begin{tabular}{|c|c|c|c|c|c|}
        \hline
        Student & Sno & Sname & Ssex & Sage & Sdept \\
        \hline
                & \uline{001} & P.\uline{李勇} &      &      & \\
        \hline
    \end{tabular}\\
    
    \begin{tabular}{|c|c|c|c|}
        \hline
        Course & Cno & Cname & Ccredit \\
        \hline
                & \uline{2} & 数据库系统原理 & \\
        \hline
    \end{tabular}\\
    
    \begin{tabular}{|c|c|c|c|}
        \hline
        SC & Sno & Cno & Grade \\
        \hline
           & \uline{001} & \uline{2} & P.\uline{90} \\
        \hline
    \end{tabular}
    \end{table}
    
    \item 用 SQL 语言查找。可以采用朴素的自然连接来实现。
\begin{lstlisting}
SELECT Sname, Grade FROM Student, Course, SC
WHERE Student.Sno = SC.Sno AND Course.Cno = SC.Cno 
    AND Course.Cname="数据库系统原理";
\end{lstlisting}
\end{itemize}

\backgroundsetup{contents=\includegraphics{空白示例.png}, center, scale=1, angle=0, opacity=1}
\BgThispage

【第 2 问】
\begin{itemize}
    \item 用关系代数查找。涉及「全部课程」,因此宜做个\Emph{除法}。查询的除了学号,还有姓名,因此宜与学生表做个连接。

\begin{equation*}
    \mathrm{(\pi_{Sno,Cno}(SC)\div \pi_{Cno}(Course)) \bowtie \pi_{Sno,Sname}(Student)}
\end{equation*}

    \item 用 ALPHA 语言查找。原意 「查询选修了全部课程的学生学号和姓名」可以翻译如下:
    
    \begin{table}[htb]
    \newcommand\Student{\mathrm{Student}}
    \newcommand\Sno{\mathrm{Sno}}
    \newcommand\Cno{\mathrm{Cno}}
    \newcommand\Sname{\mathrm{Sname}}
    \newcommand\SC{\mathrm{SC}}
    \newcommand\Course{\mathrm{Course}}
    \renewcommand\sc{\textcolor{blue}{sc}}
    \newcommand\s{\textcolor{green!40!black}{s}}
    \renewcommand\c{\textcolor{red!70!black}{c}}
        \centering
        \begin{tabular}{ll}
          $W=\{\s.\Sno,\s.\Sname | \s\in \Student \wedge$  & 查询这样的\textcolor{green!40!black}{学生},\\
           $(\forall \c) [\c\in \Course \to $ & 使得对\textbf{任意}一门\textcolor{red!70!black}{课},\\
           $(\exists \sc) (\sc\in \SC \wedge $ & 总\textbf{存在}一条\textcolor{blue}{选课记录},\\
           $\sc.\Sno=\s.\Sno \wedge \sc.\Cno=\c.\Cno )]\}$ & 是这位\textcolor{green!40!black}{学生}选的这一门\textcolor{red!70!black}{课}的记录。\\
        \end{tabular}
    \end{table}
    
    \begin{table}[htb]
    \newcommand\SCX{\textcolor{blue}{SCX}}
    \newcommand\Student{\textcolor{green!40!black}{Student}}
    \newcommand\CX{\textcolor{red!70!black}{CX}}  
        \centering
        \begin{tabular}{lll}
            \textsc{Range} & Course & \CX \\
            & SC & \SCX \\
            \multicolumn{3}{l}{\textsc{Get}\ W (\Student.Sno,\ \Student.Sname):} \\
            \multicolumn{3}{l}{\qquad $\forall$ \CX \ $\exists$ \SCX (\SCX.Sno=\Student.Sno $\wedge$ \SCX.Cno=\CX.Cno)}
        \end{tabular}
    \end{table}
    
    \item 用 SQL 语言查找。由于 SQL 不支持 $\forall$ 只支持 $\exists$(\verb!EXISTS!),因此需要把上面的分析改写:「所要查询的学生,不存在一门课程他没有选修」。
        
    \begin{table}[H]
        \centering
        \begin{tabular}{ll} %这一段有bug,有些命令只能强制废弃,AI也查不出来
            \newcommand\Student{\mathrm{Student}}
            %\newcommand\Sno{}
            %\newcommand\Cno{}
            \newcommand\Sname{\mathrm{Sname}}
            %\newcommand\SC{}
            %\newcommand\Course{\mathrm{Course}}
            %\newcommand\scRecord{}  % 避免覆盖原生\sc
            %\newcommand\s{}
            %\newcommand\c{\textcolor{red!70!black}{c}}

            $W = \{\,\textcolor{green!40!black}{s}.\mathrm{Sno}, \textcolor{green!40!black}{s}.\Sname \mid \textcolor{green!40!black}{s} \in \Student \land$
            & 查询这样的\textcolor{green!40!black}{学生},\\
            $\neg(\exists \textcolor{red!70!black}{c}) [\textcolor{red!70!black}{c} \in \mathrm{Course} \land$
            & \textbf{不存在}一门\textcolor{red!70!black}{课},\\
            $\neg(\exists \textcolor{blue}{sc}) (\textcolor{blue}{sc} \in \mathrm{SC} \land$
            & 使得\textbf{不存在}一条\textcolor{blue}{选课记录},\\
            $\textcolor{blue}{sc}.\mathrm{Sno} = \textcolor{green!40!black}{s}.\mathrm{Sno} \land \textcolor{blue}{sc}.\mathrm{Cno} = \textcolor{red!70!black}{c}.\mathrm{Cno})]]\}$
            & 是这位\textcolor{green!40!black}{学生}选的这一门\textcolor{red!70!black}{课}的记录。
        \end{tabular}
    \end{table}
        
        如果还是觉得「别扭」的话,也可以运用离散数学的知识,对 ALPHA 语言分析的这一段列出的谓词公式做出等价变换,可以看到推导结果是相同的。
\begin{align*}
          & (\forall c)[c\in\mathrm{Course} \to (\exists sc)(sc\in\mathrm{SC} \wedge (...))] \\
     \iff & (\forall c)[\neg c\in\mathrm{Course} \vee (\exists sc)(sc\in\mathrm{SC} \wedge (...))] \\
     \iff & (\forall c)\neg [c\in\mathrm{Course} \wedge \neg(\exists sc)(sc\in\mathrm{SC} \wedge (...))] \\
     \iff & \neg(\exists c) [c\in\mathrm{Course} \wedge \neg(\exists sc)(sc\in\mathrm{SC} \wedge (...))] \\
\end{align*}
        
        于是我们就可以撰写代码(可在PPT的例46中参考):
\begin{lstlisting}
SELECT Sno,Sname FROM Student WHERE NOT EXISTS 
    (SELECT * FROM Course WHERE NOT EXISTS 
        (SELECT * FROM SC WHERE 
        SC.Sno=Student.Sno AND SC.Cno=Course.Cno));
\end{lstlisting}
\end{itemize}

\vspace{1em}

{\color{cyan!80!black}
【结论】见解析。

【点评】这是一道数据库系统原理的真题,分值不低,是妥妥的重点。而且,对于「查询选修了全部课程的学生」这一问的存在关系和全称关系,需要多多加以理解,是妥妥的难点。这两道题均可以在老师的幻灯片中找到,应该掌握。
}

\backgroundsetup{contents=\includegraphics{下半示例.png}, center, scale=1, angle=0, opacity=1}
\BgThispage

\end{document} 