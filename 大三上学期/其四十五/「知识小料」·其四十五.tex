\documentclass[UTF8]{ctexart}
\usepackage{amsmath}
\usepackage{amssymb}
\usepackage{background}
\usepackage{booktabs}
\usepackage{enumitem}
\usepackage{fancyhdr}
\usepackage{float}
\usepackage{fontspec}
\usepackage{geometry}
\usepackage{listings}
\usepackage{tasks}
\usepackage{tcolorbox}
\usepackage{tikz}
\usetikzlibrary{arrows.meta}
\usepackage[table]{xcolor}

\geometry{a5paper, top=0.1cm, left=1cm, right=1cm, bottom=1cm, footskip=0.1cm}
\setCJKmainfont[BoldFont={汉仪文黑-85W},ItalicFont={汉仪文黑-55W}]{汉仪文黑-55W}
\setfontfamily\Issue{Century Schoolbook}
\setfontfamily\Genshin{Genshin Teyvat Lingua Franca}
\newCJKfontfamily\TitleFont{思源宋体 CN Heavy}
\newfontfamily\timesnewroman{Times New Roman}

%————————————————可变部分——————————————————
\settasks{label={\Alph*.\ }, label-format={\color{cyan!50!black}}, item-format={\color{cyan!50!black}}}
\newcommand\col[1]{\textcolor{green!50!black}{#1}}
\newcommand\coll[1]{\textcolor{blue}{#1}}
\newcommand\dotting{\ .\ }
\setlist[enumerate]{itemsep=0pt, parsep=0pt}
\setlist[itemize]{itemsep=0pt, parsep=0pt}
%——————————————————————————————————————————

\pagestyle{fancy}
\fancyhf{}
\cfoot{\sffamily\footnotesize{-\ \thepage\ -}}
%\CTEXsetup[format = {\centering\bfseries\large}, beforeskip = 3pt, afterskip = 3pt]{section}

\colorlet{darkcyan}{cyan!50!black}
\newcommand\Black[1]{\textcolor[gray]{0.3}{#1}}
\newcommand\Brown[1]{\textcolor[HTML]{998A4E}{#1}}
\newcommand\Emph[1]{\colorbox{green!10}{\textcolor{green!30!black}{#1}}}
\newcommand\Notes[1]{\textcolor{yellow!50!black}{\small #1}}
\newcommand\Example[1]{\textcolor{cyan!70!black}{\small #1}}
\colorlet{note}{yellow!70!black}


\newcommand\IssueNumber{45}
\newcommand\Date{2024-12-9}
%\newcommand\Contributer{@金光日}
\newcommand\Subject{计算机组成原理·实验}
%\newcommand\Source{历年考研 408 真题}


\begin{document}
\backgroundsetup{contents=\includegraphics{上半示例.png}, center, scale=1, angle=0, opacity=1}
\BgThispage
\begin{center}
%{\scriptsize\Issue \textcolor[HTML]{C8BA83}{\Genshin WEEKLY TIPS}}
\phantom{...}

{\Large\textcolor{brown!40!white}{\makebox[10cm][s]{\Genshin WEEKLY KNOWLEDGE TIPS}}}

\vspace{-2em}

{\Huge\bfseries\TitleFont \Black{知\ 识\ 小\ 料}}


\vspace{-0.1cm}
{\footnotesize \Brown{「电计 2203 班」周常规知识整理共享}}
\end{center}

\vspace{-0.5cm}


\begin{figure}[H]
\hspace{1cm}
\begin{minipage}[t]{0.3\textwidth}
\centering
    \Brown{\Genshin ISSUE}

    \vspace{-0.6cm}
    \Huge \Issue\slshape\bfseries\Black{\IssueNumber}
\end{minipage}
\hfill
\begin{minipage}[t]{0.28\textwidth}
\centering
    \Brown{日期:\Date} \\
%\vspace{-0.1cm}
%    \Brown{贡献者:\Contributer} \\
\vspace{-0.1cm}
    \Brown{学科:\Subject} \\
%\vspace{-0.1cm}
%    \Brown{来源:\Source}
\end{minipage}
\hspace{0.8cm}
\end{figure}

{\color{cyan!50!black}
\begin{center}
  《计组实验一小记》
\end{center}}

\section{图与PPT部分}

\paragraph{五条指令}
\begin{enumerate}
  \item \verb!add.w   rd rj rk!
  \item \verb!addi.w  rd rj si12!
  \item \verb!ld.w    rd rj si12!
  \item \verb!st.w    rd rj si12!
  \item \verb!bne     rj rd offs16!
\end{enumerate}

\paragraph{寄存器堆}
\begin{itemize}[itemsep=0pt,parsep=0pt]
  \item RA1, RA2, WA1 表示地址线。
  \item RD1, RD2, WD1 表示数据线。
  \item 在图中,RA1 连 rj, RD1 连 src1,表示在进入 ALU 前把 GR[rj] 的值作为 src1 加数传入。
\end{itemize}

注:\verb!addi,w!和 \verb!ld.w!不需要使用寄存器堆的 RA2。

\paragraph{控制信号表}
控制信号表如下所示,注意PPT图中表示和代码表示的对应关系。

几个缩写:rf 的全称是 register file (寄存器堆),src 的全称是 source(源头),sel 的全称是 selection(选择)。

\begin{table}[H]
\small
    \centering
    \rowcolors{2}{cyan!10}{cyan!20}
    \begin{tabular}{llp{0.3\textwidth}p{0.27\textwidth}}
        \rowcolor{cyan!50}
        图示 & 代码表示 & 表示含义 & 牵扯指令 \\

        \verb!sel_alu_src2! & \verb!src2_is_imm! & \textbf{抉择}:操作数2是来自RD2的数,或者立即数(的符号扩展) & \verb!addi.w, ld.w, st.w!(立即数) \\
        \verb!sel_rf_res! & \verb!res_from_mem! & \textbf{抉择}:ALU和是直接作为结果,或作为存储器的地址 & \verb!ld.w!(地址) \\
        \verb!sel_rf_ra2! & \verb!src_reg_is_rd! & \textbf{抉择}:寄存器的 RA2 应准许 rk ,或者 rd & \verb!st.w, bne!(准许rd) \\
        \verb!data_ram_we! & \verb!mem_we! & 对存储器写使能信号 & \verb!st.w! \\
        \verb!data_ram_ce! & - & 对存储器写片选控制信号 & \verb!ld.w, st.w! \\
        \verb!br_taken! & \verb!br_taken! & \textbf{抉择}:PC 的目标采取 PC+4,或 \verb!br_target! & \verb!bne!(采取后者) \\
        \verb!rf_we! & \verb!gr_we! & 对寄存器堆的写使能信号(表示能否写寄存器堆) & \verb!add.w, addi.w, ld.w!(开启) \\
    \end{tabular}
    \caption{控制信号表}\label{fig:control}
\end{table}

\backgroundsetup{contents=\includegraphics{空白示例.png}, center, scale=1, angle=0, opacity=1}
\BgThispage

\section{代码部分}
\paragraph{顶层设计}
\begin{itemize}
    \item \verb!inst_sram_wen!等 4 个:从指令存储器与CPU交互的 4 个信号(写使能、地址、写数据、读数据)。
    \item \verb!data_sram_wen!等 4 个:从数据存储器与CPU交互的 4 个信号(写使能、地址、写数据、读数据)。
    \item \verb!clk!:时钟。
    \item \verb!resetn!:当取 1 时,\verb!reset!取 0,\verb!valid!取 1,\verb!pc!取 \verb!nextpc!。
\end{itemize}

\paragraph{指令分割}
对 \verb!inst! 进行分割。

\begin{itemize}
    \item \verb!op_31_26! 等 4 个:表示指令特定段的内容。
    \item \verb!rd!:永远是 $ \{4:0\} $。
    \item \verb!rj!:永远是 $ \{9:5\} $。
    \item \verb!rk!:永远是 $ \{14:10\} $。
    \item \verb!i12!:12 位立即数,在 \verb!addi.w, ld.w, st.w! 指令使用。
    \item \verb!i16!:16 位立即数,在 \verb!bne! 指令使用。
\end{itemize}

\paragraph{指令操作码译码}
封装在黑箱子里。

\begin{lstlisting}
wire [ 5:0] op_31_26;
wire [ 3:0] op_25_22;
wire [ 1:0] op_21_20;
wire [ 4:0] op_19_15;
\end{lstlisting}

随后是指令操作码的确定,需参考指令手册。

\begin{table}[H]
  \centering
  \rowcolors{2}{cyan!10}{cyan!20}
  \begin{tabular}{ccccc}
    \rowcolor{cyan!50}
    指令 & 31:26 & 25:22 & 21:20 & 19:15  \\
      \verb!add.w! & 000000 & 0000 & 01 & 00000 \\
      \verb!addi.w! & 000000 & 1010 & - & - \\
      \verb!ld.w! & 001010 & 0010 & - & - \\
      \verb!st.w! & 001010 & 0110 & - &  - \\
      \verb!bne! & 010111 & - & - & - \\
  \end{tabular}
  \caption{指令操作码}\label{tab:operation}
\end{table}

\paragraph{控制信号}
通过表 \ref{fig:control} 的内容,由指令种类控制各种信号。下面的列表中,「生效」表示该变量取 1,「不生效」表示该变量取 0。

\begin{itemize}
  \item \verb!src2_is_imm  !:对指令 \verb!addi.w, ld.w, st.w! 生效,表示寄存器第二操作数 RD2 为立即数。
  \item \verb!res_from_mem !:对指令 \verb!ld.w! 生效,表示 ALU 相加和作为存储器的地址。
  \item \verb!gr_we        !:对指令 \verb!add.w, addi.w, ld.w! 生效,表示允许寄存器写数据。
  \item \verb!mem_we       !:对指令 \verb!st.w! 生效,表示允许存储器写数据。
  \item \verb!src_reg_is_rd!:对指令 \verb!st.w, bne! 生效,表示寄存器的第二地址 RA2 准许 rd。
  \item \verb!rf_raddr1!:表示寄存器第一地址 RA1 的取值,默认为 rj。
  \item \verb!rf_raddr2!:表示寄存器第二地址 RA2 的取值,当 \verb!src_reg_is_rd! 生效时取 rd,不生效取 rk。
\end{itemize}

\paragraph{寄存器黑箱}
\begin{itemize}
    \item \verb!clk, raddr1, raddr2, rdata2, we, waddr, wdata! 表示寄存器内外连接的引脚。
    \item 要填入的信息如凸出部分所示,比如 \verb!.waddr(rd)! 意思是将 \verb!rd!送入寄存器写地址的引脚,毕竟当寄存器需要被写入时(由指令 \verb!add.w, addi.w, ld.w!控制着),其地址仅可能是 rd。
\end{itemize}

\begin{lstlisting}
regfile u_regfile(
    .clk    (clk      ),
    .raddr1 (  rf_raddr1  ),
    .rdata1 (rj_value),
    .raddr2 (  rf_raddr2  ),
    .rdata2 (rkd_value),
    .we     (gr_we    ),
    .waddr  (  rd   	  ),
    .wdata  (rf_wdata )
);
\end{lstlisting}

\paragraph{其他信息}
\begin{itemize}[itemsep=2pt, parsep=3pt]
    \item \verb!br_offs!:对 \verb!bne! 指令生效,用于确定 pc 跳转的偏移量。对照指令,其目标为 16 位立即数作符号扩展得到的数值。扩展方法:往后填两个 0;往前对其最高位重复填写 14 位。

    所以是 \verb!{{14{inst[25]}}, inst[25:10], 2'b00}!

    或者是 \verb!{{14{i16[15]}}, i16[15:0], 2'b00}!

    \item \verb!br_target!:对 \verb!bne! 指令生效,用于确定 pc 跳转的目标。
    \item \verb!rj_eq_rd!:对 \verb!bne! 指令生效,表示读取 \verb!rj_value! 和 \verb!rkd_value! 是否相等。其中 \verb!rj_value! 和 \verb!rkd_value! 在寄存器黑箱里面通过 raddr1 和 raddr2 读出。
    \item \verb!br_taken!:对 \verb!bne! 指令生效,意为控制 pc 使用 \verb!br_target!(而非 pc+4)的条件。需要满足以下三点:
        \begin{enumerate}
            \item \verb!valid!取 1,也就是至少要开机;
            \item \verb!inst_bne!取 1,也就是要满足 \verb!bne!指令的操作码(或者说指令的种类是 \verb!bne!);
            \item \verb!rj_eq_rd!取 1,也就是指令所述的跳转条件。
        \end{enumerate}
    \item \verb!nextpc!:对 \verb!bne! 指令生效,确定下一个 pc 值。
    \item \verb!imm!:对 \verb!addi.w, ld.w, st.w! 指令生效,用于把 12 位立即数扩到 32 位。扩展方法:不往后填,往前对其最高位重复填写 20 位。
    \item \verb!alu_src1!:对前四条指令生效,表示 ALU 的第一操作数,其值总是 GR[rj] 即 \verb!rj_value!。
    \item \verb!alu_src2!:对前四条指令生效,表示 ALU 的第二操作数。当 \verb!sel_alu_src2!(代码表示为 \verb!src2_is_imm!)生效时,其值为刚刚扩展过的立即数 \verb!imm!,不生效时为 \verb!rkd_value!。
    \item \verb!alu_result!:对前四条指令生效,表示 ALU 加法的和。
    \item \verb!data_sram_we! 等 3 个:对 \verb!ld.w, st.w! 指令生效,是顶层设计中 CPU 和数据存储器交互的接口。
        \begin{enumerate}
            \item \verb!data_sram_we! 应取图上的 \verb!data_ram_we!(代码表示为 \verb!ram_we!)。
            \item \verb!data_sram_addr! 应取图上的 ALU 加法结果,也就是代码的 \verb!alu_result!。
            \item \verb!data_sram_wdata! 对 \verb!st.w! 指令生效,应取图上的 GR[rd] 也就是代码的 \verb!rkd_value!。
        \end{enumerate}
    \item \verb!rf_wdata!:表示对寄存器堆写入的数据,在图上的 \verb!sel_rf_res!(代码表示 \verb!res_from_mem!)生效时,取数据存储器读出的值 \verb!data_sram_rdata!(一个顶层设计变量);不生效时取 ALU 加法结果。
\end{itemize}

\backgroundsetup{contents=\includegraphics{下半示例.png}, center, scale=1, angle=0, opacity=1}
\BgThispage

\end{document}