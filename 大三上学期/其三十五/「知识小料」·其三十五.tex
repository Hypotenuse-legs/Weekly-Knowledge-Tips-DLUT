\documentclass[UTF8]{ctexart}
\usepackage{amsmath}
\usepackage{amssymb}
\usepackage{booktabs}
\usepackage{background}
\usepackage{caption,subcaption}
\usepackage{CJKfntef}
\usepackage{cprotect}
\usepackage{enumitem}
\usepackage{extarrows}
\usepackage{fancyhdr}
\usepackage{float}
\usepackage{fontspec}
%\usepackage{fourier}
\usepackage{geometry}
\usepackage{listings}
\usepackage{tcolorbox}
\tcbuselibrary{breakable}
\usepackage{tikz}
\usetikzlibrary{arrows.meta}
\usepackage[table]{xcolor}

\geometry{a5paper, top=0.1cm, left=1cm, right=1cm, bottom=1cm, footskip=0.1cm}
\setCJKmainfont[BoldFont={汉仪文黑-85W},ItalicFont={方正苏新诗柳楷简体}]{汉仪文黑-55W}
\setfontfamily\Issue{Century Schoolbook}
\setfontfamily\Genshin{Genshin Teyvat Lingua Franca}
\newCJKfontfamily\TitleFont{思源宋体 CN Heavy}
\newfontfamily\timesnewroman{Times New Roman}
%\reversemarginpar

\pagestyle{fancy}
\fancyhf{}
%\cfoot{\sffamily\footnotesize{-\ \thepage\ -}} 本次知识小料不使用页码
%\CTEXsetup[format = {\centering\bfseries\large}, beforeskip = 3pt, afterskip = 3pt]{section}

\colorlet{darkcyan}{cyan!50!black}
\newcommand\Black[1]{\textcolor[gray]{0.3}{#1}}
\newcommand\Brown[1]{\textcolor[HTML]{998A4E}{#1}}
\newcommand\Emph[1]{\colorbox{green!10}{\textcolor{green!30!black}{#1}}}
\newcommand\Notes[1]{\textcolor{yellow!50!black}{\small #1}}
\newcommand\Example[1]{\textcolor{cyan!70!black}{\small #1}}
\newcommand\keyword[1]{\textcolor{violet}{\textbf{\texttt{#1}}}}

\newcommand\AND{\  \textrm{AND}\ }
\newcommand\OR{\  \textrm{OR}\ }
\newcommand\IF{\quad\textrm{IF}\quad}
\newcommand\THEN{\quad\textrm{THEN}\quad}
\newcommand\CF{{C\hspace{-0.2em}F}}

\lstset{
    basicstyle=\small\ttfamily, %注意行末有逗号!
    keywordstyle=\bfseries\color{blue!70!black},
    commentstyle=\color{cyan!90!black},
    stringstyle=\color{green!40!black},
    columns=flexible,
    numbers=left,
    numberstyle=\footnotesize,
    escapechar=`,
    frame=shadowbox,
    %rulesepcolor=\color{red!20!blue!20!green!20}
    backgroundcolor=\color{cyan!5!white},
    language = C++,
    tabsize = 4,
    breaklines = true,
    showstringspaces = false,
}

\newcommand\IssueNumber{35}
\newcommand\Date{2024-9-27}
%\newcommand\Contributer{@金光日}
\newcommand\Subject{人工智能}


\begin{document}
\backgroundsetup{contents=\includegraphics{上半示例.png}, center, scale=1, angle=0, opacity=1}
\BgThispage
\begin{center}
%{\scriptsize\Issue \textcolor[HTML]{C8BA83}{\Genshin WEEKLY TIPS}}
\phantom{...}

{\Large\textcolor{brown!40!white}{\makebox[10cm][s]{\Genshin WEEKLY KNOWLEDGE TIPS}}}

\vspace{-2em}

{\Huge\bfseries\TitleFont \Black{知\ 识\ 小\ 料}}


\vspace{-0.1cm}
{\footnotesize \Brown{「电计 2203 班」周常规知识整理共享}}
\end{center}

\vspace{-0.5cm}


\begin{figure}[H]
\hspace{1cm}
\begin{minipage}[t]{0.3\textwidth}
\centering
    \Brown{\Genshin ISSUE}

    \vspace{-0.6cm}
    \Huge \Issue\slshape\bfseries\Black{\IssueNumber}
\end{minipage}
\hfill
\begin{minipage}[t]{0.35\textwidth}
\centering
    \Brown{日期:\Date} \\
%\vspace{-0.1cm}
%    \Brown{贡献者:\Contributer} \\
\vspace{-0.1cm}
    \Brown{学科:\Subject} \\
\end{minipage}
\hspace{0.8cm}
\end{figure}

{\color{cyan!50!black}
设有如下一组知识:
\begin{table}[htb]
\color{cyan!50!black}
  \centering
  \begin{tabular}{c@{\quad IF\quad }l@{\quad THEN\quad }cc}
  $r_1:$ & 头痛 \AND 免疫力低 & 感冒 & ($0.8$) \\
  $r_2:$ & 感冒 \AND (流清鼻涕 \OR 喉咙干痒) & 风寒感冒 & ($0.85$) \\
  $r_3:$ & 感冒 \AND 流黄鼻涕 & 风热感冒 & ($0.8$) \\
  $r_4:$ & 感冒 \AND 喉咙肿痛 & 风热感冒 & $(0.9)$ \\
  \end{tabular}
\end{table}

某同学对以下症状的可信度分别如下表,请你帮忙分别算出该同学患风寒感冒和风热感冒的可信度。

\begin{table}[htb]
  \centering
  \color{cyan!50!black}
  \begin{tabular}{cccc}
  \toprule
    症状 & 可信度 & 症状 & 可信度 \\
  \midrule
    头痛 & $0.8$ & 免疫力低 & $0.75$ \\
    流清鼻涕 & $0.6$ & 流黄鼻涕 & $0.2$ \\
    喉咙干痒 & $0.3$ & 喉咙肿痛 & $0.5$ \\
  \bottomrule  
  \end{tabular}
\end{table}
}

本题考察的是可信度方法与 C-F 模型,详见《人工智能》课本第 4 章第 117 页。

具体包括以下规则:
\begin{itemize}[itemsep=0pt,parsep=0pt]
  \item 合取规则:$\CF(E_1\AND E_2) = \min\{\CF(E_1),\ \CF(E_2)\}$
  \item 析取规则:$\CF(E_1\OR E_2) = \max\{\CF(E_1),\ \CF(E_2)\}$
  \item 传递规则:$\CF(H) = \CF(H,E)\cdot \max\{0,\CF(E)\}$
  \item 合成规则:若有 $\begin{cases}
  \IF E_1 \THEN H\quad  (\CF(H,E_1))\\
  \IF E_2 \THEN H\quad  (\CF(H,E_2))
   \end{cases}$
    ,须先由传递规则算出 $a=\CF_1(H)$、$b=\CF_2(H)$。设 $S=\CF_{12}(H)$,则
      \begin{enumerate}[itemsep=2pt,parsep=0pt]
        \item 当 $a\geqslant 0,b\geqslant 0$ 时,$S = a+b-ab$
        \item 当 $a<0,b<0$ 时,$S = a+b+ab$
        \item 当 $a,b$ 异号时,$S = \dfrac{a+b}{1-\min\{|a|,|b|\}}$
      \end{enumerate}
\end{itemize}

可以形式化地把头痛、免疫力低、流清鼻涕、流黄鼻涕、喉咙干痒、喉咙肿痛分别设为 $E_1\sim E_6$,把感冒、风寒感冒、风热感冒分别设为 $H,H_1,H_2$。如此,原始知识整理为

\backgroundsetup{contents=\includegraphics{空白示例.png}, center, scale=1, angle=0, opacity=1}
\BgThispage
\begin{table}[htb]
  \centering
  \begin{tabular}{c@{\quad IF\quad }l@{\quad THEN\quad }cc}
  $r_1:$ & $E_1$ \AND $E_2$ & $H$ & ($0.8$) \\
  $r_2:$ & $H$ \AND ($E_3$ \OR $E_5$) & $H_1$ & ($0.85$) \\
  $r_3:$ & $H$ \AND $E_4$ & $H_2$ & ($0.8$) \\
  $r_4:$ & $H$ \AND $E_6$ & $H_2$ & ($0.9$) \\
  \end{tabular}
\end{table}

\begin{table}[htb]
  \centering
  \begin{tabular}{cccc}
  \toprule
    症状 & 可信度 & 症状 & 可信度 \\
  \midrule
    $\CF(E_1)$ & $0.8$ & $\CF(E_2)$ & $0.75$ \\
    $\CF(E_3)$ & $0.6$ & $\CF(E_4)$ & $0.2$ \\
    $\CF(E_5)$ & $0.3$ & $\CF(E_6)$ & $0.5$ \\
  \bottomrule
  \end{tabular}
\end{table}

计算感冒 $H$ 的可信度:
\begin{equation*}
\begin{aligned}
    \CF(E_1\AND E_2) &\xlongequal{\text{合取}} \min\{\CF(E_1),\CF(E_2)\} = \min\{0.8,\ 0.75\} = 0.75 \\
    \CF(H) &\xlongequal{\text{传递}} 0.8\times \max\{0, \ \CF(E_1\AND E_2)\} = \mathbf{0.6}
\end{aligned}
\end{equation*}

计算风寒感冒 $H_1$ 的可信度:
\begin{equation*}
\begin{aligned}
    \CF(H_1) &\xlongequal{\text{传递}} 0.85\times \max\{0,\ \CF\big(H \AND (E_3\OR E_5)\big)\} \\
    &\xlongequal{\text{合取}} 0.85\times \max\{0, \min\{\CF(H), \CF\big( E_3\OR E_5 \big)\}\} \\
    &\xlongequal{\text{析取}} 0.85\times \max\{0, \min\{0.6, \max\{\CF(E_3),\ \CF(E_5)\} \}\} \\
    &= 0.85\times \max\{0, \min\{0.6, \max\{0.6,0.3\} \}\} \\
    &= 0.85\times 0.6 = \mathbf{0.51}
\end{aligned}
\end{equation*}

由 $r_3$ 计算风热感冒 $H_2$ 的可信度 $\CF_1(H_2)$,记为 $a$:
\begin{equation*}
\begin{aligned}
    \CF(H \AND E_4) &\xlongequal{\text{合取}} \min\{\CF(H), \ \CF(E_4)\} = \min\{0.6, 0.2\} = 0.2 \\
    a = \CF_1(H_2) &\xlongequal{\text{传递}} 0.8\times \max\{0, \CF(H\AND E_4)\} = 0.16
\end{aligned}
\end{equation*}

由 $r_4$ 计算风热感冒 $H_2$ 的可信度 $\CF_2(H_2)$,记为 $b$:
\begin{equation*}
\begin{aligned}
    \CF(H\AND E_6) &\xlongequal{\text{合取}} \min\{\CF(H),\ \CF(E_6)\} = \min\{0.6, 0.5\} = 0.5 \\
    b = \CF_2(H_2) &\xlongequal{\text{传递}} 0.9\times \max\{0, \CF(H \AND E_6)\} = 0.45 \\
\end{aligned}
\end{equation*}


通过 $a$ 和 $b$ 由结论不确定性合成规则算出 $S=CF_{12}(H_2)$,由 $a>0,b>0$ 故使用第 1 条规则:
\begin{equation*}
    S = CF_{12}(H_2) = a + b - ab = 0.16 + 0.45 - 0.072 = \mathbf{0.538}
\end{equation*}

\backgroundsetup{contents=\includegraphics{下半示例.png}, center, scale=1, angle=0, opacity=1}
\BgThispage
因此该同学患风寒感冒和风热感冒的可信度分别是 $0.51$ 和 $0.538$。采用可信度方法推断该同学更有可能患风热感冒。

\vspace{1em}
{\color{cyan!80!black} 
【结论】该同学患风寒感冒和风热感冒的可信度分别是 0.51 和 0.538。

【点评】本题考察了可信度方法,综合运用了 C-F 模型的相关规则。由于人工智能这个学科实在找不到什么合适的考题,结合可信度方法可能是大题考点这一信息,自编了这一题。题中的情景可能与真实情况有一定差异,请勿全信。

\begin{quote}
\IF 期末考试 \THEN 考可信度方法大题 (0.7)
\end{quote}

$\CF(\text{期末考试})=1$,$\CF(\text{考可信度方法大题}) \xlongequal{\text{传递}} 0.7\times \max\{0,1\} = 0.7$,果然考可信度方法大题的可信度是 0.7,实在不小。
}

\end{document} 