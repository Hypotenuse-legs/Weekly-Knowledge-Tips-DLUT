\documentclass[UTF8]{ctexart}
\usepackage[linesnumbered,ruled,vlined]{algorithm2e}
\usepackage{amsmath}
\usepackage{amssymb}
\usepackage{booktabs}
\usepackage{background}
%\usepackage{caption,subcaption}
%\usepackage{CJKfntef}
%\usepackage{cprotect}
\usepackage{enumitem}
\usepackage{extarrows}
\usepackage{fancyhdr}
\usepackage{float}
\usepackage{fontspec}
%\usepackage{fourier}
\usepackage{geometry}
\usepackage{listings}
\usepackage{pdflscape}
%\usepackage{tasks}
\usepackage{tcolorbox}
\tcbuselibrary{breakable}
\usepackage{tikz}
\usetikzlibrary{arrows.meta}
\usepackage[table]{xcolor}

\geometry{a5paper, top=0.1cm, left=1cm, right=1cm, bottom=1cm, footskip=0.1cm}
\setCJKmainfont[BoldFont={汉仪文黑-85W},ItalicFont={汉仪文黑-55W}]{汉仪文黑-55W}
\setfontfamily\Issue{Century Schoolbook}
\setfontfamily\Genshin{Genshin Teyvat Lingua Franca}
\newCJKfontfamily\TitleFont{思源宋体 CN Heavy}
\newfontfamily\timesnewroman{Times New Roman}
%\reversemarginpar

\pagestyle{fancy}
\fancyhf{}
\cfoot{\sffamily\footnotesize{-\ \thepage\ -}}
%\CTEXsetup[format = {\centering\bfseries\large}, beforeskip = 3pt, afterskip = 3pt]{section}

\colorlet{darkcyan}{cyan!50!black}
\newcommand\Black[1]{\textcolor[gray]{0.3}{#1}}
\newcommand\Brown[1]{\textcolor[HTML]{998A4E}{#1}}
\newcommand\Emph[1]{\colorbox{green!10}{\textcolor{green!30!black}{#1}}}
\newcommand\Notes[1]{\textcolor{yellow!50!black}{\small #1}}
\newcommand\Example[1]{\textcolor{cyan!70!black}{\small #1}}
\newcommand\keyword[1]{\textcolor{violet}{\textbf{\texttt{#1}}}}

\newcommand\FIRST{\textsc{First}}
\newcommand\FOLLOW{\textsc{Follow}}
\newcommand\NULLABLE{\textsc{Nullable}}
\newcommand\FIRSTS{\textsc{First\_s}}
\newcommand\VN{V_{\mathrm{N}}} %非终结符
\newcommand\VT{V_{\mathrm{T}}} %终结符
\newcommand\deduceone{\overset{+}{\implies}} %至少一次推导
\newcommand\deduce{\overset{*}{\implies}} %任意次推导

\lstset{
    basicstyle=\small\ttfamily, %注意行末有逗号!
    keywordstyle=\bfseries\color{blue!70!black},
    commentstyle=\color{cyan!90!black},
    stringstyle=\color{green!40!black},
    columns=flexible,
    numbers=left,
    numberstyle=\footnotesize,
    escapechar=`,
    frame=shadowbox,
    %rulesepcolor=\color{red!20!blue!20!green!20}
    backgroundcolor=\color{cyan!5!white},
    tabsize = 4,
    breaklines = true,
    showstringspaces = false,
}

\newcommand\IssueNumber{37}
\newcommand\Date{2024-10-18}
%\newcommand\Contributer{@金光日}
\newcommand\Subject{编译原理}


\begin{document}
\backgroundsetup{contents=\includegraphics{上半示例.png}, center, scale=1, angle=0, opacity=1}
\BgThispage
\begin{center}
%{\scriptsize\Issue \textcolor[HTML]{C8BA83}{\Genshin WEEKLY TIPS}}
\phantom{...}

{\Large\textcolor{brown!40!white}{\makebox[10cm][s]{\Genshin WEEKLY KNOWLEDGE TIPS}}}

\vspace{-2em}

{\Huge\bfseries\TitleFont \Black{知\ 识\ 小\ 料}}


\vspace{-0.1cm}
{\footnotesize \Brown{「电计 2203 班」周常规知识整理共享}}
\end{center}

\vspace{-0.5cm}


\begin{figure}[H]
\hspace{1cm}
\begin{minipage}[t]{0.3\textwidth}
\centering
    \Brown{\Genshin ISSUE}

    \vspace{-0.6cm}
    \Huge \Issue\slshape\bfseries\Black{\IssueNumber}
\end{minipage}
\hfill
\begin{minipage}[t]{0.35\textwidth}
\centering
    \Brown{日期:\Date} \\
%\vspace{-0.1cm}
%    \Brown{贡献者:\Contributer} \\
\vspace{-0.1cm}
    \Brown{学科:\Subject} \\
\end{minipage}
\hspace{0.8cm}
\end{figure}

{\color{cyan!50!black}
已知文法 $G[S]$:
\begin{align*}
  S &\to MH \ |\  a \\
  H &\to LSo \ |\  \varepsilon \\
  K &\to dML \ |\  \varepsilon \\
  L &\to eHf \\
  M &\to K \ |\  bLM
\end{align*}
\begin{enumerate}[itemsep=0pt,parsep=0pt]
    \item 求每一非终结符的 \FIRST 集和 \FOLLOW 集。
    \item 填写该文法的 LL(1) 分析表。
    \item 该文法是否为 LL(1) 文法?请解释你的回答。
\end{enumerate}
}

\begin{tcolorbox}[colback=violet!5, colframe=violet, boxrule=1pt, ]
%\begin{description}[itemsep=0pt,parsep=0pt]
%  \item[$\VN$] 非终结符
%  \item[$\VT$] 终结符
%  \item[$V$] 所有单词,即 $\VN\cup\VT$
%  \item[$V^*$] 任意单词串,即 $V$ 的闭包
%  \item[$\to$] 定义为,如 $N\to s|t|g|w$
%  \item[$\implies$] 直接推导
%  \item[$\deduceone$] 长度 $n\geqslant 1$ 的推导。若 $w_0\implies w_1\implies w_2\implies \cdots\implies w_n(n\geqslant 1) $,则称 $w_0\deduceone w_n$
%  \item[$\deduce$] 长度 $n\geqslant 0$ 的推导。若 $w_0=w_n$ 或 $w_0\deduceone w_n$,则称 $w_0\deduce w_n$
%\end{description}
%
%平时我们所说的「非终结符 $A$ 能推空 $\varepsilon$」就是指 $A\deduce \varepsilon$。

对于 \FIRST、\FOLLOW、\NULLABLE 集,定义表示如下:
\begin{enumerate}
  \item \FIRST 集:
    \begin{equation*}
      \FIRST(A) = \{a\ |\ A\deduce aB\}
    \end{equation*}
    其中,$a\in \VT$(终结符),$B \in V^*$ (任意符号串),$\deduce$ 表示「经由任意次推导」。
  \item \FOLLOW 集:
    \begin{equation*}
      \FOLLOW(A) = \{a\ |\ S\deduce \cdots Aa\cdots \}
    \end{equation*}
    其中 $S$ 为开始符号,$a\in \VT$(终结符)。
  \item \NULLABLE 集:
    假设 $X$ 为非终结符。$X\in \NULLABLE$,当且仅当 $X\to \varepsilon$ 或者 $X\to \beta_1\beta_2\cdots\beta_n$(其中每个 $\beta_i$ 均属于 \NULLABLE)。
\end{enumerate}

\textcolor{cyan}{\FIRST$(A)$ 就是非终结符 $A$ 经任意次推导后,可能得到的开头终结符的集合。\FOLLOW$(A)$ 就是开始符 $S$ 经任意次推导后,$A$ 可能紧接着的终结符的集合。}

\end{tcolorbox}


\newpage
\section{求 \NULLABLE 集、\FIRST 集}
\backgroundsetup{contents=\includegraphics{空白示例.png}, center, scale=1, angle=0, opacity=1}
\BgThispage

容易看出这里面有可能推出空单词 $\varepsilon$ 的非终结符,需要先把可推空单词的非终结符之集合:\Emph{\NULLABLE 集}求出来。

第一遍迭代容易看出,$H, K\in \NULLABLE$。第二遍迭代,因为 $K\in \NULLABLE$,所以由规则 5,$M\in \NULLABLE$。第三遍迭代,因为 $M,H\in\NULLABLE$,所以由规则 1, $S\in \NULLABLE$。因此:
\begin{equation}\label{eq:NULLABLE}
    \NULLABLE = \{S, H, K, M\}
\end{equation}

接下来是 \Emph{\FIRST 集},算法如下:


\begin{algorithm}[H]
%\SetAlgoLined
\DontPrintSemicolon
初始化:对所有非终结符 $N$,$\FIRST(N)\gets\varnothing$  \textcolor{cyan}{\tcc*[h]{「$\gets$」 表示赋值}}\;
\While{some sets changed}{
    \For{production $p: \textcolor{violet}{N\to \beta_1\beta_2\cdots\beta_n}$}{
        \For(\textcolor{cyan}{\tcc*[h]{顺序遍历产生式的每个符号}}){$\beta_i \gets \beta_1 $\ \KwTo\ $ \beta_n$} {
            \If{$\beta_i = a$}{
                $\FIRST(N) \gets \FIRST(N) \cup \{a\}$\;
                \textbf{break}\;
            }
            \If{$\beta_i = M$}{
                $\FIRST(N) \gets \FIRST(N) \cup \FIRST(M)$ \;
                \If{$M \notin \NULLABLE$}{
                    \textbf{break}\;
                }
            }
        }
    }
}
\caption{\FIRST 集的计算}
\end{algorithm}

由 \NULLABLE 集信息,得到:
\begin{itemize}[itemsep=0pt,parsep=0pt]
  \item $\FIRST(S) = \FIRST(M)\cup \FIRST(H)\cup \{a\}$
  \item $\FIRST(H) = \FIRST(L)$
  \item $\FIRST(K) = \{d\}$
  \item $\FIRST(L) = \{e\}$
  \item $\FIRST(M) = \FIRST(K)\cup \{b\}$
\end{itemize}

多轮迭代后得到各个非终结符的 \FIRST 集如下(第 4 轮结果与第 3 轮完全一致,终止):

\begin{table}[H]
    \centering
    \begin{tabular}{cccccc}
    \toprule
        非终结符 & $S$ & $H$ & $K$ & $L$ & $M$ \\
    \midrule
        \FIRST 集·第 1 轮 & $\{a\}$ & $\varnothing$ & $\{d\}$ & $\{e\}$ & $\{b,d\}$ \\
        \FIRST 集·第 2 轮 & $\{a,b,d\}$ & $\{e\}$ & $\{d\}$ & $\{e\}$ & $\{b,d\}$ \\
        \FIRST 集·第 3 轮 & $\{a,b,d,e\}$ & $\{e\}$ & $\{d\}$ & $\{e\}$ & $\{b,d\}$ \\
        \FIRST 集·第 4 轮 & $\{a,b,d,e\}$ & $\{e\}$ & $\{d\}$ & $\{e\}$ & $\{b,d\}$ \\
    \bottomrule
    \end{tabular}
    \caption{\FIRST 集结果}\label{tab:FIRST}
\end{table}

\section{求 \FOLLOW 集}

求 \Emph{\FOLLOW 集}的算法如下:

\begin{algorithm}[H]
\DontPrintSemicolon
初始化:对所有非终结符 $N$,$\FOLLOW(N)\gets\varnothing$ \;
\While{some sets changed}{
    \For{production $p: \textcolor{violet}{N\to \beta_1\beta_2\cdots\beta_n}$}{
        $temp \gets \FOLLOW(N)$\;
        \For(\textcolor{cyan}{\tcc*[h]{逆序遍历产生式的每个符号}}){$\beta_i \gets \beta_n$ \ \KwTo\ $\beta_1$}{
            \If{$\beta_i = a$}{
                $temp \gets \{a\}$\;
            }
            \If{$\beta_i = M$}{
                $\FOLLOW(M) \gets \FOLLOW(M)\cup temp$ \;
                \eIf{$M \notin \NULLABLE$}{
                    $temp \gets \FIRST(M)$\;
                }{
                    $temp \gets \FIRST(M)\cup temp$\;
                }
            }
        }
    }
}
\caption{\FOLLOW 集的计算}
\end{algorithm}

算法执行流程:初始时,每个非终结符的 $\FOLLOW = \varnothing$。在整个算法中,每个非终结符的 \FOLLOW 将会越来越充实,不会重置;而 $temp$ 可能会重置。

以下表 \ref{tab:FOLLOW-1}、表 \ref{tab:FOLLOW-2} 是计算 \FOLLOW 集的详细过程。计算到第 3 轮时,已经没有任何 \FOLLOW 集发生变化。阅读表格时,请依次从上往下阅读,注意盯住每一个符号,不要看串行了。


\begin{landscape}
\begin{table}[p]
    \centering
    \small
    \begin{tabular}{|c|c|l|l|}
    \hline
        产生式 & $\beta_i$ & 更新 $\FOLLOW$ & 更新 $temp$ 为\\
    \hline
        $S\to MH$ & - & - & $\FOLLOW(S) = \varnothing$ \\
                  & $H$ & $\FOLLOW(H)\gets \FOLLOW(H)\cup\varnothing = \varnothing$ & $temp\cup \FIRST(H) = \{e\}$\\
                  & $M$ & $\FOLLOW(M)\gets \FOLLOW(M)\cup\{e\} = \{e\}$ & $temp\cup\FIRST(M) = \{b,d,e\}$ \\
    \hline
        $S\to a$ & $a$ & - & $\{a\}$ \\
    \hline
        $H\to LSo$ & - & - & $\FOLLOW(H) = \varnothing$ \\
                   & $o$ & - & $\{o\}$ \\
                   & $S$ & $\FOLLOW(S)\gets \FOLLOW(S)\cup \{o\} = \{o\}$ & $temp\cup\FIRST(S) = \{a,b,d,e,o\}$ \\
                   & $L$ & $\FOLLOW(L)\gets \FOLLOW(L)\cup \{a,b,d,e,o\} = \{a,b,d,e,o\}$ & $\FIRST(L) = \{e\}$ \\
    \hline
        $H\to \varepsilon$ &&&\\
    \hline
        $K\to dML$ & - & - & $\FOLLOW(K) = \varnothing$ \\
                   & $L$ & $\FOLLOW(L)\gets \FOLLOW(L)\cup\varnothing = \{a,b,d,e,o\}$  & $\FIRST(L) = \{e\}$ \\
                   & $M$ & $\FOLLOW(M)\gets \FOLLOW(M)\cup \{e\} = \{e\}$ & $temp\cup\FIRST(M) = \{b,d,e\}$ \\
                   & $d$ & - & $\{d\}$ \\
    \hline
        $K\to \varepsilon$ &&&\\
    \hline
        $L\to eHf$ & - & - & $\FOLLOW(L) = \{a,b,d,e,o\}$ \\
                   & $f$ & - & $\{f\}$ \\
                   & $H$ & $\FOLLOW(H) \gets \FOLLOW(H) \cup\{f\} = \{f\}$ & $temp\cup\FIRST(H) = \{e,f\}$ \\
                   & $e$ & - & $\{e\}$ \\
    \hline
        $M\to K$ & - & - & $\FOLLOW(M) = \{e\}$ \\
                 & $K$ & $\FOLLOW(K)\gets \FOLLOW(K)\cup\FOLLOW(M) = \{e\}$ & $temp\cup\FIRST(K) = \{d,e\}$ \\
    \hline
        $M\to bLM$ & - & - & $\FOLLOW(M) = \{e\}$ \\
                   & $M$ & $\FOLLOW(M)\gets \FOLLOW(M)\cup\{e\} = \{e\}$ & $temp\cup \FIRST(M) = \{b,d,e\}$ \\
                   & $L$ & $\FOLLOW(L)\gets \FOLLOW(L)\cup\{b,d,e\} = \{a,b,d,e,o\}$ & $\FIRST(L) = \{e\}$ \\
                   & $b$ & - & $\{b\}$ \\
    \hline
    \end{tabular}
    \caption{\FOLLOW 集的第 1 轮迭代}\label{tab:FOLLOW-1}
\end{table}

\begin{table}[p] % 第 2 轮
    \centering
    \small
    \begin{tabular}{|c|c|l|l|}
    \hline
        产生式 & $\beta_i$ & 更新 $\FOLLOW$ & 更新 $temp$ 为\\
    \hline
        $S\to MH$ & - & - & $\FOLLOW(S) = \{o\}$ \\
                  & $H$ & $\FOLLOW(H)\gets \FOLLOW(H)\cup\{o\} = \textcolor{red}{\{f,o\}}$ & $temp\cup \FIRST(H) = \{e,o\}$\\
                  & $M$ & $\FOLLOW(M)\gets \FOLLOW(M)\cup\{e,o\} = \textcolor{red}{\{e,o\}}$ & $temp\cup\FIRST(M) = \{b,d,e,o\}$ \\
    \hline
        $S\to a$ & $a$ & - & $\{a\}$ \\
    \hline
        $H\to LSo$ & - & - & $\FOLLOW(H) = \{f,o\}$ \\
                   & $o$ & - & $\{o\}$ \\
                   & $S$ & $\FOLLOW(S)\gets \FOLLOW(S)\cup \{o\} = \{o\}$ & $temp\cup\FIRST(S) = \{a,b,d,e,o\}$ \\
                   & $L$ & $\FOLLOW(L)\gets \FOLLOW(L)\cup \{a,b,d,e,o\} = \{a,b,d,e,o\}$ & $\FIRST(L) = \{e\}$ \\
    \hline
        $H\to \varepsilon$ &&&\\
    \hline
        $K\to dML$ & - & - & $\FOLLOW(K) = \{e\}$ \\
                   & $L$ & $\FOLLOW(L)\gets \FOLLOW(L)\cup\{e\} = \{a,b,d,e,o\}$  & $\FIRST(L) = \{e\}$ \\
                   & $M$ & $\FOLLOW(M)\gets \FOLLOW(M)\cup \{e\} = \textcolor{red}{\{e,o\}}$ & $temp\cup\FIRST(M) = \{b,d,e\}$ \\
                   & $d$ & - & $\{d\}$ \\
    \hline
        $K\to \varepsilon$ &&&\\
    \hline
        $L\to eHf$ & - & - & $\FOLLOW(L) = \{a,b,d,e,o\}$ \\
                   & $f$ & - & $\{f\}$ \\
                   & $H$ & $\FOLLOW(H) \gets \FOLLOW(H) \cup\{f\} = \textcolor{red}{\{f,o\}}$ & $temp\cup\FIRST(H) = \{e,f\}$ \\
                   & $e$ & - & $\{e\}$ \\
    \hline
        $M\to K$ & - & - & $\FOLLOW(M) = \{e,o\}$ \\
                 & $K$ & $\FOLLOW(K)\gets \FOLLOW(K)\cup\FOLLOW(M) = \textcolor{red}{\{e,o\}}$ & $temp\cup\FIRST(K) = \{d,e,o\}$ \\
    \hline
        $M\to bLM$ & - & - & $\FOLLOW(M) = \{e,o\}$ \\
                   & $M$ & $\FOLLOW(M)\gets \FOLLOW(M)\cup\{e,o\} = \textcolor{red}{\{e,o\}}$ & $temp\cup\FIRST(M) = \{b,d,e,o\}$ \\
                   & $L$ & $\FOLLOW(L)\gets \FOLLOW(L)\cup\{b,d,e,o\} = \textcolor{red}{\{a,b,d,e,o\}}$ & $\FIRST(L) = \{e\}$ \\
                   & $b$ & - & $\{b\}$ \\
    \hline
    \end{tabular}
    \caption{\FOLLOW 集的第 2 轮迭代(\textcolor{red}{红色}文字代表相比于第 1 轮的 \FOLLOW 集的变化)}\label{tab:FOLLOW-2}
\end{table}

\end{landscape}

最终,计算得到的 \FOLLOW 集表如下表 \ref{tab:FOLLOW-1-RES} 所示。能坚持到这里,很不容易,不妨给自己点个赞吧!

\begin{table}[htb]
    \centering
    \begin{tabular}{cccccc}
    \toprule
        非终结符 & $S$ & $H$ & $K$ & $L$ & $M$ \\
    \midrule
        \FOLLOW 集·第 1 轮 & $\{o\}$ & $\{f\}$ & $\{e\}$ & $\{a,b,d,e,o\}$ & $\{e\}$ \\
        \FOLLOW 集·第 2 轮 & $\{o\}$ & $\{f,o\}$ & $\{e,o\}$ & $\{a,b,d,e,o\}$ & $\{e,o\}$ \\
        \FOLLOW 集·第 3 轮 & $\{o\}$ & $\{f,o\}$ & $\{e,o\}$ & $\{a,b,d,e,o\}$ & $\{e,o\}$ \\
    \bottomrule
    \end{tabular}
    \caption{\FOLLOW 集结果}\label{tab:FOLLOW-1-RES}
\end{table}

\section{求 \FIRSTS 集,下结论}
\Emph{\FIRSTS 集}(有时也叫 \textsc{Select} 集),表示一条产生式可以推出来的第一个终结符的集合。

\begin{algorithm}[H]
\DontPrintSemicolon
\SetKwInOut{Input}{输入}
\SetKwInOut{Output}{输出}
\Input{产生式 \textcolor{violet}{$p:N\to \beta_1\beta_2\cdots\beta_n$}}
\Output{$\FIRSTS(p)$}

\For(\textcolor{cyan}{\tcc*[h]{顺序遍历产生式的每个符号}}){$\beta_i\gets \beta_1$ \ \KwTo\ $\beta_n$}{
    \If{$\beta_i = a$}{
        $\FIRSTS(p) \gets \FIRSTS(p)\cup \{a\}$ \;
        \textbf{return}\;
    }
    \If{$\beta_i = M$}{
        $\FIRSTS(p) \gets \FIRSTS(p) \cup \FIRST(M)$\;
        \If{$M\notin \NULLABLE$}{ \textbf{return}\; }
    }
}
$\FIRSTS(p) \gets \FIRSTS(p) \cup \FOLLOW(N)$\;
\textbf{return}\;
\caption{\FIRSTS 集的计算}
\end{algorithm}

通过先前的 \FIRST 和 \FOLLOW 集(集中列出于表 \ref{tab: FIRST_AND_FOLLOW}),可以得到每个产生式的 \FIRSTS 集,如表 \ref{tab:FIRSTS} 所示。

于是,我们就可以填写 \Emph{LL(1) 分析表}了,如表 \ref{tab:RESULT} 所示。

\begin{table}[p]
    \centering
    \begin{tabular}{cp{0.3\textwidth}p{0.3\textwidth}}
    \toprule
        & \FIRST & \FOLLOW \\
    \midrule
    $S$ & $\{a,b,d,e\}$ & $\{o\}$\\
    $H$ & $\{e\}$ & $\{f,o\}$\\
    $K$ & $\{d\}$& $\{e,o\}$\\
    $L$ & $\{e\}$& $\{a,b,d,e,o\}$\\
    $M$ & $\{b,d\}$& $\{e,o\}$\\
    \bottomrule
    \end{tabular}
    \caption{\FIRST 和 \FOLLOW 集结果}\label{tab: FIRST_AND_FOLLOW}
\end{table}

\begin{table}[p]
    \centering
    \begin{tabular}{clll}
    \toprule
        & 产生式 & 展开式 & 结果 \\
    \midrule
    1 & $\FIRSTS(S\to MH)$ & $\FIRST(M)\cup\FIRST(H)\cup\FOLLOW(S)$ & $\{b,d,e,o\}$ \\
    2 & $\FIRSTS(S\to a)$ & - & $\{a\}$ \\
    3 & $\FIRSTS(H\to LSo)$ & $\FIRST(L)$ & $\{e\}$ \\
    4 & $\FIRSTS(H\to\varepsilon)$ & $\FOLLOW(H) $ & $ \{f,o\}$ \\
    5 & $\FIRSTS(K\to dML)$ & - & $\{d\}$ \\
    6 & $\FIRSTS(K\to\varepsilon) $ & $ \FOLLOW(K)$ & $ \{e,o\}$ \\
    7 & $\FIRSTS(L\to eHf)$ & - & $\{e\}$ \\
    8 & $\FIRSTS(M\to K)$ & $\FIRST(K)\cup\FOLLOW(M)$ & $\{d,e,o\}$ \\
    9 & $\FIRSTS(M\to bLM)$ & - & $\{b\}$ \\
    \bottomrule
    \end{tabular}
    \caption{\FIRSTS 集结果}\label{tab:FIRSTS}
\end{table}

\begin{table}[p]
    \centering
    \begin{tabular}{c|cccccc}
    \toprule
        & $a$ & $b$ & $d$ & $e$ & $f$ & $o$ \\
    \midrule
    $S$ & $\to a$ & $\to MH$ & $\to MH$ & $\to MH$ & - & $\to MH$ \\
    $H$ & - & - & - & $\to LSo$ & $\to\varepsilon$ & $\to\varepsilon$ \\
    $K$ & - & - & $\to dML$ & $\to\varepsilon$ & - & $\to\varepsilon$ \\
    $L$ & - & - & - & $\to eHf$ & - & - \\
    $M$ & - & $\to bLM$ & $\to K$ & $\to K$ & - & $\to K$\\
    \bottomrule
    \end{tabular}
    \caption{LL(1) 分析表结果}\label{tab:RESULT}
\end{table}

\newpage

最后判别是不是 LL(1) 文法。

\begin{tcolorbox}[colback=violet!5, colframe=violet, boxrule=1pt, ]
一个上下文无关文法是 LL(1) 文法的充分必要条件是,对每个非终结符 $A$ 的两条产生式 $A\to \alpha$、$A\to \beta$(其中 $\alpha,\beta$ 不同时 $\deduce\varepsilon$),均有
\begin{equation*}
    \FIRSTS(A\to\alpha) \cap \FIRSTS(A\to\beta) = \varnothing
\end{equation*}
反映在分析表上,就是看是否存在同一表项中至少有 2 个产生式(即构成冲突)。如果没有冲突,那就是 LL(1) 文法;如果有冲突,就要对文法进行等价变换,比如消除左递归和提取左公因子。
\end{tcolorbox}

通过表 \ref{tab:RESULT} 可知,没有冲突,所以是 LL(1) 文法。

\backgroundsetup{contents=\includegraphics{下半示例.png}, center, scale=1, angle=0, opacity=1}
\BgThispage
\vspace{1em}
{\color{cyan!80!black}
【结论】1. 如表 \ref{tab: FIRST_AND_FOLLOW} 所示;2. 如表 \ref{tab:RESULT} 所示;3. 是 LL(1) 文法。

【点评】这是一道编译原理的大题,考察了 LL(1) 文法的分析方法,即依次求解 \NULLABLE—\FIRST—\FOLLOW—\FIRSTS—LL(1)分析表。本题过程繁多,持续时间也很长,需要同学们求解时保持清醒,尤其是算 \FOLLOW 集时更要打起十二分精神。由于给出了详细过程,本文档篇幅较长。

需要指出的是,本文档采用孙景昊式解法:对于每一个非终结符,默认其 \FIRST 集不含 $\varepsilon$,\FOLLOW 集不含 $\#$,可能与《编译原理》课本的结果有所不同。
}

\end{document} 