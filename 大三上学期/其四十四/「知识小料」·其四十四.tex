\documentclass[UTF8]{ctexart}
\usepackage{amsmath}
\usepackage{amssymb}
\usepackage{background}
\usepackage{booktabs}
\usepackage{caption}
\usepackage{chngcntr} % 本宏包用于把图片和章节序号相关联
\usepackage{CJKfntef}
\usepackage{enumitem}
\usepackage{fancyhdr}
\usepackage{float}
\usepackage{fontspec}
\usepackage{geometry}
\usepackage{hyperref}
%\usepackage{tasks}
\usepackage{tcolorbox}
\usepackage{tikz}
\usetikzlibrary{arrows.meta, positioning}
\usepackage[table]{xcolor}

\geometry{a5paper, top=0.1cm, left=1cm, right=1cm, bottom=1cm, footskip=0.1cm}
\setCJKmainfont[BoldFont={汉仪文黑-85W},ItalicFont={汉仪文黑-55W}]{汉仪文黑-55W}
\setfontfamily\Issue{Century Schoolbook}
\setfontfamily\Genshin{Genshin Teyvat Lingua Franca}
\newCJKfontfamily\TitleFont{思源宋体 CN Heavy}
\newfontfamily\timesnewroman{Times New Roman}

%————————————————可变部分——————————————————
%\settasks{label={\Alph*.\ }, label-format={\color{cyan!50!black}}, item-format={\color{cyan!50!black}}}
%\newcommand\col[1]{\textcolor{green!50!black}{#1}}
%\newcommand\coll[1]{\textcolor{blue}{#1}}
%\newcommand\dotting{\ .\ }
\newcommand\keyword[1]{\textcolor{darkcyan}{#1}}
\setlist[enumerate]{itemsep=0pt, parsep=0pt}

\newtcolorbox{summary}[1][内容概述]{colback=violet!10, colframe=violet!70!black, boxrule=1pt, title={#1}}
\newtcolorbox{double_segment}[1][警告:双字段匹配]{colback=green!10, colframe=green!30!black, boxrule=1pt, title={#1}}
\counterwithin{figure}{section}
\captionsetup[figure]{font=small, labelfont=bf}
\hypersetup{bookmarksnumbered = true, pdfborder = 0 0 0}
%——————————————————————————————————————————

\pagestyle{fancy}
\fancyhf{}
\cfoot{\sffamily\footnotesize{-\ \thepage\ -}}
%\CTEXsetup[format = {\centering\bfseries\large}, beforeskip = 3pt, afterskip = 3pt]{section}

\colorlet{darkcyan}{cyan!50!black}
\newcommand\Black[1]{\textcolor[gray]{0.3}{#1}}
\newcommand\Brown[1]{\textcolor[HTML]{998A4E}{#1}}
\newcommand\Emph[1]{\colorbox{green!10}{\textcolor{green!30!black}{#1}}}
\newcommand\Notes[1]{\textcolor{yellow!50!black}{\small #1}}
\newcommand\Example[1]{\textcolor{cyan!70!black}{\small #1}}
\colorlet{note}{yellow!70!black}


\newcommand\IssueNumber{44}
\newcommand\Date{2024-12-5}
%\newcommand\Contributer{@金光日}
\newcommand\Subject{习思想}
%\newcommand\Source{历年考研 408 真题}


\begin{document}
\backgroundsetup{contents=\includegraphics{上半示例.png}, center, scale=1, angle=0, opacity=1}
\BgThispage
\begin{center}
%{\scriptsize\Issue \textcolor[HTML]{C8BA83}{\Genshin WEEKLY TIPS}}
\phantom{...}

{\Large\textcolor{brown!40!white}{\makebox[10cm][s]{\Genshin WEEKLY KNOWLEDGE TIPS}}}

\vspace{-2em}

{\Huge\bfseries\TitleFont \Black{知\ 识\ 小\ 料}}


\vspace{-0.1cm}
{\footnotesize \Brown{「电计 2203 班」周常规知识整理共享}}
\end{center}

\vspace{-0.5cm}


\begin{figure}[H]
\hspace{1cm}
\begin{minipage}[t]{0.3\textwidth}
\centering
    \Brown{\Genshin ISSUE}

    \vspace{-0.6cm}
    \Huge \Issue\slshape\bfseries\Black{\IssueNumber}
\end{minipage}
\hfill
\begin{minipage}[t]{0.3\textwidth}
\centering
    \Brown{日期:\Date} \\
%\vspace{-0.1cm}
%    \Brown{贡献者:\Contributer} \\
\vspace{-0.1cm}
    \Brown{学科:\Subject} \\
%\vspace{-0.1cm}
%   \Brown{来源:\Source}
\end{minipage}
\hspace{0.8cm}
\end{figure}

{\color{darkcyan}
本文整理与《习近平新时代中国特色社会主义思想概论》(以下简称习思想)相关的关系图。由于篇幅较大,分成上、下两期。本期为下期,收录索引值十六进制为 $\mathrm{[0x08,\  0x11]}$ 的章节相关内容。
}

\begin{table}[htb]
    \small
    \centering
    \rowcolors{2}{violet!10}{violet!20}
    \begin{tabular}{ccl}
        \rowcolor{violet!40} 十六进制 & 十进制 & 标题字段 \\
        0x00 & 0 & \textcolor{violet!50}{(该字段为空)}\\
        0x01 & 1 & 新时代坚持和发展\keyword{中国特色社会主义} \\
        0x02 & 2 & 以\keyword{中国式现代化}全面推进中华民族伟大复兴\\
        0x03 & 3 & 坚持\keyword{党的全面领导}\\
        0x04 & 4 & 坚持以\keyword{人民}为中心\\
        0x05 & 5 & 全面深化\keyword{改革开放}\\
        0x06 & 6 & 推动高质量\keyword{发展}\\
        0x07 & 7 & 社会主义现代化建设的\keyword{教育、科技、人才}战略\\
        0x08 & 8 & 发展全过程人民\keyword{民主}\\
        0x09 & 9 & 全面\keyword{依法治国}\\
        0x0a & 10 & 建设社会主义\keyword{文化强国}\\
        0x0b & 11 & 以保障和改善民生为重点加强\keyword{社会建设}\\
        0x0c & 12 & 建设社会主义\keyword{生态文明}\\
        0x0d & 13 & 全面贯彻落实\keyword{总体国家安全观}\\
        0x0e & 14 & 建设巩固\keyword{国防}和强大人民\keyword{军队}\\
        0x0f & 15 & 坚持「\keyword{一国两制}」和推进祖国\keyword{完全统一}\\
        0x10 & 16 & 中国特色大国\keyword{外交}和推动构建人类命运共同体\\
        0x11 & 17 & 全面\keyword{从严治党}\\
    \end{tabular}
\end{table}

\newpage
\backgroundsetup{contents=\includegraphics{空白示例.png}, center, scale=1, angle=0, opacity=1}
\BgThispage
\setcounter{section}{7}
\section{发展全过程人民民主}\label{sec:8}

\begin{summary}
\begin{enumerate}
  \item 人民民主与全过程人民民主
  \item 我国政治制度、国体与政体
  \item 协商民主与基层民主
  \item 统一战线与民族团结
\end{enumerate}
\end{summary}

\begin{figure}[H]
    \centering
    \begin{tikzpicture}[
        seq/.style = {circle, fill=cyan!20, draw=cyan!50, thick},
        item/.style = {rectangle, fill=blue!10, draw=blue!30, thick, rounded corners=2mm, minimum size=0.6cm, text width=3cm, align=center},
        prop/.style = {rectangle, fill=violet!10, draw=violet!30, thick, rounded corners=2mm, minimum size=0.6cm, text width=6.8cm, align=left}, % property的缩写
        node distance=0.4cm,
        font=\small,
        >=Stealth]

        % 铺垫形状
        \node [seq] (seq1) {\color{darkcyan}是};
        \foreach \x in {2,3}{
            \pgfmathtruncatemacro{\prev}{\x-1} % 使用一个宏来预处理上一个节点的编号
            \node [seq, below=of seq\prev] (seq\x)  {\color{darkcyan}是};
            \draw [cyan!50, thick] (seq\x) to (seq\prev);
        }
        \node[item, left=of seq1] (item1){人民民主};
        \node[item, left=of seq2] (item2){全过程人民民主};
        \node[item, left=of seq3] (item3){人民当家作主};
        \node[prop, right=of seq1](prop1){社会主义的生命};
        \node[prop, right=of seq2](prop2){社会主义民主政治的本质属性};
        \node[prop, right=of seq3](prop3){社会主义民主政治的本质特征(本质与核心)};
        \foreach \x in {1,2,3}{
            \draw[blue!30] (seq\x) to (item\x);
            \draw[violet!30] (seq\x) to (prop\x);
        }
    \end{tikzpicture}
    \caption{人民民主等基本定义}\label{fig:sec8-基本定义}
\end{figure}


\begin{figure}[H]
    \centering
    \begin{tikzpicture}[
        seq/.style = {circle, fill=cyan!20, draw=cyan!50, thick, minimum size=2cm},
        item/.style = {rectangle, fill=blue!10, draw=blue!30, thick, rounded corners=2mm},
        prop/.style = {rectangle, fill=violet!10, draw=violet!30, thick, rounded corners=2mm}, % property的缩写
        node distance=0.4cm,
        font=\small,
        >=Stealth]

        % 使用极坐标系布置等边三角形节点
        % 等边三角形的每个内角是60度,所以三个节点分别位于90度、210度和330度
        \node[seq] (A) at (90:2cm) {党的领导}; % 半径可以调整以匹配您想要的三角形大小
        \node[seq] (B) at (210:2cm) {人民当家作主};
        \node[seq] (C) at (330:2cm) {依法治国};
        \draw[cyan!50, thick] (A) -- (B) -- (C) -- (A);
        \node[item] (A1) at (90:2.7cm) {根本保证};
        \node[item] (B1) at (210:3cm) {本质特征};
        \node[item] (C1) at (330:3cm) {基本方式};
    \end{tikzpicture}
    \caption{党的领导、人民当家作主、依法治国有机统一}\label{fig:sec8-党的领导、人民当家作主、依法治国有机统一}
\end{figure}

统一战线工作的本质要求是「大团结大联合」,关键是坚持求同存异,其最核心最根本的问题是坚持党的领导。

\section{全面依法治国}\label{sec:9}
\begin{summary}
\begin{enumerate}
  \item 法治、依法治国、全面依法治国的基本定义
  \item 习近平法治思想「十一个坚持」
  \item 法治体系:法律规范、法治实施、法治监督、法治保障、党内法规体系
  \item 建设法治中国
\end{enumerate}
\end{summary}

\begin{figure}[H]
    \centering
    \begin{tikzpicture}[
        seq/.style = {circle, fill=cyan!20, draw=cyan!50, thick},
        item/.style = {rectangle, fill=blue!10, draw=blue!30, thick, rounded corners=2mm, text width=2cm, align=right},
        prop/.style = {rectangle, fill=violet!10, draw=violet!30, thick, rounded corners=2mm,  text width=7.5cm, align=left}, % property的缩写
        node distance=0.4cm,
        font=\small,
        >=Stealth]

        % 铺垫形状
        \node [seq] (seq1) {\color{darkcyan}是};
        \foreach \x in {2,3}{
            \pgfmathtruncatemacro{\prev}{\x-1} % 使用一个宏来预处理上一个节点的编号
            \node [seq, below=of seq\prev] (seq\x)  {\color{darkcyan}是};
            \draw [cyan!50, thick] (seq\x) to (seq\prev);
        }
        \node[item, left=of seq1] (item1){法治};
        \node[item, left=of seq2] (item2){法治体系};
        \node[item, left=of seq3] (item3){全面依法治国};
        \node[prop, right=of seq1](prop1){治国理政的基本方式};
        \node[prop, right=of seq2](prop2){国家治理体系的骨干工程};
        \node[prop, right=of seq3](prop3){坚持和发展中国特色社会主义的本质要求和重要保障};
        \foreach \x in {1,2,3}{
            \draw[blue!30] (seq\x) to (item\x);
            \draw[violet!30] (seq\x) to (prop\x);
        }
    \end{tikzpicture}
    \caption{依法治国等基本定义}\label{fig:sec9-基本定义}
\end{figure}

\begin{figure}[H]
    \centering
    \begin{tikzpicture}[
        seq/.style = {circle, fill=cyan!20, draw=cyan!50, thick},
        item/.style = {rectangle, fill=blue!10, draw=blue!30, thick, rounded corners=2mm,},
        prop/.style = {rectangle, fill=violet!10, draw=violet!30, thick, rounded corners=2mm,  text width=5cm, align=left}, % property的缩写
        node distance=0.4cm,
        font=\small,
        >=Stealth]

        % 铺垫形状
        \node [seq] (seq1) {\color{darkcyan}是};
        \foreach \x in {2}{
            \pgfmathtruncatemacro{\prev}{\x-1} % 使用一个宏来预处理上一个节点的编号
            \node [seq, below=of seq\prev] (seq\x)  {\color{darkcyan}是};
            \draw [cyan!50, thick] (seq\x) to (seq\prev);
        }
        \node[item, left=of seq1] (item1){{\footnotesize 全面推进依法治国的}总目标};
        \node[item, left=of seq2] (item2){{\footnotesize 全面推进依法治国的}总抓手};
        \node[prop, right=of seq1](prop1){建设社会主义法治国家};
        \node[prop, right=of seq2](prop2){建设中国特色社会主义法治体系};
        \foreach \x in {1,2}{
            \draw[blue!30] (seq\x) to (item\x);
            \draw[violet!30] (seq\x) to (prop\x);
        }
    \end{tikzpicture}
    \caption{全面推进依法治国的「总目标」和「总抓手」}\label{fig:sec9-全面推进依法治国的「总目标」和「总抓手」}
\end{figure}

\begin{figure}[H]
    \centering
    \begin{tikzpicture}[
        seq/.style = {circle, fill=cyan!20, draw=cyan!50, thick},
        item/.style = {rectangle, fill=blue!10, draw=blue!30, thick, rounded corners=2mm,text width=5cm, align=center},
        prop/.style = {rectangle, fill=violet!10, draw=violet!30, thick, rounded corners=2mm,  text width=6cm, align=left}, % property的缩写
        node distance=0.4cm,
        font=\small,
        >=Stealth]

        % 铺垫形状
        \node [seq] (seq1) {\color{darkcyan}1};
        \foreach \x in {2,3}{
            \pgfmathtruncatemacro{\prev}{\x-1} % 使用一个宏来预处理上一个节点的编号
            \node [seq, below=of seq\prev] (seq\x)  {\color{darkcyan}\x};
            \draw [cyan!50, thick] (seq\x) to (seq\prev);
        }
        \node[item, right=of seq1] (item1){坚持党对依法治国的领导};
        \node[item, right=of seq2] (item2){坚持以人民为中心};
        \node[item, right=of seq3] (item3){坚持中国特色社会主义法治道路};
        \foreach \x in {1,2,3}{
            \draw[blue!30] (seq\x) to (item\x);
            \draw[violet!30] (item\x) to (prop\x);
        }
    \end{tikzpicture}
    \caption{习近平法治思想「十一个坚持」的前三条内容}\label{fig:sec9-习近平法治思想}
\end{figure}

中国特色社会主义法治道路,是全面依法治国的\textcolor{red}{唯一}正确道路。

\begin{figure}[H]
    \centering
    \begin{tikzpicture}[
        general/.style = {circle, fill=cyan!20, draw=cyan!50, thick},
        item/.style = {rectangle, fill=blue!10, draw=blue!30, thick, rounded corners=2mm, minimum size=0.7cm, text width=3cm, align=center},
        prop/.style = {rectangle, fill=violet!10, draw=violet!30, thick, rounded corners=2mm, }, % property的缩写
        node distance=0.4cm,
        font=\small,
        >=Stealth]
        \node[item] (item1) {法律规范体系};
        \node[item, below=of item1] (item2) {法治实施体系};
        \node[item, below=of item2] (item3) {法治监督体系};
        \node[item, below=of item3] (item4) {法治保障体系};
        \node[item, below=of item4] (item5) {党内法规体系};
        \node[general, left=1cm of item3] (general) {法治体系};
        \node[prop, right=of item1] (prop1) {前提、制度基础};
        \node[prop, right=of item2] (prop2) {重点};
        %\node[prop, right=of item3] (prop3) {全心全意为人民服务};
        %\node[prop, right=of item4] (prop4) {坚持以人民为中心};
        \node[prop, right=of item5] (prop5) {本质要求};
        \draw[thick, cyan!50, ->] (general) edge[bend left, looseness=1] (item1.west);
        \draw[thick, cyan!50, ->] (general) edge[bend left, looseness=0.8] (item2.west);
        \draw[thick, cyan!50, ->] (general) to (item3);
        \draw[thick, cyan!50, ->] (general) edge[bend right, looseness=0.8] (item4.west);
        \draw[thick, cyan!50, ->] (general) edge[bend right, looseness=1] (item5.west);
        \foreach\x in {1,2,5}{
            \draw[blue!30] (item\x) to (prop\x);
        }
    \end{tikzpicture}
    \caption{中国特色社会主义法治体系}\label{fig:sec9-法治体系}
\end{figure}

\begin{figure}[H]
    \centering
    \begin{tikzpicture}[
        general/.style = {circle, fill=cyan!20, draw=cyan!50, thick},
        item/.style = {rectangle, fill=blue!10, draw=blue!30, thick, rounded corners=2mm, minimum size=0.7cm, text width=1.5cm, align=center},
        prop/.style = {rectangle, fill=violet!10, draw=violet!30, thick, rounded corners=2mm, text width = 7cm, align = left}, % property的缩写
        node distance=0.4cm,
        font=\small,
        >=Stealth]
        \node[item] (item1) {总目标};
        \node[item, below=1.5cm of item1] (item2) {工作布局};
        \node[prop, right=of item1] (goal) {\footnotesize 法律规范,执法司法,权力运行,人民权益,法治信仰,全面建成(总共6点,简写)};
        \node[prop, right=of item2] (prop2) {2. 法治国家、法治政府、法治社会一体建设};
        \node[prop, above=of prop2] (prop1) {1. 依法治国、依法执政、依法行政共同推进};
        \node[prop, below=of prop2] (prop3) {3. 推进国内法治和涉外法治};
        \node[general, left=1cm of item1, yshift=-1.2cm] (general) {法治中国};
        \draw[thick, cyan!50, ->] (general) edge[bend left, looseness=1] (item1.west);
        \draw[thick, cyan!50, ->] (general) edge[bend right, looseness=1] (item2.west);
        \draw[violet!30] (item1) -- (goal);
        \draw[violet!30] (item2) -- (prop2);
        \draw[violet!30] (prop1.west) -- (item2);
        \draw[violet!30] (prop3.west) -- (item2);
    \end{tikzpicture}
    \caption{建设法治中国(工作布局有两处出现了「三条目」,多选题注意)}\label{fig:sec9-建设法治中国}
\end{figure}

此外,建设法治中国,需要推进科学立法、严格执法、公正司法、全民守法,这也是习近平法治思想「十一个坚持」之一(但不属于建设法治中国的工作布局)。

\newpage
\section{建设社会主义文化强国}\label{sec:10}
\begin{summary}
\begin{enumerate}
  \item 文化与文化自信、三种文化各自的作用
  \item 根本文化制度:马克思主义在意识形态领域的指导地位
  \item 核心价值观、中国共产党人精神谱系
  \item 中华文明的五个特性等
\end{enumerate}
\end{summary}

\begin{figure}[H]
    \centering
    \begin{tikzpicture}[
        general/.style = {circle, fill=cyan!20, draw=cyan!50, thick},
        item/.style = {rectangle, fill=blue!10, draw=blue!30, thick, rounded corners=2mm, minimum size=0.7cm, text width=7cm, align=center},
        prop/.style = {rectangle, fill=violet!10, draw=violet!30, thick, rounded corners=2mm, text width = 7cm, align = left}, % property的缩写
        node distance=0.4cm,
        font=\small,
        >=Stealth]
        \node[item] (item1) {更基础、更广泛、更深厚的自信};
        \node[item, below=of item1] (item2) {最基本、最深沉、最持久的力量};
        \node[item, below=of item2] (item3) {\footnotesize 坚定其他三个自信,说到底就是要坚定文化自信$^{\mathrm{0x01}}$};
        \node[general, left=1cm of item2] (general) {文化自信};
        \draw[thick, cyan!50, ->] (general) edge[bend left, looseness=1] (item1.west);
        \draw[thick, cyan!50, ->] (general) to (item2);
        \draw[thick, cyan!50, ->] (general) edge[bend right, looseness=1] (item3.west);
    \end{tikzpicture}
    \caption{文化自信在「四个自信」中的地位}\label{fig:sec10-文化自信}
\end{figure}

\begin{figure}[H]
    \centering
    \begin{tikzpicture}[
        seq/.style = {circle, fill=cyan!20, draw=cyan!50, thick},
        item/.style = {rectangle, fill=blue!10, draw=blue!30, thick, rounded corners=2mm, text width=5cm, align=right, minimum size=0.6cm},
        prop/.style = {rectangle, fill=violet!10, draw=violet!30, thick, rounded corners=2mm, align=left, minimum size=0.6cm}, % property的缩写
        node distance=0.4cm,
        font=\small,
        >=Stealth]

        % 铺垫形状
        \node [seq] (seq1) {\color{darkcyan}远};
        \node [seq, below=of seq1] (seq2) {\color{darkcyan}中};
        \node [seq, below=of seq2] (seq3) {\color{darkcyan}近};
        \node[item, left=of seq1] (item1){中华优秀传统文化};
        \node[item, left=of seq2] (item2){革命文化、社会主义先进文化};
        \node[item, left=of seq3] (item3){中国特色社会主义伟大实践};
        \node[prop, right=of seq1](prop1){深厚基础};
        \node[prop, right=of seq2](prop2){坚强基石};
        \node[prop, right=of seq3](prop3){现实基础};
        \foreach \x in {1,2,3}{
            \draw[blue!30] (seq\x) to (item\x);
            \draw[violet!30] (seq\x) to (prop\x);
        }
    \end{tikzpicture}
    \caption{三种文化及其定性}\label{fig:sec10-三种文化}
\end{figure}

根本文化制度:坚持马克思主义在意识形态领域指导地位的制度。

中国共产党人精神谱系内容广泛,以\textcolor{red}{伟大建党精神}为源头。

互联网是意识形态工作的主阵地、主战场、最前沿,管好用好互联网是新形势下做好新闻舆论工作的关键。核心价值观是一个民族赖以维系的精神纽带,是一个国家共同的思想道德基础。

\begin{figure}[H]
    \centering
    \begin{tikzpicture}[
        general/.style = {circle, fill=cyan!20, draw=cyan!50, thick},
        item/.style = {rectangle, fill=blue!10, draw=blue!30, thick, rounded corners=2mm, minimum size=0.7cm, text width=2cm, align=center},
        prop/.style = {rectangle, fill=violet!10, draw=violet!30, thick, rounded corners=2mm, }, % property的缩写
        node distance=0.4cm,
        font=\small,
        >=Stealth]
        \node[item] (item1) {连续性};
        \node[item, below=of item1] (item2) {创新性};
        \node[item, below=of item2] (item3) {统一性};
        \node[item, below=of item3] (item4) {包容性};
        \node[item, below=of item4] (item5) {和平性};
        \node[general, left=1cm of item3] (general) {中华文明};
        \node[prop, right=of item1] (prop1) {{\footnotesize 决定中华民族}必然走自己的路};
        \node[prop, right=of item2] (prop2) {{\footnotesize 决定中华民族}守正不守旧,遵古不复古};
        \node[prop, right=of item3] (prop3) {{\footnotesize 决定中华民族}各民族文化融为一体};
        \node[prop, right=of item4] (prop4) {{\footnotesize 决定中华民族}交往交融的历史取向};
        \node[prop, right=of item5] (prop5) {{\footnotesize 决定中华民族}中国始终是和平的建设者};
        \draw[thick, cyan!50, ->] (general) edge[bend left, looseness=1] (item1.west);
        \draw[thick, cyan!50, ->] (general) edge[bend left, looseness=0.8] (item2.west);
        \draw[thick, cyan!50, ->] (general) to (item3);
        \draw[thick, cyan!50, ->] (general) edge[bend right, looseness=0.8] (item4.west);
        \draw[thick, cyan!50, ->] (general) edge[bend right, looseness=1] (item5.west);
        \foreach\x in {1,2,5}{
            \draw[blue!30] (item\x) to (prop\x);
        }
    \end{tikzpicture}
    \caption{中华文明的五大特性}\label{fig:sec10-中华文明的五大特性}
\end{figure}

\begin{figure}[H]
    \centering
    \begin{tikzpicture}[
        seq/.style = {circle, fill=cyan!20, draw=cyan!50, thick},
        item/.style = {rectangle, fill=blue!10, draw=blue!30, thick, rounded corners=2mm,text width=6cm, align=left},
        anti/.style = {rectangle, fill=gray!20, draw=gray!40, thick, rounded corners=2mm, text width=6cm, align=left},
        node distance=0.4cm,
        font=\small,
        >=Stealth]

        % 铺垫形状
        \node [seq] (seq1) {\color{darkcyan}1};
        \foreach \x in {2,3,4}{
            \pgfmathtruncatemacro{\prev}{\x-1} % 使用一个宏来预处理上一个节点的编号
            \node [seq, below=of seq\prev] (seq\x)  {\color{darkcyan}\x};
            \draw [cyan!50, thick] (seq\x) to (seq\prev);
        }
        \node[item, right=of seq1] (item1){继续推动文化繁荣};
        \node[item, right=of seq2] (item2){建设文化强国};
        \node[item, right=of seq3] (item3){建设中华民族现代文明};
        \node[anti, right=of seq4] (item4){\color{gray}\CJKsout{弘扬中国精神、传播中国价值、凝聚中国力量}(文艺工作者的职责)};
        \foreach \x in {1,2,3,4}{
            \draw[blue!30] (seq\x) to (item\x);
        }
    \end{tikzpicture}
    \caption{新时代新的文化使命}\label{fig:sec10-新时代新的文化使命}
\end{figure}

\newpage
\section{以保障和改善民生为重点加强社会建设}\label{sec:11}
\begin{summary}
\begin{enumerate}
  \item 民生与增进民生福祉
  \item 民生与发展,民生之源(收入)和最基本的民生(就业)
  \item 社会治理与「枫桥经验」、「浦江经验」
\end{enumerate}
\end{summary}

\begin{figure}[H]
    \centering
    \begin{tikzpicture}[
        item/.style = {rectangle, fill=blue!10, draw=blue!30, thick, rounded corners=2mm, minimum size=0.6cm},
        prop/.style = {rectangle, fill=violet!10, draw=violet!30, thick, rounded corners=2mm, minimum size=0.6cm},
        node distance=2cm,
        font=\small,
        >=Stealth]
        \node[item] (people) {人民};
        \node[item, right=of people] (livelihood) {民生};
        \node[prop, below=1.5cm of livelihood] (develop) {发展};
        \node[item, right=3cm of livelihood, yshift=1cm] (salary) {收入分配};
        \node[item, right=3cm of livelihood, yshift=-1cm] (job) {就业};
        \node[right=0.1cm of salary, font=\footnotesize, blue!50] {(改善民生最直接最重要的方式)};
        \node[right=0.1cm of job, font=\footnotesize, blue!50] {(最大的民生工程、民心工程)};
        \draw[thick, ->, blue!30] (people) -- (livelihood) node[above, midway, font=\footnotesize] {\color{blue}人民幸福之基};
        \draw[thick, ->, blue!30, ] (livelihood) edge[bend left] node[right, font=\footnotesize, midway] {\color{blue}「指南针」} (develop) ;
        \draw[thick, ->, violet!30, ] (develop) edge[bend left] node[left, font=\footnotesize, midway] {\color{violet}「总钥匙」} (livelihood) ;
        \draw[thick, ->, blue!30] (livelihood) -- (salary) node[above, midway, sloped, font=\footnotesize] {\color{blue}民生之源};
        \draw[thick, ->, blue!30] (livelihood) -- (job) node[above, midway, sloped, font=\footnotesize] {\color{blue}最基本的民生};
    \end{tikzpicture}
    \caption{民生相关概念的关系}\label{fig:sec11-民生相关概念}
\end{figure}

\begin{figure}[H]
    \centering
    \begin{tikzpicture}[
        general/.style = {circle, fill=cyan!20, draw=cyan!50, thick, minimum size=2cm},
        item/.style = {rectangle, fill=blue!10, draw=blue!30, thick, rounded corners=2mm, text width=7cm, align=center},
        prop/.style = {rectangle, fill=violet!10, draw=violet!30, thick, rounded corners=2mm, },
        %property的缩写
        anti/.style = {rectangle, fill=gray!20, draw=gray!40, thick, rounded corners=2mm, text width=7cm, align=center},
        node distance=0.4cm,
        font=\small,
        >=Stealth]
        \node[item] (item1) {坚持立党为公、执政为民的本质要求};
        \node[item, below=of item1] (item2) {社会主义生产的根本目的};
        \node[item, below=of item2] (item3) {全面建设社会主义现代化国家的应有之义};
        \node[anti, below=of item3] (item4) {\textcolor{gray}{\CJKsout{推动高质量发展的战略基点}(应为新发展格局)}};
        \node[general, left=1cm of item3, yshift=0.4cm] (general) {增进民生福祉};
        \draw[thick, cyan!50, ] (general.east) edge[bend left] (item1.west);
        \draw[thick, cyan!50, ] (general.east) edge[bend left] (item2.west);
        \draw[thick, cyan!50, ] (general.east) edge[bend right] (item3.west);
        \draw[thick, cyan!50, ] (general.east) edge[bend right] (item4.west);
    \end{tikzpicture}
    \caption{增进民生福祉}\label{fig:sec11-增进民生福祉}
\end{figure}

党的十八届三中全会提出创新社会治理体制。常见的治理经验有「枫桥经验」和「浦江经验」。

\textcolor{gray}{\CJKsout{社会保障是民族复兴的根基、国家强盛的前提?}}(应为国家安全、社会稳定)

\textcolor{gray}{\CJKsout{就业是改善民生、实现发展成果由人民共享的最重要、最直接的方式?}}(应为收入分配)

\begin{figure}[H]
    \centering
    \begin{tikzpicture}[
        general/.style = {circle, fill=cyan!20, draw=cyan!50, thick, text width=1.3cm},
        item/.style = {rectangle, fill=blue!10, draw=blue!30, thick, rounded corners=2mm},
        prop/.style = {rectangle, fill=violet!10, draw=violet!30, thick, rounded corners=2mm, text width=8cm, align=center},
        anti/.style = {rectangle, fill=gray!20, draw=gray!40, thick, rounded corners=2mm, text width=7cm, align=center},
        node distance=0.4cm,
        font=\small,
        >=Stealth]
        \node[item] (item1) {重要前提};
        \node[item, below=of item1] (item2) {工作思路};
        \node[item, below=of item2] (item3) {重中之重};
        \node[item, below=of item3] (item4) {重要方针};
        \node[item, below=of item4] (item5) {重要原则};
        \node[general, left=1cm of item3] (general) {保障和改善民生};
        \node[prop, right=of item1] (prop1) {正确把握民生和发展的关系};
        \node[prop, right=of item2] (prop2) {坚守底线、突出重点、完善制度、引导预期};
        \node[prop, right=of item3] (prop3) {解决人民群众最关心最直接最现实的利益问题};
        \node[prop, right=of item4] (prop4) {尽力而为、量力而行};
        \node[prop, right=of item5] (prop5) {人人尽责、人人享有;让所有劳动者分享发展成果};
        \draw[thick, cyan!50, ] (general.east) edge[bend left] (item1.west);
        \draw[thick, cyan!50, ] (general.east) edge[bend left] (item2.west);
        \draw[thick, cyan!50, ] (general.east) to (item3.west);
        \draw[thick, cyan!50, ] (general.east) edge[bend right] (item4.west);
        \draw[thick, cyan!50, ] (general.east) edge[bend right] (item5.west);
        \foreach\x in {1,2,3,4,5}{
            \draw[violet!30] (item\x) to (prop\x);
        }
    \end{tikzpicture}
    \caption{保障和改善民生}\label{fig:sec11-保障和改善民生}
\end{figure}

\newpage
\section{建设社会主义生态文明}\label{sec:12}
\begin{summary}
\begin{enumerate}
  \item 生态文明建设,绿水青山就是金山银山
  \item 建设美丽中国,山水林田湖草沙一体
  \item 全球生态文明建设
\end{enumerate}
\end{summary}

\begin{figure}[H]
    \centering
    \begin{tikzpicture}[
        seq/.style = {circle, fill=cyan!20, draw=cyan!50, thick},
        content/.style = {rectangle, fill=blue!10, draw=blue!30, thick, rounded corners=2mm},
        node distance=0.4cm,
        font=\small]

        % 铺垫形状
        \node [seq] (seq1) {\color{darkcyan}1};
        \foreach \x in {2,3,4}{
            \pgfmathtruncatemacro{\prev}{\x-1} % 使用一个宏来预处理上一个节点的编号
            \node [seq, right=2cm of seq\prev] (seq\x)  {\color{darkcyan}\x};
            \draw [cyan!50, thick] (seq\x) to (seq\prev);
        }
        %填写文字
        \node [content, below=of seq1] (con1) {自然财富};
        \node [content, below=of seq2] (con2) {生态财富};
        \node [content, below=of seq3] (con3) {社会财富};
        \node [content, below=of seq4] (con4) {经济财富};
        \foreach \x in {1,2,3,4}{
            \draw[dashed, blue!30] (seq\x) to (con\x);
        }
    \end{tikzpicture}
    \caption{绿水青山的地位}
    \label{fig:sec12-绿水青山的地位}
\end{figure}

\begin{figure}[H]
    \centering
    \begin{tikzpicture}[
        general/.style = {circle, fill=cyan!20, draw=cyan!50, thick},
        item/.style = {rectangle, fill=blue!10, draw=blue!30, thick, rounded corners=2mm, minimum size=0.7cm, text width=4cm, align=center},
        prop/.style = {rectangle, fill=violet!10, draw=violet!30, thick, rounded corners=2mm, text width = 4cm, align = left}, % property的缩写
        node distance=0.4cm,
        font=\small,
        >=Stealth]
        \node[item] (item1) {重大经济问题};
        \node[item, below=of item1] (item2) {重大政治问题};
        \node[item, below=of item2] (item3) {重大社会问题};
        \node[general, left=1cm of item2] (general) {生态文明建设};
        \draw[thick, cyan!50, ->] (general) edge[bend left, looseness=1] (item1.west);
        \draw[thick, cyan!50, ->] (general) to (item2);
        \draw[thick, cyan!50, ->] (general) edge[bend right, looseness=1] (item3.west);
    \end{tikzpicture}
    \caption{生态文明建设的地位}\label{fig:sec12-生态文明建设的地位}
\end{figure}

\begin{itemize}
  \item 生态环境问题归根到底是\textcolor{darkcyan}{经济发展方式和生活方式}问题。
  \item 人与自然是一种\textcolor{darkcyan}{共生}关系。
  \item 习近平「两山论」是\textcolor{darkcyan}{马克思主义生产力观}的新成果。
  \item 「三条红线」:生态保护红线、环境质量底线、资源利用上线。\textcolor{cyan}{$^{\text{课本p.247}}$}
  \item 「三条控制线」:生态保护红线、永久基本农田、城镇开发边界(及各类海域保护线)。\textcolor{cyan}{$^{\text{课本p.251}}$}
  \item 生态文明制度体系的四个内涵:产权清晰、多元参与、激励约束并重、系统完整。
\end{itemize}



\newpage
\section{全面贯彻落实总体国家安全观}\label{sec:13}
\begin{summary}
\begin{enumerate}
  \item 总体国家安全观
  \item 统筹发展和安全,把维护政治安全放在首要位置
  \item 维护重点领域国家安全(国土、经济、社会安全等)
\end{enumerate}
\end{summary}

\begin{figure}[H]
    \centering
    \begin{tikzpicture}[
        general/.style = {circle, fill=cyan!20, draw=cyan!50, thick, minimum size=1cm, text width=1.5cm},
        item/.style = {rectangle, fill=blue!10, draw=blue!30, thick, rounded corners=2mm, minimum size=0.7cm, text width=1.5cm, align=center},
        prop/.style = {rectangle, fill=violet!10, draw=violet!30, thick, rounded corners=2mm, minimum size=0.7cm}, % property的缩写
        node distance=0.4cm,
        font=\small,
        >=Stealth]
        \node[item] (item1) {宗旨};
        \node[item, below=of item1] (item2) {根本};
        \node[item, below=of item2] (item3) {基础};
        \node[item, below=of item3] (item4) {保障};
        \node[item, below=of item4] (item5) {依托};
        \node[general, left=1cm of item3] (general) {总体国家安全观};
        \node[prop, right=of item1] (prop1) {\textcolor{violet}{人民安全}};
        \node[prop, right=of item2] (prop2) {\textcolor{violet}{政治安全}};
        \node[prop, right=of item3] (prop3) {\textcolor{violet}{经济安全}};
        \node[prop, right=of item4] (prop4) {军事、科技、文化、社会安全};
        \node[prop, right=of item5] (prop5) {促进国际安全};
        \draw[thick, cyan!50, ->] (general) edge[bend left, looseness=1] (item1.west);
        \draw[thick, cyan!50, ->] (general) edge[bend left, looseness=0.8] (item2.west);
        \draw[thick, cyan!50, ->] (general) to (item3);
        \draw[thick, cyan!50, ->] (general) edge[bend right, looseness=0.8] (item4.west);
        \draw[thick, cyan!50, ->] (general) edge[bend right, looseness=1] (item5.west);
        \foreach\x in {1,2,3,4,5}{
            \draw[blue!30] (item\x) to (prop\x);
        }
    \end{tikzpicture}
    \caption{总体国家安全观}\label{fig:sec13-总体国家安全观}
\end{figure}

\begin{figure}[H]
    \centering
    \begin{tikzpicture}[
        >=Stealth,
        item/.style = {circle, fill=cyan!20, draw=cyan!50, thick},
        node distance = 3cm,
        prop/.style = {circle, fill=violet!10, draw=violet!30, thick},
        font=\small]

        \node[item] (develop) {发展};
        \node[prop, right=of develop] (save) {安全};
        \draw[->, cyan!50, thick] (develop) edge[bend left, looseness=0.7] node[above] {\color{darkcyan}基础和目的} (save);
        \draw[->, violet!30, thick] (save) edge[bend left, looseness=0.7] node[below] {\color{violet}条件和保障} (develop);
        \node[left, outer sep=0.5cm] at (develop) {\color{cyan!50}动力/根本支撑};
        \node[right, outer sep=0.5cm] at (save) {\color{violet!30}保障/坚强柱石};
    \end{tikzpicture}
    \caption{发展和安全的关系:一体两翼、驱动双轮}\label{fig:sec13-发展和安全的关系}
\end{figure}

\begin{itemize}
  \item 在所有的安全中,居于首要位置的是维护\textcolor{red}{政治安全}。它包括3点:维护政权安全、制度安全、意识形态安全。
  \item 政治安全(根本保证)、人民安全(中心地位)、国家利益至上(要求与原则)有机统一。
  \item 国家安全的高风险期:由大向强、将强未强之际。
  \item 全民国家安全教育日:4 月 15 日。
\end{itemize}


\newpage
\section{建设巩固国防和强大人民军队}\label{sec:14}
\begin{summary}
\begin{enumerate}
  \item 新时代人民军队使命任务(四个战略支撑)
  \item 强军目标的科学内涵(听党指挥、能打胜仗、作风优良)
  \item 国防军队新三步走
  \item 党对人民军队的领导的相关制度
\end{enumerate}
\end{summary}

\begin{figure}[H]
    \centering
    \begin{tikzpicture}[
        general/.style = {circle, fill=cyan!20, draw=cyan!50, thick, minimum size=1cm, text width=1cm},
        item/.style = {rectangle, fill=blue!10, draw=blue!30, thick, rounded corners=2mm, minimum size=0.7cm, text width=7cm, align=center},
        prop/.style = {rectangle, fill=violet!10, draw=violet!30, thick, rounded corners=2mm, minimum size=0.7cm}, % property的缩写
        node distance=0.4cm,
        font=\small,
        >=Stealth]
        \node[item] (item1) {为巩固中国共产党领导和我国社会主义制度提供战略支撑};
        \node[item, below=of item1] (item2) {为捍卫国家主权、统一和领土完整提供战略支撑};
        \node[item, below=of item2] (item3) {为维护我国海外利益提供战略支撑};
        \node[item, below=of item3] (item4) {为促进世界和平与发展提供战略支撑};
        \node[general, left=1cm of item3, yshift=0.6cm] (general) {四个战略支撑};
        \draw[thick, cyan!50, ->] (general) edge[bend left] (item1.west);
        \draw[thick, cyan!50, ->] (general) edge[bend left] (item2.west);
        \draw[thick, cyan!50, ->] (general) edge[bend right] (item3.west);
        \draw[thick, cyan!50, ->] (general) edge[bend right] (item4.west);
    \end{tikzpicture}
    \caption{新时代人民军队使命任务(四个战略支撑)}\label{fig:sec14-新时代人民军队使命任务(四个战略支撑)}
\end{figure}

\begin{figure}[H]
    \centering
    \begin{tikzpicture}[
        general/.style = {circle, fill=cyan!20, draw=cyan!50, thick, minimum size=1cm, text width=0.7cm},
        item/.style = {rectangle, fill=blue!10, draw=blue!30, thick, rounded corners=2mm, minimum size=0.7cm, text width=3cm, align=center},
        prop/.style = {rectangle, fill=violet!10, draw=violet!30, thick, rounded corners=2mm, minimum size=0.7cm}, % property的缩写
        node distance=0.4cm,
        font=\small,
        >=Stealth]
        \node[item] (item1) {听党指挥(灵魂)};
        \node[item, below=of item1] (item2) {能打胜仗(核心)};
        \node[item, below=of item2] (item3) {作风优良(保证)};
        \node[general, left=1cm of item2] (general) {强军目标};
        \draw[thick, cyan!50, ->] (general) edge[bend left] (item1.west);
        \draw[thick, cyan!50, ->] (general) to (item2.west);
        \draw[thick, cyan!50, ->] (general) edge[bend right] (item3.west);
        \node[prop, right=of item1] (prop1-1) {政治方向};
        \node[prop, right=of item2] (prop2-1) {根本职能};
        \node[prop, right=0cm of prop2-1] (prop2-2) {根本指向};
        \node[prop, right=of item3] (prop3-1) {鲜明特色};
        \node[prop, right=0cm of prop3-1] (prop3-2) {政治优势};
        \node[prop, right=0cm of prop3-2] (prop3-3) {性质、宗旨、本色};
        \foreach\x in {1,2,3}{
            \draw[blue!30] (item\x) to (prop\x-1);
        }
    \end{tikzpicture}
    \caption{强军目标的科学内涵(注意政治方向和政治优势不一样)}\label{fig:sec14-强军目标的科学内涵}
\end{figure}

\begin{figure}[H]
    \centering
    \begin{tikzpicture}[
        seq/.style = {circle, fill=cyan!20, draw=cyan!50, thick},
        item/.style = {rectangle, fill=blue!10, draw=blue!30, thick, rounded corners=2mm, minimum size=0.7cm, text width=5cm, align=center},
        prop/.style = {rectangle, fill=violet!10, draw=violet!30, thick, rounded corners=2mm, minimum size=0.7cm},
        node distance=0.4cm,
        font=\small,
        >=Stealth]

        % 铺垫形状
        \node [seq] (seq1) {\color{darkcyan}1};
        \foreach \x in {2,3}{
            \pgfmathtruncatemacro{\prev}{\x-1} % 使用一个宏来预处理上一个节点的编号
            \node [seq, below=of seq\prev] (seq\x)  {\color{darkcyan}\x};
            \draw [cyan!50, thick] (seq\x) to (seq\prev);
        }
        \node[item, right=of seq1] (item1){2027:实现建军一百年奋斗目标};
        \node[item, right=of seq2] (item2){2035:基本实现国防和军队现代化};
        \node[item, right=of seq3] (item3){本世纪中叶:全面建成世界一流军队};
        \node[prop, right=of item2] (prop2) {理论、组织、人员、装备现代化};
        \node[prop, right=of item3] (prop3) {强国地位、国家安全、国际影响力};
        \foreach \x in {1,2,3}{
            \draw[blue!30] (seq\x) to (item\x);
        }
        \draw[violet!30] (item2) -- (prop2);
        \draw[violet!30] (item3) -- (prop3);
    \end{tikzpicture}
    \caption{国防军队的「新三步走」}\label{fig:sec14-军队新三步走}
\end{figure}

\begin{figure}[H]
    \centering
    \begin{tikzpicture}[
        seq/.style = {circle, fill=cyan!20, draw=cyan!50, thick},
        item/.style = {rectangle, fill=blue!10, draw=blue!30, thick, rounded corners=2mm, minimum size=0.6cm, text width=3cm, align=center},
        prop/.style = {rectangle, fill=violet!10, draw=violet!30, thick, rounded corners=2mm, minimum size=0.6cm, text width=7cm, align=left}, % property的缩写
        node distance=0.4cm,
        font=\small,
        >=Stealth]

        % 铺垫形状
        \node [seq] (seq1) {\color{darkcyan}是};
        \foreach \x in {2,3,4,5}{
            \pgfmathtruncatemacro{\prev}{\x-1} % 使用一个宏来预处理上一个节点的编号
            \node [seq, below=of seq\prev] (seq\x)  {\color{darkcyan}是};
            \draw [cyan!50, thick] (seq\x) to (seq\prev);
        }
        \node[item, left=of seq1] (item1){军委主席负责制};
        \node[item, left=of seq2] (item2){政治建军};
        \node[item, left=of seq3] (item3){改革强军};
        \node[item, left=of seq4] (item4){战斗力};
        \node[item, left=of seq5] (item5){军政军民团结};
        \node[prop, right=of seq1](prop1){党对人民军队绝对领导的根本制度和根本实现形式};
        \node[prop, right=of seq2](prop2){人民军队的立军之本};
        \node[prop, right=of seq3](prop3){决定人民军队发展壮大、制胜未来的关键一招};
        \node[prop, right=of seq4](prop4){唯一、根本的标准};
        \node[prop, right=of seq5](prop5){实现富国和强军相统一的重要政治保障};
        \foreach \x in {1,2,3,4,5}{
            \draw[blue!30] (seq\x) to (item\x);
            \draw[violet!30] (seq\x) to (prop\x);
        }
    \end{tikzpicture}
    \caption{人民军队的其他概念}\label{fig:sec14-其他概念}
\end{figure}

\newpage
\section{坚持「一国两制」和推进祖国完全统一}\label{sec:15}
\begin{summary}
\begin{enumerate}
    \item 「一国两制」的内容(三个分句)
    \item 「一国两制」的意义
    \item 全面管治权与高度自治权的关系
\end{enumerate}
\end{summary}

\begin{figure}[H]
    \centering
    \begin{tikzpicture}[
        general/.style = {circle, fill=cyan!20, draw=cyan!50, thick, minimum size=1cm, text width=0.7cm},
        item/.style = {rectangle, fill=blue!10, draw=blue!30, thick, rounded corners=2mm, minimum size=0.7cm, text width=2cm, align=center},
        prop/.style = {rectangle, fill=violet!10, draw=violet!30, thick, rounded corners=2mm, minimum size=0.7cm, text width=8cm, align=center}, % property的缩写
        node distance=0.4cm,
        font=\small,
        >=Stealth]
        \node[item] (item1) {内容};
        \node[item, below=of item1] (item2) {最高原则};
        \node[item, below=of item2] (item3) {根本宗旨};
        \node[general, left=1cm of item2] (general) {一国两制};

        \node[prop, right=of item1] (prop1) {\color{violet}(1)“一国”是实行“两制”的前提和基础,(2)“两制”从属和派生于“一国”,(3)并统一于“一国”之内};
        \node[prop, right=of item2] (prop2) {维护国家主权、安全、发展利益};
        \node[prop, right=of item3] (prop3) {维护国家主权、安全、发展利益,保持香港澳门长期繁荣稳定};

        \draw[thick, cyan!50, ->] (general) edge[bend left] (item1.west);
        \draw[thick, cyan!50, ->] (general) to (item2.west);
        \draw[thick, cyan!50, ->] (general) edge[bend right] (item3.west);
        \foreach\x in {1,2,3}{
            \draw[blue!30] (item\x) to (prop\x);
        }
    \end{tikzpicture}
    \caption{一国两制}\label{fig:sec15-一国两制}
\end{figure}

\begin{figure}[H]
    \centering
    \begin{tikzpicture}[
        general/.style = {circle, fill=cyan!20, draw=cyan!50, thick, minimum size=1cm, text width=1.3cm, align=center},
        item/.style = {rectangle, fill=blue!10, draw=blue!30, thick, rounded corners=2mm, minimum size=0.7cm, text width=2cm, align=center},
        prop/.style = {rectangle, fill=violet!10, draw=violet!30, thick, rounded corners=2mm, minimum size=0.7cm, text width=8cm, align=center}, % property的缩写
        node distance=1cm,
        font=\small,
        >=Stealth]
        \node[general] (A) {全面管治权};
        \node[general, below=of A] (B) {高度自治权};
        \draw[->, thick, cyan!50] (A) -- (B);
        \node[right, xshift=1cm] at (A) {\color{cyan} 前提/基础};
        \node[right, xshift=1cm] at (B) {\color{cyan} 体现};
    \end{tikzpicture}
    \caption{全面管治权与高度自治权:源流关系}\label{fig:sec15-全面管治权与高度自治权}
\end{figure}

\begin{itemize}
  \item 「一国两制」最早针对台湾问题,首先运用于解决香港和澳门问题。
  \item 粤港澳大湾区建设实行「一个国家、两种制度、三个关税区、三种货币」。
  \item 香港从「由乱到治」走向「由治及兴」。
\end{itemize}


\newpage
\section{中国特色大国外交和推动构建人类命运共同体}\label{sec:16}
\begin{summary}
\begin{enumerate}
  \item 中国外交(对外工作)的出发点和落脚点、总目标、基本原则
  \item 人类命运共同体、全人类共同价值
  \item 新型国际关系
\end{enumerate}
\end{summary}

\begin{figure}[H]
    \centering
    \begin{tikzpicture}[
        general/.style = {circle, fill=cyan!20, draw=cyan!50, thick, minimum size=1cm, text width=0.7cm},
        item/.style = {rectangle, fill=blue!10, draw=blue!30, thick, rounded corners=2mm, minimum size=0.7cm, text width=2cm, align=center},
        prop/.style = {rectangle, fill=violet!10, draw=violet!30, thick, rounded corners=2mm, minimum size=0.7cm, text width=5cm, align=center}, % property的缩写
        node distance=0.4cm,
        font=\small,
        >=Stealth]
        \node[item] (item1) {出发点落脚点};
        \node[item, below=of item1] (item2) {总目标};
        \node[item, below=of item2] (item3) {基本原则};
        \node[general, left=1cm of item2] (general) {中国外交};

        \node[prop, right=of item1] (prop1) {维护国家主权、安全、发展利益};
        \node[prop, right=of item2] (prop2) {推动构建人类命运共同体};
        \node[prop, right=of item3] (prop3) {走和平发展道路};

        \draw[thick, cyan!50, ->] (general) edge[bend left] (item1.west);
        \draw[thick, cyan!50, ->] (general) to (item2.west);
        \draw[thick, cyan!50, ->] (general) edge[bend right] (item3.west);
        \foreach\x in {1,2,3}{
            \draw[blue!30] (item\x) to (prop\x);
        }
    \end{tikzpicture}
    \caption{中国外交工作}\label{fig:sec16-外交工作}
\end{figure}


\begin{figure}[H]
    \centering
    \begin{tikzpicture}[
        seq/.style = {circle, fill=cyan!20, draw=cyan!50, thick},
        item/.style = {rectangle, fill=blue!10, draw=blue!30, thick, rounded corners=2mm, minimum size=0.7cm, text width=5cm, align=center},
        prop/.style = {rectangle, fill=violet!10, draw=violet!30, thick, rounded corners=2mm, minimum size=0.7cm},
        node distance=0.4cm,
        font=\small,
        >=Stealth]

        % 铺垫形状
        \node [seq] (seq1) {\color{darkcyan}1};
        \foreach \x in {2,3}{
            \pgfmathtruncatemacro{\prev}{\x-1} % 使用一个宏来预处理上一个节点的编号
            \node [seq, below=of seq\prev] (seq\x)  {\color{darkcyan}\x};
            \draw [cyan!50, thick] (seq\x) to (seq\prev);
        }
        \node[item, right=of seq1] (item1){推动构建新型\textcolor{darkcyan}{国际关系}的原则};
        \node[item, right=of seq2] (item2){\textcolor{darkcyan}{大国关系}格局};
        \node[item, right=of seq3] (item3){构建新型\textcolor{darkcyan}{政党关系}秉持的理念};
        \node[prop, right=of item1] (prop1) {相互尊重、公平正义、合作共赢$^{\text{ p.323}}$};
        \node[prop, right=of item2] (prop2) {和平共处、总体稳定、均衡发展$^{\text{ p.320}}$};
        \node[prop, right=of item3] (prop3) {求同存异、相互尊重、互学互鉴$^{\text{ p.325}}$};
        \foreach \x in {1,2,3}{
            \draw[blue!30] (seq\x) to (item\x);
            \draw[violet!30 ] (item\x) to (prop\x);
        }
    \end{tikzpicture}
    \caption{中国外交的理念、格局(易混词语,注意「相互尊重」出现了两次)}\label{fig:sec16-易混词语}
\end{figure}

\begin{itemize}
  \item 和平、发展、公平、正义、民主、自由,是全人类的共同价值。
  \item 2013年,习近平提出了共同建设「丝绸之路经济带」和「21世纪海上丝绸之路」的倡议。
  \item 周边国家是我国安身立命之所、发展繁荣之基。需要坚持「亲诚惠容」和「与邻为善、以邻为伴」周边外交方针。
\end{itemize}

\newpage
\section{全面从严治党}\label{sec:17}
\begin{summary}
\begin{enumerate}
  \item 全面从严治党——新时代党的建设的鲜明主题
  \item 党的政治建设——根本性建设
  \item 反腐败、勇于自我革命、建设长期执政的马克思主义政党
\end{enumerate}
\end{summary}

\begin{figure}[H]
    \centering
    \begin{tikzpicture}[
        seq/.style = {circle, fill=cyan!20, draw=cyan!50, thick, text width=1cm},
        item/.style = {rectangle, fill=blue!10, draw=blue!30, thick, rounded corners=2mm, minimum size=0.7cm, text width=1cm, align=center},
        prop/.style = {rectangle, fill=violet!10, draw=violet!30, thick, rounded corners=2mm, minimum size=0.7cm,},
        node distance=0.4cm,
        font=\small,
        >=Stealth]

        \node[seq] (general) {全面从严治党};
        \node[above=of general, font=\footnotesize, cyan] {(新时代党的建设的鲜明主题)};
        \node[item, left=of general, yshift=1cm] (item1) {核心};
        \node[item, left=of general, yshift=-1cm] (item2) {基础};
        \node[item, right=of general, yshift=1cm] (item3) {关键};
        \node[item, right=of general, yshift=-1cm] (item4) {要害};
        \node[prop, left=of item1] (prop1) {加强党的领导};
        \node[prop, left=of item2] (prop2) {全面};
        \node[prop, right=of item3] (prop3) {严};
        \node[prop, right=of item4] (prop4) {治};
        \foreach\i in {1,2,3,4}{
            \draw[thick, cyan!50, ->] (general) -- (item\i);
            \draw[blue!30] (item\i) -- (prop\i);
        }
    \end{tikzpicture}
    \caption{全面从严治党}\label{fig:sec17-全面从严治党}
\end{figure}

\begin{figure}[H]
    \centering
    \begin{tikzpicture}[
        seq/.style = {rectangle, fill=cyan!20, draw=cyan!50, thick, rounded corners=2mm, minimum size=0.7cm, text width=2cm},
        item/.style = {rectangle, fill=blue!10, draw=blue!30, thick, rounded corners=2mm, minimum size=0.7cm, text width=2cm, align=center},
        prop/.style = {rectangle, fill=violet!10, draw=violet!30, thick, rounded corners=2mm, minimum size=0.7cm, text width=6cm, align=center},
        node distance=0.4cm,
        font=\small,
        >=Stealth]

        \node[seq] (item1) {党的政治建设};
        \node[item, below=of item1, xshift=1cm] (item2) {根本要求};
        \node[item, below=of item2] (item3) {首要任务};
        \node[prop, right=of item1] (prop1) {党的根本性建设};
        \node[prop, right=of item2] (prop2) {旗帜鲜明讲政治};
        \node[prop, right=of item3] (prop3) {保证全党服从中央,维护党中央权威和集中统一领导};
        \foreach\i in {1,2,3}{
            \draw[blue!30] (item\i) -- (prop\i);
        }
        \draw[thick, cyan!50, xshift=0.2cm] (item1.west) edge[bend right] (item2.west);
        \draw[thick, cyan!50, xshift=0.2cm] (item1.west) edge[bend right] (item3.west);
    \end{tikzpicture}
    \caption{党的政治建设}\label{fig:sec17-党的政治建设}
\end{figure}


\begin{figure}[H]
    \centering
    \begin{tikzpicture}[
        seq/.style = {circle, fill=cyan!20, draw=cyan!50, thick},
        item/.style = {rectangle, fill=blue!10, draw=blue!30, thick, rounded corners=2mm, minimum size=0.6cm, text width=3.5cm, align=center},
        prop/.style = {rectangle, fill=violet!10, draw=violet!30, thick, rounded corners=2mm, minimum size=0.6cm, text width=6cm, align=left}, % property的缩写
        node distance=0.4cm,
        font=\small,
        >=Stealth]

        % 铺垫形状
        \node [seq] (seq1) {\color{darkcyan}是};
        \node [seq, below=1cm of seq1] (seq2) {\color{darkcyan}有};
        \node [seq, below=1cm of seq2] (seq3) {\color{darkcyan}是};
        \draw [cyan!50] (seq1) -- (seq2) -- (seq3);

        \node[item, left=of seq1] (item1){反腐败};
        \node[item, left=of seq2] (item2){跳出历史周期率答案};
        \node[item, left=of seq3] (item3){勇于自我革命};
        
        \node[prop, right=of seq1](prop1){最彻底的自我革命};
        \node[prop, right=of seq2, yshift=0.4cm](prop2-1){(1)让人民监督政府};
        \node[prop, right=of seq2, yshift=-0.4cm](prop2-2){(2)自我革命};
        \node[prop, right=of seq3, yshift=0.4cm](prop3-1){(1)中国共产党区别于其他政党的\textcolor{violet}{显著标志}};
        \node[prop, right=of seq3, yshift=-0.4cm](prop3-2){(2)党最大的优势、\textcolor{violet}{最鲜明的品格}};
        
        \foreach \x in {1,2,3}{
            \draw[blue!30] (seq\x) to (item\x);
        }
        \draw[blue!30] (seq1) to (prop1);
        \draw[blue!30] (seq2) to (prop2-1.west);
        \draw[blue!30] (seq2) to (prop2-2.west);
        \draw[blue!30] (seq3) to (prop3-1.west);
        \draw[blue!30] (seq3) to (prop3-2.west);
    \end{tikzpicture}
    \caption{全面从严治党的其他概念}\label{fig:sec17-其他概念}
\end{figure}

\begin{figure}[H]
    \centering
    \begin{tikzpicture}[
        seq/.style = {circle, fill=cyan!20, draw=cyan!50, thick},
        item/.style = {rectangle, fill=blue!10, draw=blue!30, thick, rounded corners=2mm, minimum size=0.7cm, text width=5cm, align=center},
        prop/.style = {rectangle, fill=violet!10, draw=violet!30, thick, rounded corners=2mm, minimum size=0.7cm},
        node distance=0.4cm,
        font=\small,
        >=Stealth]

        % 铺垫形状
        \node [seq] (seq1) {\color{darkcyan}1};
        \foreach \x in {2,3}{
            \pgfmathtruncatemacro{\prev}{\x-1} % 使用一个宏来预处理上一个节点的编号
            \node [seq, below=of seq\prev] (seq\x)  {\color{darkcyan}\x};
            \draw [cyan!50, thick] (seq\x) to (seq\prev);
        }
        \node[item, right=of seq1] (item1){务必不忘初心、牢记使命};
        \node[item, right=of seq2] (item2){务必谦虚谨慎、艰苦奋斗};
        \node[item, right=of seq3] (item3){务必敢于斗争、善于斗争};
        \foreach \x in {1,2,3}{
            \draw[blue!30] (seq\x) to (item\x);
        }
    \end{tikzpicture}
    \caption{「三个务必」}\label{fig:sec17-「三个务必」}
\end{figure}

\begin{itemize}
  \item 加强作风建设,必须紧紧围绕保持党同人民群众的血肉联系,不断厚植党执政的群众基础。
  \item 不敢腐(前提)、不能腐(关键)、不想腐(根本)一体推进。
  \item 勇于自我革命实现自我净化、自我完善、自我革新、自我提高。
\end{itemize}

\end{document} 