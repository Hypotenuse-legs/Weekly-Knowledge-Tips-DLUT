\documentclass[UTF8]{ctexart}
\usepackage{amsmath}
\usepackage{amssymb}
\usepackage{background}
\usepackage{booktabs}
\usepackage{enumitem}
\usepackage{fancyhdr}
\usepackage{float}
\usepackage{fontspec}
\usepackage{geometry}
\usepackage{tasks}
\usepackage{tcolorbox}
\usepackage{tikz}
\usetikzlibrary{arrows.meta}
\usepackage{xcolor}

\geometry{a5paper, top=0.1cm, left=1cm, right=1cm, bottom=1cm, footskip=0.1cm}
\setCJKmainfont[BoldFont={汉仪文黑-85W},ItalicFont={汉仪文黑-55W}]{汉仪文黑-55W}
\setfontfamily\Issue{Century Schoolbook}
\setfontfamily\Genshin{Genshin Teyvat Lingua Franca}
\newCJKfontfamily\TitleFont{思源宋体 CN Heavy}
\newfontfamily\timesnewroman{Times New Roman}

%————————————————可变部分——————————————————
\settasks{label={\Alph*.\ }, label-format={\color{cyan!50!black}}, item-format={\color{cyan!50!black}}}
\newcommand\col[1]{\textcolor{green!50!black}{#1}}
\newcommand\coll[1]{\textcolor{blue}{#1}}
\newcommand\dotting{\ .\ }
%——————————————————————————————————————————

\pagestyle{fancy}
\fancyhf{}
\cfoot{\sffamily\footnotesize{-\ \thepage\ -}}
%\CTEXsetup[format = {\centering\bfseries\large}, beforeskip = 3pt, afterskip = 3pt]{section}

\colorlet{darkcyan}{cyan!50!black}
\newcommand\Black[1]{\textcolor[gray]{0.3}{#1}}
\newcommand\Brown[1]{\textcolor[HTML]{998A4E}{#1}}
\newcommand\Emph[1]{\colorbox{green!10}{\textcolor{green!30!black}{#1}}}
\newcommand\Notes[1]{\textcolor{yellow!50!black}{\small #1}}
\newcommand\Example[1]{\textcolor{cyan!70!black}{\small #1}}
\colorlet{note}{yellow!70!black}


\newcommand\IssueNumber{42}
\newcommand\Date{2024-11-21}
%\newcommand\Contributer{@金光日}
\newcommand\Subject{计算机组成原理}
\newcommand\Source{历年考研 408 真题}


\begin{document}
\backgroundsetup{contents=\includegraphics{上半示例.png}, center, scale=1, angle=0, opacity=1}
\BgThispage
\begin{center}
%{\scriptsize\Issue \textcolor[HTML]{C8BA83}{\Genshin WEEKLY TIPS}}
\phantom{...}

{\Large\textcolor{brown!40!white}{\makebox[10cm][s]{\Genshin WEEKLY KNOWLEDGE TIPS}}}

\vspace{-2em}

{\Huge\bfseries\TitleFont \Black{知\ 识\ 小\ 料}}


\vspace{-0.1cm}
{\footnotesize \Brown{「电计 2203 班」周常规知识整理共享}}
\end{center}

\vspace{-0.5cm}


\begin{figure}[H]
\hspace{1cm}
\begin{minipage}[t]{0.3\textwidth}
\centering
    \Brown{\Genshin ISSUE}

    \vspace{-0.6cm}
    \Huge \Issue\slshape\bfseries\Black{\IssueNumber}
\end{minipage}
\hfill
\begin{minipage}[t]{0.35\textwidth}
\centering
    \Brown{日期:\Date} \\
%\vspace{-0.1cm}
%    \Brown{贡献者:\Contributer} \\
\vspace{-0.1cm}
    \Brown{学科:\Subject} \\
\vspace{-0.1cm}
    \Brown{来源:\Source}
\end{minipage}
\hspace{0.8cm}
\end{figure}

{\color{cyan!50!black}
\begin{center}
  《I/O设备·计算专题》
\end{center}


\begin{description}
  \item[【简单挑战】] (2011)某计算机处理器主频为 50MHz,采用定时查询方式控制设备 A 的 I/O,查询程序运行一次所用的时钟周期数至少为 500。在设备 A 工作期间,为保证数据不丢失,每秒需对其查询至少 200 次,则 CPU 用于设备 A 的 I/O 的时间占整个 CPU 时间的百分比至少是
  \begin{tasks}(4)
    \task 0.02\%
    \task 0.05\%
    \task 0.20\%
    \task 0.50\%
  \end{tasks}

  \item[【普通挑战】] (2019)某设备以中断方式与 CPU 进行数据交换,CPU 主频为 1GHz,设备接口中的数据缓冲寄存器为 32 位,设备的数据传输速率为 50kB/s。若每次中断开销(包括中断响应和中断处理)为 1000 个时钟周期,则 CPU 用于该设备输入/输出的时间占整个 CPU 时间的百分比最多是
  \begin{tasks}(4)
    \task 1.25\%
    \task 2.5\%
    \task 5\%
    \task 12.5\%
  \end{tasks}

  \item[【困难挑战】] (2018)假定计算机的主频为 500MHz,CPI 为 4。现有设备 A 和 B,其数据传输速率分别为 2MB/s 和 40MB/s,对应 I/O 接口中各有一个 32 位数据缓冲寄存器。若设备 B 采用 DMA 方式,每次 DMA 传送的数据块大小为 1000B,CPU 用于 DMA 预处理和后处理的总时钟周期数为 500,则 CPU 用于设备 B 输入/输出的时间占 CPU 总时间的百分比最多是\underline{\qquad}。

  \item[【无畏挑战】] (2014)若某设备中断请求的响应和处理时间为 100ns,每 400ns 发出一次中断请求,中断响应所允许的最长延迟时间为 50ns,则在该设备持续工作过程中,CPU用于该设备的 I/O 时间占整个CPU时间的百分比至少是
  \begin{tasks}(4)
    \task 12.5\%
    \task 25\%
    \task 37.5\%
    \task 50\%
  \end{tasks}
\end{description}

}

\newpage
\backgroundsetup{contents=\includegraphics{空白示例.png}, center, scale=1, angle=0, opacity=1}
\BgThispage
这四题都与 I/O 设备有关,都是计算题。以下用 $T_c$ 表示时钟周期,用 $f_c$ 表示主频(时钟频率)。

\paragraph{【简单挑战】} 一个朴素的想法,即是计算出\textcolor{cyan!70!black}{「每秒钟 CPU 对设备的查询所占 $T_c$ 数」}与\textcolor{cyan!70!black}{「每秒钟总 $T_c$ 数」}的比值。

已知条件翻译:$f_c = \mathrm{50MHz = 5\times 10^7 Hz}$,每秒查询 200 次,每次占用 $500$ 个 $T_c$。

\begin{itemize}[itemsep=1pt, parsep=1pt]
  \item 每秒的对设备查询所占 $T_c$ 数:$200\times 500 = 100000 = 10^5$ 个 $T_c$。
  \item 每秒总 $T_c$ 数:$5\times 10^7$ 个 $T_c$(也就是时钟频率)。
  \item 比值:$r = \dfrac{10^5}{5\times 10^7}\times 100\% = 0.2\%$。
\end{itemize}
因此答案为 0.20\%,选 C。

\paragraph{【普通挑战】} 本题求解结果与「简单挑战」一样,计算方式也一样。

已知条件翻译:$f_c = \mathrm{1 GHz = 10^9 Hz}$,每秒传输 50kB 数据,每传输 32bit(4B)即触发一次中断,每次中断占 $1000$ 个 $T_c$。

\begin{itemize}[parsep=1pt]
  \item 每秒触发中断的次数:$\mathrm{\dfrac{50kB}{4B} = 12.5k = 1.25\times 10^4}$ 次。
  \item 每秒中断所需要的 $T_c$ 数:$\mathrm{(1.25\times 10^4)\times 1000 = 1.25\times 10^7}$ 个 $T_c$。
  \item 每秒总 $T_c$ 数:$10^9$ 个 $T_c$(也就是时钟频率)。
  \item 比值:$r = \dfrac{1.25\times 10^7}{10^9}\times 100\% = 1.25\%$。
\end{itemize}
因此答案为 1.25\%,选 A。

\paragraph{【困难挑战】} 本题求解结果还与「简单挑战」一样。

这题跟上一题多了一些干扰信息,所以显得困难一些。已知条件翻译:$f_c = \mathrm{500 MHz = 5\times 10^8 Hz}$,一条指令需要 4 个 $T_c$(即 CPI),设备 B 每秒传输 40MB 数据,32 位数据缓冲寄存器,每计满 1000B(一个数据块)触发一次 DMA,一次 DMA 的预、后处理需要 $500$ 个 $T_c$。

\begin{itemize}
  \item 每秒触发 DMA 的次数:$\dfrac{\text{每秒传输的数据量}}{\text{一次 DMA 需要的数据量}} = \mathrm{\dfrac{ 40MB }{ 1000B} = \dfrac{40\times 10^6}{10^3}} = 4\times 10^4$ 次。
  \item 每秒 DMA 所需要的 $T_c$ 数:$(4\times 10^4) \times 500 = 2\times 10^7$ 个 $T_c$。
  \item 每秒总 $T_c$ 数:$5\times 10^8$(也就是时钟频率)。
  \item 比值:$r = \dfrac{2\times 10^7}{5\times 10^8}\times 100\% = 4\%$。
\end{itemize}
因此答案为 4\%。\textcolor{cyan}{(注:DMA 仅预、后处理需要 CPU,其过程不依赖 CPU。)}

\paragraph{【无畏挑战】} 本题要求解的结果还是一样的,但是题目明显不按套路出牌,至少连主频都未给出,着实无从下手。怎么解决?

这题只有三条信息:中断请求的响应和处理需要 100ns,每 400ns 提一次请求,最多允许延迟 50ns 响应。

如果只考虑前两条信息,即每次中断都立刻「有求必应」,则如下图所示:

\begin{figure}[htb]
    \centering
    \begin{tikzpicture}[>=Stealth]
        \draw[->] (-0.5, 0) -- (8.5, 0 ) node[above] {$t$};
        \foreach \i in {0,4,8}{
            \draw (\i, 0) -- (\i, 0.5);
            \draw [->, thick] (\i, -1) node[below, font=\footnotesize] {中断请求} -- (\i, 0);
        }
        \filldraw[fill=cyan!20!white] (0,0) -- (1,0) -- (1, 0.3) -- (0, 0.3) -- cycle;
        \filldraw[fill=cyan!20!white] (4,0) -- (5,0) -- (5, 0.3) -- (4, 0.3) -- cycle;
        \draw [<->, gray] (0, -0.2) -- (4, -0.2) node[below, midway] {400ns};
        \draw [<->, gray] (4, -0.2) -- (8, -0.2) node[below, midway] {400ns};
        \node [above, cyan!50!black] at (0.5, 0.3) {100ns};
        \node [above, cyan!50!black] at (4.5, 0.3) {100ns};
    \end{tikzpicture}
\end{figure}
在这种情况下,CPU用于该设备的 I/O 时间占整个CPU时间的比值就是 $\mathrm{\dfrac{100ns}{400ns} \times 100\% = 25\%}$。

如果考虑「最多延迟 50ns」这一条件,那么我们可以把表示响应的这一段时间向后挪 50ns,像这样:

\begin{figure}[htb]
    \centering
    \begin{tikzpicture}[>=Stealth]
        \draw[->] (-0.5, 0) -- (8.5, 0 ) node[above] {$t$};
        \foreach \i in {0,4,8}{
            \draw (\i, 0) -- (\i, 0.5);
            \draw [->, thick] (\i, -1) node[below, font=\footnotesize] {中断请求} -- (\i, 0);
        }
        \filldraw[fill=cyan!20!white] (0.5, 0) -- (1.5, 0) -- (1.5, 0.3) -- (0.5, 0.3) -- cycle;
        \filldraw[fill=cyan!20!white] (4.5, 0) -- (5.5, 0) -- (5.5, 0.3) -- (4.5, 0.3) -- cycle;
        \draw [<->, gray] (0, -0.2) -- (4, -0.2) node[below, midway] {400ns};
        \draw [<->, gray] (4, -0.2) -- (8, -0.2) node[below, midway] {400ns};
        \node [above, cyan!50!black] at (1, 0.3) {100ns};
        \node [above, cyan!50!black] at (5, 0.3) {100ns};
        \node [brown, font=\footnotesize] at (0.25, 0.15) {50ns};
        \node [brown, font=\footnotesize] at (4.25, 0.15) {50ns};
    \end{tikzpicture}
\end{figure}
可以发现,延迟 50ns 并不影响这一比值,仍旧是 25\%。不难发现,无论延迟多长时间,这一比值都不变;也就是说「最多延迟 50ns」是干扰信息,只需要 100ns 和 400ns 的信息便足够了。

因此答案是 25\%,选 B。

\backgroundsetup{contents=\includegraphics{下半示例.png}, center, scale=1, angle=0, opacity=1}
\BgThispage
\vspace{1em}
{\color{cyan!80!black}
【结论】C, A, 4\%, B

【点评】这是四道计组的考研题,涉及到 I/O 时间的相关计算。对于多数比值问题,都可以用「每秒钟 CPU 对设备的查询所占 $T_c$ 数」与「每秒钟总 $T_c$ 数」做除法计算出答案;对于少数偏门的问题,则需要具体问题具体分析。
}

\end{document}