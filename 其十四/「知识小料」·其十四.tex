\documentclass[UTF8]{ctexart}
\usepackage{amsmath}
\usepackage{amssymb}
\usepackage{booktabs}
\usepackage{background}
\usepackage{caption, subcaption}
\usepackage{enumitem}
\usepackage{float}
\usepackage{fontspec}
\usepackage{geometry}
\usepackage{pifont}
\usepackage{ulem}
\usepackage{xcolor}

\geometry{a5paper, top=0.1cm, left=1cm, right=1cm, bottom=1.1cm}
\setCJKmainfont[BoldFont={汉仪文黑-85W},ItalicFont={方正苏新诗柳楷简体}]{汉仪文黑-55W}
\setfontfamily\Issue{Century Schoolbook}
\newCJKfontfamily\TitleFont{思源宋体 CN Heavy}
\newfontfamily\timesnewroman{Times New Roman}
\captionsetup{font=small, labelfont=bf}
\reversemarginpar

\CTEXsetup[format = {\centering\bfseries\large}, beforeskip = 3pt, afterskip = 3pt]{section}

\colorlet{darkcyan}{cyan!50!black}
\newcommand\Black[1]{\textcolor[gray]{0.3}{#1}}
\newcommand\Brown[1]{\textcolor[HTML]{998A4E}{#1}}
\newcommand\Concept[1]{\colorbox{cyan!10!white}{\textcolor{cyan!40!black}{#1}}}
\newcommand\Emph[1]{\colorbox{violet!10}{\textcolor{violet}{\bfseries #1}}}
\newcommand\Notes[1]{\textcolor{yellow!50!black}{\small #1}}
\newcommand\Example[1]{\textcolor{cyan!70!black}{\small #1}}
\newcommand\pos[1]{\hspace{0pt} \marginpar{\footnotesize\textcolor{yellow!50!black}{\hfill #1}}}

\newcommand\IssueNumber{14}
\newcommand\Date{2023-12-20}
%\newcommand\Contributer{@金光日}
\newcommand\Subject{马克思主义基本原理}


\begin{document}
\backgroundsetup{contents=\includegraphics{上半示例.png}, center, scale=1, angle=0, opacity=1}
\BgThispage
\begin{center}
{\scriptsize\Issue \textcolor[HTML]{C8BA83}{WEEKLY TIPS}}

{\Huge\bfseries\TitleFont \Black{知\ 识\ 小\ 料}}

\vspace{-0.1cm}
{\footnotesize \Brown{「电计 2203 班」周常规知识整理共享}}
\end{center}

\vspace{-0.5cm}

\begin{figure}[H]
\hspace{1cm}
\begin{minipage}[t]{0.3\textwidth}
\centering
    \Brown{ISSUE.}

    \vspace{-0.6cm}
    \Huge \Issue\slshape\bfseries\Black{\IssueNumber}
\end{minipage}
\hfill
\begin{minipage}[t]{0.35\textwidth}
\centering
    \Brown{日期:\Date} \\
%\vspace{-0.1cm}
%    \Brown{贡献者:\Contributer} \\
\vspace{-0.1cm}
    \Brown{学科:\Subject} \\
\end{minipage}
\hspace{0.8cm}
\end{figure}

{\color{cyan!50!black}
本文档给出马原解答题涉及到的知识点的位置,以老师最后一节课划出的知识范围为依据收录知识点。

这些知识点需要背诵,通常需要背诵的范围为所录知识点起的一段,或若干段的首句;为了便于背诵,详细知识点已经在本文档中列出。
}

\setcounter{section}{-1}
\section{导\quad 论}
\begin{description}[itemsep=0pt,parsep=0pt]
    \item[\Concept{马克思主义的基本特征}] \pos{p15} 马克思主义的基本特征,用一句话来概括就是科学性与革命性的统一。
    \Example{\begin{enumerate}[itemsep=0pt, parsep=0pt, leftmargin=15pt]
            \item 革命性是马克思主义的内在品质。
            \item 革命性是建立在科学性基础上的,是与科学性高度统一的。
        \end{enumerate}}

    \Notes{这是导论部分唯一可能的解答题。}
\end{description}

\section{世界的物质性及发展规律}
\begin{description}[itemsep=0pt,parsep=0pt]
    \item[\Concept{矛盾普遍性与特殊性}] \pos{p47} 矛盾的普遍性和特殊性是辩证统一的关系。
    \Example{\begin{enumerate}[itemsep=0pt, parsep=0pt, leftmargin=15pt]
            \item 矛盾的普遍性即矛盾的共性,是无条件的、绝对的。
            \item 矛盾的特殊性即矛盾的个性,是有条件的、相对的。
        \end{enumerate}}
\end{description}

\section{实践与认识及其发展规律}
\begin{description}[itemsep=0pt]
    \item[\Concept{实践对认识的决定作用}] \pos{p79} 实践对认识的决定作用表现在 \Emph{4} 个方面:\uline{来源}、\uline{动力}、\uline{目的}、\uline{标准}。
    \Example{\begin{enumerate}[itemsep=0pt, parsep=0pt, leftmargin=15pt]
            \item 实践是认识的来源。
            \item 实践是认识发展的动力。
            \item 实践是认识的目的。
            \item 实践是检验认识真理性的唯一标准。
        \end{enumerate}}

    \item[\Concept{感性和理性认识}] \pos{p85} 感性认识和理性认识的性质虽然不同,但二者的关系是辩证统一的。需要掌握 \Emph{3} 个因素。
    \Example{\begin{enumerate}[itemsep=0pt, parsep=0pt, leftmargin=15pt]
            \item 理性认识依赖于感性认识。
            \item 感性认识有待于发展和深化为理性认识。
            \item 感性认识和理性认识相互渗透、相互包含。
        \end{enumerate}}

    \item[\Concept{真理的绝对性和相对性}] \pos{p93} 真理既具有绝对性,又具有相对性。需要掌握 \Emph{3} 个内容。
    \Example{\begin{enumerate}[itemsep=0pt, parsep=0pt, leftmargin=15pt]
            \item 绝对性:是指真理主客观统一的确定性和发展的无限性。任何真理都标志着主观与客观相符合,都包含着不依赖于人和人的意识的客观内容,都同谬误有原则的界限。
            \item 相对性:是指人们在一定条件下对客观事物及其本质和发展规律的正确认识总是有限度的、不完善的。任何真理都只能是主观对客观事物近似正确(即相对正确)的反映。
            \item 辩证统一:真理的绝对性与相对性相互依存、相互包含。人类的认识是一个不断深化的过程,永远处在由真理的相对性走向绝对性、接近绝对性的转化和发展过程中。
        \end{enumerate}}

    \item[\Concept{实践作为检验真理唯一标准的原因}] \pos{p100} 实践之所以能够作为检验真理的唯一标准,是由真理的本性和实践的特点决定的。需要掌握 \Emph{2} 个内容。
    \Example{\begin{enumerate}[itemsep=0pt, parsep=0pt, leftmargin=15pt]
            \item 从真理的本性看,真理是人们对客观事物及其发展规律的正确反映,它的本性在于主观和客观相符合。
            \item 从实践的特点看,实践具有直接现实性。实践的直接现实性是它的客观实在性的具体表现。
        \end{enumerate}}
\end{description}

\backgroundsetup{contents=\includegraphics{空白示例.png}, center, scale=1, angle=0, opacity=1}
\BgThispage
\section{人类社会及其发展规律}
\begin{description}[itemsep=0pt]
    \item[\Concept{社会存在与社会意识}] \pos{p129} 社会存在与社会意识是辩证统一的。需要掌握以下 \Emph{3} 个内容。
    \Example{\begin{enumerate}[itemsep=0pt, parsep=0pt, leftmargin=15pt]
               \item 社会存在决定社会意识,社会意识是社会存在的反映,并反作用于社会存在。
               \item 社会存在是社会意识内容的客观来源,社会意识是社会物质生活过程及其条件的主观反映。
               \item 社会意识根治于社会存在,是对以实践为基础的不断变化发展的现实世界的反映。
             \end{enumerate}}

    \item[\Concept{生产力与生产关系}] \pos{p135} 生产力与生产关系的相互关系是:需要掌握以下 \Emph{3} 个内容。
    \Example{\begin{enumerate}[itemsep=0pt, parsep=0pt, leftmargin=15pt]
               \item 生产力决定生产关系。
               \item 生产关系对生产力具有能动的反作用。
               \item 生产力与生产关系的相互作用是一个过程,表现为二者的矛盾运动。
             \end{enumerate}}

    \item[\Concept{经济基础与上层建筑}] \pos{p139} 经济基础与上层建筑是辩证统一的。需要掌握以下 \Emph{3} 个内容。
    \Example{\begin{enumerate}[itemsep=0pt, parsep=0pt, leftmargin=15pt]
               \item 经济基础决定上层建筑。
               \item 上层建筑对经济基础具有反作用。
               \item 当上层建筑为适合生产力发展要求的经济基础服务时,成为推动社会发展的进步力量;反正,当它为落后的经济基础服务时,成为阻碍社会发展的消极力量。
             \end{enumerate}}

    \item[\Concept{人民群众创造历史}] \pos{p172} 人民群众是社会历史的主体,是历史的创造者。需要掌握以下 \Emph{3} 个内容。
    \Example{\begin{enumerate}[itemsep=0pt, parsep=0pt, leftmargin=15pt]
                \item 人民群众是社会物质财富的创造者。
                \item 人民群众是社会精神财富的创造者。
                \item 人民群众是社会变革的决定力量。
            \end{enumerate}}

    \item[\Concept{群众观点和群众路线}] \pos{p174} 坚持马克思主义群众观点,贯彻党的群众路线。需要掌握以下 \Emph{2} 个内容。
    \Example{\begin{enumerate}[itemsep=0pt, parsep=0pt, leftmargin=15pt]
                \item 群众观点:\ding{172} 坚信人民群众自己解放自己的观点;\ding{173} 全心全意为人民服务的观点;\ding{174} 一切向人民群众负责的观点;\ding{175} 虚心向人民群众学习的观点。
                \item 群众路线:一切为了群众,一切依靠群众,从群众中来,到群众中去。这是党的生命线和根本工作路线。
            \end{enumerate}}

    \item[\Concept{评价历史人物的方法}] \pos{p177} 评价历史人物时应该坚持历史分析方法和阶级分析方法。需要掌握以下 \Emph{2} 个内容。
    \Example{\begin{enumerate}[itemsep=0pt, parsep=0pt, leftmargin=15pt]
                \item 历史分析方法要求从特定的历史背景出发,根据当时的历史条件,对历史人物的是非功过进行具体的、全面的考察。
                \item 阶级分析方法要求把历史人物置于一定的阶级关系中,同他们所属的阶级联系起来加以考察和评价。
            \end{enumerate}}
\end{description}

\backgroundsetup{contents=\includegraphics{下半示例.png}, center, scale=1, angle=0, opacity=1}
\BgThispage
\section{资本主义的本质及规律}
\begin{description}[itemsep=0pt]
    \item[\Concept{价值规律}] \pos{p189} 价值规律是商品生产和商品交换的基本规律。需要掌握主要内容、表现形式共 \Emph{2} 个内容。
    \Example{\begin{enumerate}[itemsep=0pt, parsep=0pt, leftmargin=15pt]
                \item 主要内容(客观要求):商品的价值量由生产商品的社会必要劳动时间决定,商品交换以价值量为基础,按照等价交换的原则进行。
                \item 表现形式:商品的价格围绕商品的价值自发波动。
            \end{enumerate}}

    \item[\Concept{资本主义经济危机}] \pos{p223} 资本主义经济危机的本质特征是生产的\uline{相对}过剩。(不是「生产过剩」!)经济危机爆发的根本原因是资本主义的基本矛盾。
\end{description}

以现有信息来看,第 5、 6、 7 章无解答题。

\end{document} 