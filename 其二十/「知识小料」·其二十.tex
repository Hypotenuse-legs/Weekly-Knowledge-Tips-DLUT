\documentclass[UTF8]{ctexart}
\usepackage{amsmath}
\usepackage{amssymb}
\usepackage{booktabs}
\usepackage{background}
\usepackage{caption,subcaption}
\usepackage{enumitem}
\usepackage{float}
\usepackage{fontspec}
\usepackage{fourier}
\usepackage{geometry}
\usepackage{tikz}
\usetikzlibrary{arrows.meta}
\usepackage{xcolor}

\geometry{a5paper, top=0.1cm, left=1cm, right=1cm, bottom=0.3cm, footskip=0.1cm}
\setCJKmainfont[BoldFont={汉仪文黑-85W},ItalicFont={方正苏新诗柳楷简体}]{汉仪文黑-55W}
\setfontfamily\Issue{Century Schoolbook}
\setfontfamily\Genshin{Genshin Teyvat Lingua Franca}
\newCJKfontfamily\TitleFont{思源宋体 CN Heavy}
\newfontfamily\timesnewroman{Times New Roman}
%\reversemarginpar

%\CTEXsetup[format = {\centering\bfseries\large}, beforeskip = 3pt, afterskip = 3pt]{section}
\CTEXsetup[format = {\color{cyan!50!black}\bfseries\large}]{subsection}

\colorlet{darkcyan}{cyan!50!black}
\newcommand\Black[1]{\textcolor[gray]{0.3}{#1}}
\newcommand\Brown[1]{\textcolor[HTML]{998A4E}{#1}}
\newcommand\Emph[1]{\colorbox{green!10}{\textcolor{green!30!black}{#1}}}
\newcommand\Notes[1]{\textcolor{yellow!50!black}{\small #1}}
\newcommand\Example[1]{\textcolor{cyan!70!black}{\small #1}}

\renewcommand\d{\mathrm{d}}

\newcommand\IssueNumber{20}
\newcommand\Date{2024-3-21}
%\newcommand\Contributer{@金光日}
\newcommand\Subject{概率与统计 A}


\begin{document}
\backgroundsetup{contents=\includegraphics{上半示例.png}, center, scale=1, angle=0, opacity=1}
\BgThispage
\begin{center}
%{\scriptsize\Issue \textcolor[HTML]{C8BA83}{\Genshin WEEKLY TIPS}}
\phantom{...}

{\Large\textcolor{brown!40!white}{\makebox[10cm][s]{\Genshin WEEKLY KNOWLEDGE TIPS}}}

\vspace{-2em}

{\Huge\bfseries\TitleFont \Black{知\ 识\ 小\ 料}}


\vspace{-0.1cm}
{\footnotesize \Brown{「电计 2203 班」周常规知识整理共享}}
\end{center}

\vspace{-0.5cm}

\begin{figure}[H]
\hspace{1cm}
\begin{minipage}[t]{0.3\textwidth}
\centering
    \Brown{\Genshin ISSUE}

    \vspace{-0.6cm}
    \Huge \Issue\slshape\bfseries\Black{\IssueNumber}
\end{minipage}
\hfill
\begin{minipage}[t]{0.35\textwidth}
\centering
    \Brown{日期:\Date} \\
%\vspace{-0.1cm}
%    \Brown{贡献者:\Contributer} \\
\vspace{-0.1cm}
    \Brown{学科:\Subject} \\
\end{minipage}
\hspace{0.8cm}
\end{figure}

{\color{cyan!50!black}
已知随机变量 $X,Y$ 的联合密度函数为
\begin{equation*}
    f(x,y) = \begin{cases}
               Ay^2, & 0<y<x<1 \\
               0, & \text{其他} \\
             \end{cases}
\end{equation*}
求:(1)$A$;(2)$f_{X|Y}(x|y)$。
}

【第 1 问】考察的是密度函数的归一性,直接用公式即可。
\begin{align*}
    \textcolor{blue}{1} &=\textcolor{blue}{ \iint_{\mathbb{R}^2} f(x,y)\d x\d y} \\
    &= \int_0^1 \d x\int _0^x Ay^2\d y = \int_0^1 \dfrac{A}{3}x^3 \d x = \dfrac{A}{12}
\end{align*}
因此 $A=12$。

【第 2 问】要求的是二维条件密度 $f_{X|Y} (x|y)$,给定 $y$ 求 $x$ 的密度,公式为
\begin{equation*}
    \textcolor{blue}{f_{X|Y}(x|y) = \dfrac{f(x,y)}{f_Y(y)} = \dfrac{\text{联合密度}}{y\text{的边缘密度}}}
\end{equation*}
联合密度已知,接下来求 $y$ 的边缘密度:
\begin{align*}
    \textcolor{blue}{f_Y(y)} &=\textcolor{blue}{\int_{-\infty}^{+\infty} f(x,y)\d x} \\
    &= \int_y^1 (12y^2)\d x \\
    &= 12y^2(1-y)\quad (0<y<x<1)
\end{align*}
因此得到条件密度(记得写范围):
\begin{equation*}
    f_{X|Y}(x|y) = \dfrac{f(x,y)}{f_Y(y)} = \dfrac{12y^2}{12y^2(1-y)} = \dfrac{1}{1-y} \quad \textcolor{red}{(0<y<x<1)}
\end{equation*}
完整的写法是
\begin{equation*}
    f_{X|Y}(x|y) = \begin{cases}
                     \dfrac{1}{1-y}, & 0<y<x<1 \\
                     0, &\text{其他}
                   \end{cases}
\end{equation*}


\newpage
\backgroundsetup{contents=\includegraphics{下半示例.png}, center, scale=1, angle=0, opacity=1}
\BgThispage

至此这题就做完了。不过在积分方面还有些细节:

在草稿纸上可以画出密度非 0 的区域,是一个直角三角形。

第 1 问我们把它当成 $x$ 型域:如图 (a),外层对 $x$ 积分;内层对 $y$ 积分,下限 0 上限 $x$。

第 2 问求边缘密度时,$\displaystyle\int_{-\infty}^{+\infty} f(x,y)\d x$,积分变量是 $x$,因此定住 $y$ ,对 $x$ 积分。如图 (b),$x$ 的下限 $y$,上限 1。


\begin{figure}[htb]
\begin{minipage}[b]{.5\textwidth}
    \centering
    \begin{tikzpicture}[>=Stealth,scale=3]
        \draw[->] (-0.2,0) -- (1.2,0) node[above] {$X$};
        \draw[->] (0,-0.2) -- (0,1.2) node[right] {$Y$};
        \filldraw[fill=cyan!10] (0,0) -- (1,0) -- (1,1) -- cycle;
        \node[below left] at (0,0) {$O$};
        \node[below] at (1,0) {1};
        \draw (0.05,1) -- (0,1) node[left] {1};
        \draw[dashed,cyan] (0.6,0.7) -- (0.6,-0.1) node[below] {$x$};
        \fill[green!50!black] (0.6,0) circle (0.5pt) node[above] {$y_{\text{下}}=0$};
        \fill[green!50!black] (0.6,0.6) circle (0.5pt) node[above] {$y_{\text{上}}=x$};
    \end{tikzpicture}
    \subcaption{第 1 问:$x$ 型域积分,外层 $x$ 内层 $y$}
\end{minipage}
\begin{minipage}[b]{.5\textwidth}
    \centering
    \begin{tikzpicture}[>=Stealth,scale=3]
        \draw[->] (-0.2,0) -- (1.2,0) node[above] {$X$};
        \draw[->] (0,-0.2) -- (0,1.2) node[right] {$Y$};
        \filldraw[fill=cyan!10] (0,0) -- (1,0) -- (1,1) -- cycle;
        \node[below left] at (0,0) {$O$};
        \node[below] at (1,0) {1};
        \draw (0.05,1) -- (0,1) node[left] {1};
        \draw[dashed,cyan] (1.1,0.6) -- (-0.1,0.6) node[left] {$y$};
        \fill[green!50!black] (0.6,0.6) circle (0.5pt) node[left] {$x_{\text{下}}=y$};
        \fill[green!50!black] (1,0.6) circle (0.5pt) node[right] {$x_{\text{上}}=1$};
    \end{tikzpicture}
    \subcaption{第 2 问:给定 $y$,对 $x$ 积分}
\end{minipage}
\end{figure}

\vspace{1em}
\textcolor{cyan!80!black}{【结论】
\begin{enumerate}
    \item $A=12$
    \item $f_{X|Y}(x|y) = \begin{cases}
                     \dfrac{1}{1-y}, & 0<y<x<1 \\
                     0, &\text{其他}
                   \end{cases}$
\end{enumerate}
}

\textcolor{cyan!80!black}{【点评】本题考察二维连续性随机变量的相关知识点,包括密度函数的性质、条件密度的求解、由联合求边缘的方法等。主要难点可能在于不知道用哪条公式进行计算,或者二重积分计算有错误。本题在试卷中占据 15 分,同学们应掌握。}


\end{document} 