\documentclass[UTF8]{ctexart}
\usepackage{amsmath}
\usepackage{amssymb}
\usepackage{booktabs}
\usepackage{background}
\usepackage{caption,subcaption}
\usepackage{colortbl}
\usepackage{diagbox}
\usepackage{enumitem}
\usepackage{float}
\usepackage{fontspec}
\usepackage{fourier}
\usepackage{geometry}
\usepackage{longtable}
\usepackage{xcolor}

\geometry{a5paper, top=0.1cm, left=1cm, right=1cm, bottom=0.3cm, footskip=0.1cm}
\setCJKmainfont[BoldFont={汉仪文黑-85W},ItalicFont={方正苏新诗柳楷简体}]{汉仪文黑-55W}
\setfontfamily\Issue{Century Schoolbook}
\setfontfamily\Genshin{Genshin Teyvat Lingua Franca}
\newCJKfontfamily\TitleFont{思源宋体 CN Heavy}
\newfontfamily\timesnewroman{Times New Roman}
%\reversemarginpar

%\CTEXsetup[format = {\centering\bfseries\large}, beforeskip = 3pt, afterskip = 3pt]{section}
\CTEXsetup[format = {\color{cyan!50!black}\bfseries\large}]{subsection}

\colorlet{darkcyan}{cyan!50!black}
\newcommand\Black[1]{\textcolor[gray]{0.3}{#1}}
\newcommand\Brown[1]{\textcolor[HTML]{998A4E}{#1}}
\newcommand\Emph[1]{\colorbox{green!10}{\textcolor{green!30!black}{#1}}}
\newcommand\Notes[1]{\textcolor{yellow!50!black}{\small #1}}
\newcommand\Example[1]{\textcolor{cyan!70!black}{\small #1}}

\newcommand\x{\boldsymbol{x}}
\newcommand\A{\boldsymbol{A}}
\newcommand\W{\boldsymbol{W}}

\newcommand\IssueNumber{19}
\newcommand\Date{2024-3-14}
%\newcommand\Contributer{@金光日}
\newcommand\Subject{数字逻辑}


\begin{document}
\backgroundsetup{contents=\includegraphics{上半示例.png}, center, scale=1, angle=0, opacity=1}
\BgThispage
\begin{center}
%{\scriptsize\Issue \textcolor[HTML]{C8BA83}{\Genshin WEEKLY TIPS}}
\phantom{...}

{\Large\textcolor{brown!40!white}{\makebox[10cm][s]{\Genshin WEEKLY KNOWLEDGE TIPS}}}

\vspace{-2em}

{\Huge\bfseries\TitleFont \Black{知\ 识\ 小\ 料}}


\vspace{-0.1cm}
{\footnotesize \Brown{「电计 2203 班」周常规知识整理共享}}
\end{center}

\vspace{-0.5cm}

\begin{figure}[H]
\hspace{1cm}
\begin{minipage}[t]{0.3\textwidth}
\centering
    \Brown{\Genshin ISSUE}

    \vspace{-0.6cm}
    \Huge \Issue\slshape\bfseries\Black{\IssueNumber}
\end{minipage}
\hfill
\begin{minipage}[t]{0.35\textwidth}
\centering
    \Brown{日期:\Date} \\
%\vspace{-0.1cm}
%    \Brown{贡献者:\Contributer} \\
\vspace{-0.1cm}
    \Brown{学科:\Subject} \\
\end{minipage}
\hspace{0.8cm}
\end{figure}

{\color{cyan!50!black}
将下列函数化成最简与或式。
\begin{enumerate}
  \item $F = (A\oplus B)\overline{\bar{A}\bar{B} + AB} + AB$
  \item $F(A,B,C,D) = \sum m(0,1,2,4,8) + \sum d(10,11,12,13,14,15)$
\end{enumerate}
(建议大家先独立做一下第 2 题,指不定卡诺图会出一点点小问题。)
}


这两题分别考察公式法和卡诺图法化简与或式。

第 1 题,这里主要用到的是异或的定义($A\oplus B = A\bar{B}+\bar{A}B$)、摩根定律,同时在最后用到了 $\left\{\begin{aligned} &AB+A\bar{B}=A \\ &A+\bar{A}B = A+B  \end{aligned}\right.$ 这两个定律。
\begin{equation*}
\begin{aligned}
   F= & (A\oplus B)\overline{\bar{A}\bar{B} + AB} + AB && \\
    = & (A\oplus B)\cdot\left( \overline{\bar{A}\bar{B}}\cdot \overline{AB} \right) +AB &\text{\textcolor{yellow!50!black}{(摩根定律)}}&\\
    = & (A\oplus B)\cdot\left( (A+B)\cdot (\bar{A}+\bar{B}) \right)+AB &\text{\textcolor{yellow!50!black}{(摩根定律)}}& \\
    = & (A\bar{B}+\bar{A}B)\cdot (A\bar{B} + \bar{A}B) +AB &\text{\textcolor{yellow!50!black}{(异或展开;$A\bar{A}=B\bar{B}=0$)}}&\\
    = & A\bar{B}+\bar{A}B + AB &\text{\textcolor{yellow!50!black}{($A\cdot A=A$)}}& \\
    = & A + \bar{A}B &\text{\textcolor{yellow!50!black}{($AB+A\bar{B}=A$)}}&\\
    = & A + B &\text{\textcolor{yellow!50!black}{($A+\bar{A}B = A+B$)}}&\\
\end{aligned}
\end{equation*}

第 2 题,卡诺图法,$\sum m$ 表示小项记 1,$\sum d$ 表示随意项记 $\varphi$。画出卡诺图如下表 \ref{tab:kanuo1} 所示。
\begin{table}[htb]
  \centering
  \caption{第 2 题卡诺图}\label{tab:kanuo1}
  \rowcolors{2}{cyan!10}{cyan!15}
  \begin{tabular}{|c|c|c|c|c|}
    \hline
    \rowcolor{cyan!50}
    \diagbox{$CD$}{$AB$} & 00 & 01 & 11 & 10 \\
    \hline
    \cellcolor{cyan!50}{00} & 1 & 1 & $\varphi$ & 1\\
    \hline
    \cellcolor{cyan!50}{01} & 1 &  &  $\varphi$ & \\
    \hline
    \cellcolor{cyan!50}{11} &   &  & $\varphi$ & $\varphi$ \\
    \hline
    \cellcolor{cyan!50}{10} &  1 &  & $\varphi$ & $\varphi$\\
    \hline
  \end{tabular}
\end{table}

\newpage
\backgroundsetup{contents=\includegraphics{下半示例.png}, center, scale=1, angle=0, opacity=1}
\BgThispage
随意项 $\varphi$ 既可以当 1 也可以当 0(只是不能中途更改),我们通过表 \ref{tab:kanuo2} 方式圈出所有的 1,如红字所示。但是!\textbf{注意表 \ref{tab:kanuo2} 第三步},圈 $ABCD=0010$ 位的 1 的时候不要漏圈了,因为标出来的四个位置分别是 $\begin{bmatrix}
         0000 & 1000 \\
         0010 & 1010 \\
       \end{bmatrix}$,横纵向每相邻两个数只有一位不同,因此它们确实是相邻的,应该圈上。
       
\begin{table}[htb]
\caption{卡诺图画圈方式}\label{tab:kanuo2}
\begin{minipage}[t]{0.5\textwidth}
    \rowcolors{2}{cyan!10}{cyan!15}
    \centering
    \subcaption{第一步}
    \begin{tabular}{|c|c|c|c|c|}
    \hline
    \rowcolor{cyan!50}
    \diagbox{$CD$}{$AB$} & 00 & 01 & 11 & 10 \\
    \hline
    \cellcolor{cyan!50}{00} & \textcolor{red}{1} & \textcolor{red}{1} & \textcolor{red}{$\varphi$} & \textcolor{red}{1}\\
    \hline
    \cellcolor{cyan!50}{01} & 1 &  &  $\varphi$ & \\
    \hline
    \cellcolor{cyan!50}{11} &   &  & $\varphi$ & $\varphi$ \\
    \hline
    \cellcolor{cyan!50}{10} &  1 &  & $\varphi$ & $\varphi$\\
    \hline
    \end{tabular}
\end{minipage}
\begin{minipage}[t]{0.5\textwidth}
    \rowcolors{2}{cyan!10}{cyan!15}
    \centering
    \subcaption{第二步}
    \begin{tabular}{|c|c|c|c|c|}
    \hline
    \rowcolor{cyan!50}
    \diagbox{$CD$}{$AB$} & 00 & 01 & 11 & 10 \\
    \hline
    \cellcolor{cyan!50}{00} & \textcolor{red}{1} & 1 & $\varphi$ & 1\\
    \hline
    \cellcolor{cyan!50}{01} & \textcolor{red}{1} &  &  $\varphi$ & \\
    \hline
    \cellcolor{cyan!50}{11} &   &  & $\varphi$ & $\varphi$ \\
    \hline
    \cellcolor{cyan!50}{10} &  1 &  & $\varphi$ & $\varphi$\\
    \hline
    \end{tabular}
\end{minipage}

\centering
\begin{minipage}[t]{0.5\textwidth}
    \rowcolors{2}{cyan!10}{cyan!15}
    \centering
    \subcaption{第三步}
    \begin{tabular}{|c|c|c|c|c|}
    \hline
    \rowcolor{cyan!50}
    \diagbox{$CD$}{$AB$} & 00 & 01 & 11 & 10 \\
    \hline
    \cellcolor{cyan!50}{00} & \textcolor{red}{1} & 1 & $\varphi$ & \textcolor{red}{1}\\
    \hline
    \cellcolor{cyan!50}{01} & 1 &  &  $\varphi$ & \\
    \hline
    \cellcolor{cyan!50}{11} &   &  & $\varphi$ & $\varphi$ \\
    \hline
    \cellcolor{cyan!50}{10} & \textcolor{red}{1} &  & $\varphi$ & \textcolor{red}{$\varphi$}\\
    \hline
    \end{tabular}
\end{minipage}
\end{table}

通过卡诺图画圈的方法得到结果是 $F(A,B,C,D) = \bar{C}\bar{D} + \bar{A}\bar{B}\bar{C} + \bar{B}\bar{D}$。

\vspace{1em}
\textcolor{cyan!80!black}{【结论】
\begin{enumerate}
    \item $F=A+B$
    \item $F(A,B,C,D) = \bar{C}\bar{D} + \bar{A}\bar{B}\bar{C} + \bar{B}\bar{D}$
\end{enumerate}
}

\textcolor{cyan!80!black}{【点评】本题考察数字逻辑的函数关系化简,难度不大,但是容易失误。化简公式、圈卡诺图的时候务必仔细一些。}

\end{document}