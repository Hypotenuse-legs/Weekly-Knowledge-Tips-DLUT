\documentclass[UTF8]{ctexart}
\usepackage{amsmath}
\usepackage{amssymb}
\usepackage{background}
\usepackage{booktabs}
\usepackage{caption,subcaption}
\usepackage{enumitem}
\usepackage{fancyhdr}
\usepackage{float}
\usepackage{fontspec}
\usepackage{geometry}
\usepackage{makecell}
%\usepackage{mathptmx} %mathptmx和times结合使得公式使用times new roman字体
\usepackage{pifont}
\usepackage{tcolorbox}
\tcbuselibrary{breakable, raster}
\usepackage{tikz}
\usetikzlibrary{arrows.meta}
%\usepackage{times}
\usepackage[table]{xcolor}

\geometry{a5paper, top=0.1cm, left=1cm, right=1cm, bottom=1cm, footskip=0.6cm, marginparsep=0.1cm}

\setCJKmainfont[BoldFont={汉仪文黑-85W},ItalicFont={方正苏新诗柳楷简体}]{汉仪文黑-55W}
\setfontfamily\Issue{Century Schoolbook}
\setfontfamily\Genshin{Genshin Teyvat Lingua Franca}
\newfontfamily\timesnewroman{Times New Roman}
%\setmainfont{Times New Roman}
\newCJKfontfamily\TitleFont{思源宋体 CN Heavy}
\captionsetup{font=small, labelfont=bf}
\setlist[itemize]{itemsep=0pt, parsep=0pt}
%\reversemarginpar

%\CTEXsetup[format = {\centering\bfseries\large}, beforeskip = 3pt, afterskip = 3pt]{section}
\CTEXsetup[format = {\color{cyan!50!black}\bfseries\large}]{subsection}
\pagestyle{fancy}
\fancyhf{}
\cfoot{\ttfamily\footnotesize{-\ \thepage\ -}}

\newtcolorbox{mybox}[1]{colback=c#1-light, colframe=c#1-emph, boxrule=0.5pt, breakable, fontupper=\small}

%主题颜色设定区
\definecolor{pyro}{RGB}{245,86,60}
\colorlet{c1-light}{pyro!15}
\colorlet{c1-med}{pyro!30}
\colorlet{c1-heavy}{pyro!60}
\colorlet{c1-emph}{pyro!50!black}
\definecolor{hydro}{RGB}{2,182,255}
\colorlet{c2-light}{hydro!15}
\colorlet{c2-med}{hydro!30}
\colorlet{c2-heavy}{hydro!50}
\colorlet{c2-emph}{hydro!50!black}
\definecolor{anemo}{RGB}{86,226,193}
\colorlet{c3-light}{anemo!20}
\colorlet{c3-med}{anemo!40}
\colorlet{c3-heavy}{anemo!60}
\colorlet{c3-emph}{anemo!40!black}
\definecolor{electro}{RGB}{173,101,216}
\colorlet{c4-light}{electro!10}
\colorlet{c4-med}{electro!20}
\colorlet{c4-heavy}{electro!30}
\colorlet{c4-emph}{electro!70!black}
\colorlet{dendro}{green!50!yellow!50}
\colorlet{c5-light}{dendro!15}
\colorlet{c5-med}{dendro!50}
\colorlet{c5-heavy}{dendro!80}
\colorlet{c5-emph}{dendro!50!black}
\definecolor{cryo}{RGB}{155,232,238}
\colorlet{c6-light}{cryo!20}
\colorlet{c6-med}{cryo!40}
\colorlet{c6-heavy}{cryo!60}
\colorlet{c6-emph}{cyan!80!black}
\definecolor{geo}{RGB}{237,199,82}
\colorlet{c7-light}{geo!20}
\colorlet{c7-med}{geo!40}
\colorlet{c7-heavy}{geo!60}
\colorlet{c7-emph}{brown!80!black}
\definecolor{eighth}{HTML}{E57BEF}
\colorlet{c8-light}{eighth!10}
\colorlet{c8-med}{eighth!30}
\colorlet{c8-heavy}{eighth!60}
\colorlet{c8-emph}{eighth!70!black}

%控制编号设定区
\newcounter{seq}[section]
\setcounter{seq}{0}
\newlist{myenumerate}{enumerate}{1}
\setlist[myenumerate]{
    label = {\thesection-\arabic{seq}.},
    before = \setcounter{seq}{\value{enumi}},
    after = \refstepcounter{seq},
}
\setlist{
    label = {\thesection-\arabic{enumi}.},
    format = {\bfseries},
    resume,
    itemsep = 0pt,
    parsep = 0pt,
}

%\restartlist{section}




\colorlet{darkcyan}{cyan!50!black}
\newcommand\Black[1]{\textcolor[gray]{0.3}{#1}}
\newcommand\Brown[1]{\textcolor[HTML]{998A4E}{#1}}
\newcommand\Emph[2]{\colorbox{c#1-light}{\textcolor{c#1-emph}{\textbf{#2}}}}
\newcommand\Subknowledge[2]{\textcolor{c#1-heavy}{#2}}
\newcommand\Concept[1]{\textcolor{cyan!70!black}{#1}}
\newcommand\Notes[1]{\textcolor{yellow!50!black}{\small #1}}
\newcommand\Example[1]{\textcolor{cyan!70!black}{\small #1}}
\newcommand\means[1]{\textcolor{cyan!70!black}{#1}}

\newcommand\Ohm{\text{\timesnewroman Ω}}

\newcommand\IssueNumber{28}
\newcommand\Date{2024-5-20}
%\newcommand\Contributer{@金光日}
\newcommand\Subject{毛概}
%\newcommand\Source{2023 考研 408 第 9 题}


\begin{document}
\backgroundsetup{contents=\includegraphics{上半示例.png}, center, scale=1, angle=0, opacity=1}
\BgThispage
\begin{center}
%{\scriptsize\Issue \textcolor[HTML]{C8BA83}{\Genshin WEEKLY TIPS}}
\phantom{...}

{\Large\textcolor{brown!40!white}{\makebox[10cm][s]{\Genshin WEEKLY KNOWLEDGE TIPS}}}

\vspace{-2em}

{\Huge\bfseries\TitleFont \Black{知\ 识\ 小\ 料}}


\vspace{-0.1cm}
{\footnotesize \Brown{「电计 2203 班」周常规知识整理共享}}
\end{center}

\vspace{-0.5cm}


\begin{figure}[H]
\hspace{1cm}
\begin{minipage}[t]{0.3\textwidth}
\centering
    \Brown{\Genshin ISSUE}

    \vspace{-0.6cm}
    \Huge \Issue\slshape\bfseries\Black{\IssueNumber}
\end{minipage}
\hfill
\begin{minipage}[t]{0.35\textwidth}
\small
\centering
    \Brown{日期:\Date} \\
%\vspace{-0.1cm}
%    \Brown{贡献者:\Contributer} \\
\vspace{-0.1cm}
    \Brown{学科:\Subject} \\
%\vspace{-0.1cm}
%    \Brown{来源:\Source}
\end{minipage}
\hspace{0.8cm}
\end{figure}

{\color{cyan!50!black} 本文档用于重写《毛泽东思想和中国特色社会主义理论体系概论》(简称《毛概》)知识要点,作为更加明晰的背诵版本。为了便于背诵,可能采取诸多手段,比如删减次要内容、换用颜色主题等。本文档共 19 页,总体积为 296KiB。

下表 \ref{tab:清单} 为《毛概》的内容清单:

\begin{table}[htb]
  \centering\small
  \begin{tabular}{ccccc}
  \rowcolor{brown}\color{white}
   \# & \color{white} 关键字 & \color{white} 配色方案 & \color{white} 所占页数 & \color{white} 起始页码 \\
  \rowcolor{brown!30}
  0 & 马克思主义中国化时代化 & — & 1 & \pageref{sec:0} \\
  \rowcolor{c1-med}
  1 & 毛泽东思想的历史 & 火(Pyro) & 2 & \pageref{sec:1} \\
  \rowcolor{c2-med}
  2 & 「新民主主义革命」相关理论 &  水(Hydro) & 2 & \pageref{sec:2} \\
  \rowcolor{c3-med}
  3 & 「社会主义改造」相关理论 &  风(Anemo) & 2 & \pageref{sec:3} \\
  \rowcolor{c4-med}
  4 & 社会主义建设道路初步探索的理论成果 &  雷(Electro) & 2 & \pageref{sec:4} \\
  \rowcolor{c5-med}
  5 & 中国特色社会主义理论体系的形成、发展 &  草(Dendro) & 2 & \pageref{sec:5} \\
  \rowcolor{c6-med}
  6 & 邓小平理论 &  冰(Cryo) & 3 & \pageref{sec:6} \\
  \rowcolor{c7-med}
  7 & 「三个代表」思想 &  岩(Geo) & 2 & \pageref{sec:7} \\
  \rowcolor{c8-med}
  8 & 科学发展观 &  第八(Eighth) & 2 & \pageref{sec:8} \\
  \end{tabular}
  \caption{内容清单}\label{tab:清单}
\end{table}

对于需考研者,本期「知识小料」建议留存并重命名,以备复习。
}

\newpage
\backgroundsetup{contents=\includegraphics{空白示例.png}, center, scale=1, angle=0, opacity=1}
\BgThispage
\setlist{
    label = {\thesection-\arabic{enumi}.},
    resume,
    itemsep = 0pt,
    parsep = 0pt,
}
\setcounter{section}{-1}
\section{马克思主义中国化时代化}\label{sec:0}
\begin{enumerate}[start=1]
  \item 中国的先进分子在反复比较中坚定地选择了马克思主义,原因:
  \begin{enumerate}[label={\roman{enumii})}, start=1]
    \item 学习西方的痛苦经历加之十月革命成功的启示;
    \item 马克思主义与中国传统文化有着相通之处;
    \item 马克思主义揭示了人类社会发展的基本规律。
  \end{enumerate}
  \item 1938 年党的六届六中全会,毛泽东在《论新阶段》的政治报告中,正式提出了「马克思主义中国化」的重大命题。
  \item 解决中国革命问题的根本之「道」就是马克思主义中国化,即将马克思主义与中国革命具体实际相结合。
\end{enumerate}

\begin{tcolorbox}[colback=brown!10, colframe=brown!20!black, boxrule=0.5pt, breakable, fontupper=\small]
\textbf{关于马克思主义中国化时代化的相关信息…}
\begin{enumerate}
  \item 马克思主义中国化时代化的必要性:(i)是马克思主义理论本身发展的内在要求;(ii)是解决中国实际问题的客观需要。
  \item 马克思主义中国化时代化,就是立足中国国情和时代特点,坚持把马克思主义基本原理与中国具体实际、与中华优秀传统文化相结合。
  \item 马克思主义中国化时代化的科学内涵,具体包括三层意思。
  \item 马克思主义中国化时代化的历史进程主要包括四个阶段:
  \begin{enumerate}[label={\roman{enumii})}, start=1]
    \item 新民主主义革命时期——毛泽东思想;
    \item 社会主义革命和建设时期——毛泽东思想(进一步发展);
    \item 改革开放和社会主义现代化建设新时期——中国特色社会主义理论体系;
    \item 党的十八大以来——习思想。
  \end{enumerate}
  \item 在马克思诞辰200周年大会,习近平用三次「伟大飞跃」精准概括了中国共产党 97 年的奋斗历程。
  \item 马克思主义中国化时代化理论成果\textcolor{gray}{(包含毛泽东思想、邓小平理论、「三个代表」重要思想、科学发展观、习近平新时代中国特色社会主义思想)}及其关系:
  \begin{enumerate}[label={\roman{enumii})}, start=1]
    \item 理论成果是一脉相承又与时俱进的关系;
    \item 中国特色社会主义理论体系在新的历史条件下进一步丰富和发展了毛泽东思想。
  \end{enumerate}
\end{enumerate}

\end{tcolorbox}

\newpage
\backgroundsetup{contents={}, center, scale=1, angle=0, opacity=1}
\BgThispage
\pagecolor{c1-med}
\setlist{
    label = {\thesection-\arabic{enumi}.},
    format = {\color{c1-emph}},
    resume,
    itemsep = 0pt,
    parsep = 0pt,
}
\section{毛泽东思想的历史}\label{sec:1}
\begin{enumerate}[start=1]
  \item 毛泽东思想是在「新民主主义革命」、「社会主义革命」、「社会主义建设」的过程中发展起来的。
  \item 毛泽东的功绩:
  \begin{enumerate}[label={\roman{enumii})}]
    \item 中国共产党的创立发展;
    \item 中国人民解放军的创立发展;
    \item 新中国的缔造;
    \item 社会主义基本制度的建立;
    \item 毛泽东思想的创立。
  \end{enumerate}
  \item 19 世纪末 20 世纪初,世界进入帝国主义、无产阶级革命时代,\Emph{1}{战争与革命}为时代主题。
\end{enumerate}

\begin{mybox}{1}
\textbf{关于毛泽东思想的发展进程…}
\begin{enumerate}
  \item 毛泽东思想在\Emph{1}{土地革命战争}时期形成,在\Emph{1}{抗日战争}时期走向成熟,并在\Emph{1}{解放战争}时期和中华人民共和国成立后继续发展。
  \item 大革命时期,毛泽东写出《中国社会各阶层的分析》《湖南农民运动考察报告》,提出新民主主义革命基本思想,标志着毛泽东思想\Emph{1}{开始萌芽}。
  \item 毛泽东在《中国的红色政权为什么能够存在?》《井冈山的斗争》《星星之火,可以燎原》《反对本本主义》等著作中,提出农村包围城市、武装夺取政权的思想,标志着毛泽东思想\Emph{1}{初步形成}。
  \item 遵义会议后,毛泽东在《实践论》《矛盾论》著作中,分析了“左”的和右的错误的思想根源。系统阐述了新民主主义革命理论,标志着毛泽东思想得到多方面展开而\Emph{1}{趋于成熟}。
  \item 解放战争时期和新中国成立以后继续发展,毛泽东先后提出了人民民主专政理论、社会主义改造理论、关于严格区分和正确处理两类矛盾的学说等。
  \item \Emph{1}{1945 中共七大}把毛泽东思想写入党章。
\end{enumerate}
\end{mybox}

\begin{enumerate}[start=10]
  \item 毛泽东思想历史地位:
  \begin{enumerate}[label={\roman{enumii})}]
    \item 马克思主义中国化、时代化的\Emph{1}{第一个}理论成果
    \item 中国革命和建设的科学指南
    \item 党和人民的宝贵精神财富
  \end{enumerate}
\end{enumerate}

\begin{mybox}{1}
\textbf{关于毛泽东思想「活的灵魂」…}
\begin{enumerate}
  \item \Emph{1}{实事求是}、\Emph{1}{群众路线}、\Emph{1}{独立自主} $\in S_1$,其中 $S_1=\{\text{活的灵魂}\}$。
  \item $i=1$:毛泽东界定「实事求是」的含义:「实事」就是客观存在着的一切事物,「是」就是客观事物的规律性,「求」是研究。
  \item $i=1'$:实事求是就是一切从实际出发,理论联系实际,坚持在实践中检验真理和发展真理。这是毛泽东思想的\Emph{1}{基本点、精髓}。
  \item $i=2$:群众路线就是一切为了群众,一切依靠群众,从群众中来,到群众中去,把党的正确主张变为群众的自觉行动。
  \item $i=3$:独立自主,就是坚持独立思考,走自己的路同时积极争取外援,开展国际交流。
\end{enumerate}
\end{mybox}

\begin{enumerate}[start=16]
  \item 1981 年十一届六中全会《关于建国以来党的\Emph{1}{若干历史问题的决议}》对毛泽东思想科学体系的基本内容作了概括。毛泽东思想的主要内容有:
  \begin{enumerate}[label={\roman{enumii})}, start=1]
    \item 新民主主义革命理论
    \item 社会主义革命和社会主义建设理论
    \item 革命军队建设和军事战略的理论
    \item 政策和策略的理论
    \item 思想政治工作和文化工作的理论
    \item 党的建设理论
  \end{enumerate}
  \item 毛泽东思想是马克思主义中国化时代化的第一个重大理论成果,是马克思列宁主义在中国的运用和发展。
\end{enumerate}

\newpage
\backgroundsetup{contents={}, center, scale=1, angle=0, opacity=1}
\BgThispage
\pagecolor{c2-med}
\setlist{
    label = {\thesection-\arabic{enumi}.},
    format = {\color{c2-emph}},
    resume,
    itemsep = 0pt,
    parsep = 0pt,
}
\section{「新民主主义革命」相关理论}\label{sec:2}
\begin{enumerate}[start=1]
  \item \Emph{2}{认清中国国情}是解决中国革命问题的基本前提。
  \item 近代中国社会的主要矛盾是 (i)$^{\text{主要}}$ 帝国主义vs中华民族,(ii) 封建主义vs人民大众。近代中国革命的根本任务是,推翻帝国主义、封建主义、官僚资本主义的统治,实现民族独立和人民解放。
  \item 以\Emph{2}{五四运动}为标志,中国革命进入新民主主义革命阶段。
\end{enumerate}

\begin{mybox}{2}
\textbf{关于「新民主主义革命」的相关信息…}
\begin{enumerate}
  \item 1948 年,毛泽东完整表述了新民主主义革命总路线的内容,即无产阶级领导的,人民大众的,反对帝国主义、封建主义和官僚资本主义的革命。
  \item 新民主主义革命的\Emph{2}{对象}是:(i)$^{\text{首要}}$帝国主义、(ii) 封建主义、(iii) 官僚资本主义。
  \item 新民主主义革命的\Emph{2}{动力}包含:
  \begin{enumerate}[label={\roman{enumii})}]
    \item 无产阶级——基本动力
    \item 农民——主力军
    \item 城市小资产阶级——可靠同盟者
    \item 民族资产阶级——之一
  \end{enumerate}
  \item 农民问题是中国革命的\Emph{2}{基本问题}:
  \begin{enumerate}[label={\roman{enumii})}, start=1]
    \item 新民主主义革命实质上就是党领导下的农民革命;
    \item 中国革命战争实质上就是党领导下的农民战争;
    \item 工人阶级对农民的领导是实现革命领导权的基础。
  \end{enumerate}
  \item 无产阶级的领导权是中国革命的\Emph{2}{中心问题},也是新民主主义革命理论的核心问题。
\end{enumerate}
\end{mybox}

\begin{enumerate}[start=9]
  \item 新民主主义革命的\Emph{2}{性质}是资产阶级民主主义革命$^{\text{不是无产阶级!}}$,前途是社会主义。
  \item 新、旧民主主义革命的主要区别如表 \ref{tab:新-旧民主主义革命} 所示。
\end{enumerate}
\begin{table}[htb]
  \centering  \small
  \rowcolors{2}{hydro!5}{hydro!15}
  \begin{tabular}{ccc}
  \rowcolor{hydro}
   & \textcolor{white}{新民主主义革命} & \textcolor{white}{旧民主主义革命} \\
  领导力量$^{\text{根本标志}}$ & 无产阶级 & 资产阶级 \\
  和世界革命关系 & 无产阶级革命的一部分 & 资产阶级革命的一部分 \\
  指导思想 & 马克思列宁主义 & 资产阶级平等、自由、民主共和观念 \\
  革命前途 & 进入社会主义社会 & 建立资产阶级民主共和国 \\
  \end{tabular}
  \caption{新旧民主主义革命区别}\label{tab:新-旧民主主义革命}
\end{table}

\begin{mybox}{2}
\textbf{关于「新民主主义革命」的纲领…}
\begin{enumerate}
  \item 中国革命的历史进程分两步走:(i) 新民主主义革命,(ii) 社会主义革命。
  \item 新民主主义的\Emph{2}{政治纲领}是:推翻帝国主义和封建主义的统治,建立一个无产阶级领导的、以工农联盟为基础的、各革命阶级联合专政的新民主主义的共和国。国体为各革命阶级联合专政,政体为民主集中制的人民代表大会制度。
  \item 新民主主义的\Emph{2}{经济纲领}是:没收封建地主阶级的土地归农民所有$^{\text{也是主要内容}}$,没收官僚资产阶级的垄断资本归新民主主义的国家所有,保护民族工商业。
  \item \textcolor{hydro!60}{没收封建地主阶级的土地归农民所有,是新民主主义革命的主要内容。}
  \item 新民主主义\Emph{2}{文化},就是无产阶级领导的人民大众的反帝反封建的文化,即民族的科学的大众的文化。
\end{enumerate}
\end{mybox}

\begin{enumerate}[start=16]
  \item 走农村包围城市、武装夺取政权道路,根本在于处理好土地革命、武装斗争、农村革命根据地建设三者之间的关系。
\end{enumerate}

\begin{mybox}{2}
\textbf{关于「三大法宝」…}
\begin{enumerate}
  \item \Emph{2}{统一战线}、\Emph{2}{武装斗争}、\Emph{2}{党的建设}$\in S_2$,其中 $S_2=\{\text{战胜敌人的三大法宝}\}$。
  \item $i=1$:中国共产党领导的统一战线先后经历了几个时期。新民主主义革命统一战线的策略方针为区别对待,即坚持发展进步势力,争取中间势力,孤立顽固势力。
  \item $i=2$:坚持共产党对人民军队的绝对领导,是建设新型人民军队的根本原则,是毛泽东建军思想的核心。
  \item $i=3$:新民主主义革命时期党的建设的主要经验:(i) 把党的思想建设放在首位;(ii) 重视党的组织建设;(iii) 重视党的作风建设;(iv) 必须联系党的政治路线加强党的建设。
\end{enumerate}
\end{mybox}

\begin{enumerate}[start=21]
  \item 贯彻\Emph{2}{民主集中制}这一根本组织原则。
  \item 党在新民主主义革命的过程中,把党的建设作为一项「伟大的工程」,逐步形成了理论联系实际、密切联系群众、批评与自我批评相结合的三大优良作风,这是中国共产党区别于其他任何政党的\Emph{2}{显著标志}。
\end{enumerate}

\newpage
\backgroundsetup{contents={}, center, scale=1, angle=0, opacity=1}
\BgThispage
\pagecolor{c3-med}
\setlist{
    label = {\thesection-\arabic{enumi}.},
    format = {\color{c3-emph}},
    resume,
    itemsep = 0pt,
    parsep = 0pt,
}
\section{「社会主义改造」相关理论}\label{sec:3}
\begin{enumerate}[start=1]
  \item 从新中国成立到社会主义改造基本完成,是我国从新民主主义到社会主义过渡的时期。这一时期我国社会的性质是「\Emph{3}{新民主主义社会}」。
\end{enumerate}

\begin{mybox}{3}
\textbf{关于新民主主义社会…}
\begin{enumerate}
  \item 新民主主义社会的特点:
  \begin{enumerate}[label={\roman{enumii})}]
    \item 经济上,五种经济成分并存
    \item 政治上,工人阶级为领导、工农联盟为基础,包括小资产阶级和民族资产阶级联合专政的人民民主专政的国家制度。
    \item 文化上,马克思主义为指导的民族的、科学的、大众的文化。
  \end{enumerate}
  \item 新民主主义社会的主要矛盾:
  \begin{enumerate}[label={\roman{enumii})}, start=1]
    \item 建国初到解决土地问题之前,人民大众 vs \{地主阶级、国民党残余势力和帝国主义势力\}的矛盾仍然是当时社会的主要矛盾。
    \item 土地改革基本完成后,工人阶级 vs 资产阶级的矛盾逐步成为主要矛盾。
  \end{enumerate}
\end{enumerate}
\end{mybox}

\begin{enumerate}[start=4]
  \item 1953 年 6 月正式提出\Emph{3}{过渡时期}的总路线和总任务。即要在一个相当长的时期内,逐步实现国家的社会主义工业化,并逐步实现国家对农业、对手工业和对资本主义工商业的社会主义改造。
\end{enumerate}

\begin{mybox}{3}
\textbf{关于过渡时期总路线…}
\begin{enumerate}
  \item 党在过渡时期的总路线的\Emph{3}{理论依据}:
  \begin{enumerate}[label={\roman{enumii})}, start=1]
    \item 马克思、恩格斯——科学社会主义理论提出了资本主义向社会主义过渡的问题
    \item 列宁——实践发展了上述转变思想
    \item 毛泽东——积极探讨中国社会逐步过渡的问题
  \end{enumerate}
  \item 党在过渡时期的总路线的\Emph{3}{现实依据}:
  \begin{enumerate}[label={\roman{enumii})}, start=1]
    \item 物质基础——社会主义国营经济
    \item 重要依据——农民互助合作的需求
    \item 重要因素——调整过程中出现加工订货等国家资本主义形式
    \item 国际环境——社会主义国际充满向上发展的活力
  \end{enumerate}
\end{enumerate}
\end{mybox}

\begin{mybox}{3}
\textbf{关于「三大改造」…}
\begin{enumerate}[start=7]
  \item 适合中国特点的\Emph{3}{农业社会主义改造}:
  \begin{enumerate}[label={\roman{enumii})}, start=1]
    \item 引导农民组织走互助合作道路
    \item 遵循自愿互利、典型示范和国家帮助原则
    \item 制定正确的阶级政策
    \item 坚持积极领导、稳步前进方针
    \item 采取循序渐进的步骤:创造了「互助组 $\to$ 初级社 $\to$ 高级社」的过渡形式
  \end{enumerate}
  \item 对\Emph{3}{手工业的社会主义改造}经历了三个步骤:手工业供销小组 $\to$ 供销合作社 $\to$ 生产合作社。
  \item \Emph{3}{自愿互利原则},是中国对农业、手工业实行社会主义改造的基本原则。
  \item 适合中国特点的\Emph{3}{资本主义工商业}社会主义改造:
  \begin{enumerate}[label={\roman{enumii})}, start=1]
    \item 从低级到高级的国家资本主义形式过渡
    \item \Emph{3}{和平赎买}+「四马分肥」
    \item 把对企业的改造和对人的改造相结合
  \end{enumerate}
\end{enumerate}
\end{mybox}

\begin{enumerate}[start=11]
    \item 社会主义改造的历史经验:
    \begin{enumerate}[label={\roman{enumii})}, start=1]
        \item 坚持社会主义工业化建设与社会主义改造同时并举;
        \item 采取积极引导、逐步过渡的方式;
        \item 用和平方法进行改造。
    \end{enumerate}
    \item \Emph{3}{1956 年底},我国三大改造基本完成,标志着社会主义基本制度初步确立,我国从此进入\Emph{3}{社会主义初级阶段}。
    \item \Emph{3}{1954 年}颁布第一部宪法,即「五四宪法」,确立制度体系。
    \begin{enumerate}[label={\roman{enumii})}, start=1]
      \item 我国的\Emph{3}{国体}是人民民主专政;
      \item 我国的\Emph{3}{政体}是人民代表大会制度。
    \end{enumerate}
    \item 伴随着社会经济制度和社会经济结构的根本变化,我国社会的阶级关系也发生了根本的变化。包括:
    \begin{enumerate}[label={\roman{enumii})}, start=1]
        \item 帝国主义消灭
        \item 官僚资产灭绝
        \item 地主转为劳动者
        \item 民族资产阶级也改造
        \item 工人阶级领导
        \item 农民成为集体劳动者
        \item 知识界为社会主义服务
    \end{enumerate}
    \item 社会主义基本制度确立有重大意义。
\end{enumerate}

\newpage
\backgroundsetup{contents={}, center, scale=1, angle=0, opacity=1}
\BgThispage
\pagecolor{c4-med}
\setlist{
    label = {\thesection-\arabic{enumi}.},
    format = {\color{c4-emph}},
    resume,
    itemsep = 0pt,
    parsep = 0pt,
}
\section{社会主义建设道路初步探索的理论成果}\label{sec:4}
\begin{enumerate}[start=1]
  \item 1956 年毛泽东提出《论十大关系》,提出「以苏为鉴」,开始探索中国的社会主义建设道路。
  \item 《论十大关系》中谈到的 10 大关系,都是围绕着一个基本方针,就是要把积极因素调动起来为社会主义事业服务。它标志着党探索中国社会主义建设道路的\Emph{4}{良好开端}。
\end{enumerate}

\begin{mybox}{4}
\textbf{关于社会主义的矛盾…}
\begin{enumerate}
  \item 社会主义的\Emph{4}{基本矛盾}。
  \begin{enumerate}[label={\roman{enumii})}, start=1]
    \item 内容仍然是:生产关系vs生产力之间的矛盾,经济基础vs上层建筑之间的矛盾。
    \item 同以往矛盾不同,具备非对抗性。
    \item 可以经由社会主义制度不断解决。
  \end{enumerate}
  \item 我国社会的\Emph{4}{主要矛盾}和\Emph{4}{根本任务}。1956 中共八大提出:
  \begin{enumerate}[label={\roman{enumii})}, start=1]
    \item 主要矛盾:人民对于经济文化迅速发展的需要vs当前经济文化不能满足人民需要的状况之间的矛盾
    \item 根本任务:由解放生产力,变为在新的生产关系下面保护和发展生产力
  \end{enumerate}
  \item 社会主义社会两类\Emph{4}{不同性质}的矛盾,是 (i) 敌我矛盾、 (ii) 人民内部矛盾$^{\text{主导}}$。
  \item 正确处理人民内部矛盾的具体方针、原则,总方针是\Emph{4}{民主}的方法解决。
  \begin{enumerate}[label={\roman{enumii})}, start=1]
    \item 政治思想——「团结—批评—团结」方针
    \item 物质分配——统筹兼顾、适当安排
    \item 人民与政府——民主集中制
    \item 科学文化——「百花齐放、百家争鸣」
    \item 共产党和民主党——「长期共存、互相监督」
    \item 民族矛盾——民族平等、团结互助
  \end{enumerate}
\end{enumerate}
\end{mybox}

\begin{mybox}{4}
\textbf{关于社会主义建设道路初步探索的「主要」理论成果…}
\begin{enumerate}[start=7]
  \item 关于走中国工业化道路的思想:以工业为主导,把重工业作为我国经济建设的重点。
  \item 关于经济建设方针,党的八大提出既反保守又反冒进、在综合平衡中稳步前进的方针。提出「以农业为基础,工业为主导」。陈云提出「\Emph{4}{三个主体、三个补充}」。
  \item 社会主义民主政治建设上,提出扩大社会主义民主,加强社会主义法制。
  \item 关于发展社会主义民主政治,党的八大提出
  \begin{enumerate}[label={\roman{enumii})}, start=1]
    \item 扩大社会主义民主,开展反对官僚主义的斗争
    \item 加强对国家工作的监督
    \item 着手系统制定完备法律
  \end{enumerate}
\end{enumerate}
\end{mybox}

\begin{enumerate}[start=11]
  \item 社会主义发展阶段:(i)不发达的社会主义;(ii)比较发达的社会主义$^{\text{时间较长}}$。
  \item 「四个现代化」:
  \begin{enumerate}[label={\roman{enumii})}, start=1]
    \item 率先由\Emph{4}{周恩来}在 1954 年提出构想;
    \item 毛泽东 1957 年进一步「声明方法」;
    \item 毛泽东 1959 年补充,包含工、农、科学文化、国防现代化 4 个元素。
  \end{enumerate}
  \item 周恩来 1964 年\Emph{4}{正式提出}「四个现代化」的战略。
\end{enumerate}

\begin{mybox}{4}
\textbf{关于初步探索的其他理论成果…}
\begin{enumerate}
    \item 关于科学和教育,提出「向科学进军」口号。
    \item 关于国防建设,中央作出开展「三线建设」、加强备战的重大战略部署。
    \item 关于祖国统一,党对台政策的调整经历一个变化过程。周恩来将对台政策归纳为「一纲四目」。
    \item 关于国际战略和外交工作,提出2个「中间地带」、「三个世界划分」等战略思想。
    \item 关于党的建设,提出反对党内主观主义、宗派主义、官僚主义,批评脱离实际、脱离群众的思想作风,强调坚持民主集中制和集体领导制度。
\end{enumerate}
\end{mybox}

\begin{enumerate}[start=19]
  \item 社会主义建设道路初步探索的意义:
  \begin{enumerate}[label={\roman{enumii})}, start=1]
    \item 巩固和发展了我国的社会主义制度;
    \item 为开创中国特色社会主义提供了宝贵经验、理论准备和物质基础;
    \item 丰富了科学社会主义的理论和实践。
  \end{enumerate}
\end{enumerate}

\begin{mybox}{4}
\textbf{关于初步探索的经验教训…}
\begin{enumerate}[start=20]
  \item 社会主义建设道路初步探索的经验教训:$^{\text{(考研主观题作此要求)}}$
  \begin{enumerate}[label={\roman{enumii})}, start=1]
    \item 把马克思主义与中国实际结合,探索符合中国特点的社会主义建设道路;
    \item 正确认识社会主义社会的主要矛盾和根本任务,集中力量发展生产力;
    \item 从实际出发进行社会主义建设,建设规模和速度要和国力相适应,不能急于求成;
    \item 发展社会主义民主,健全社会主义法制;
    \item 坚持党的民主集中制和集体领导制度,加强执政党建设;
    \item 坚持对外开放,借鉴吸收人类文明成果建设社会主义,不能关门搞建设。
  \end{enumerate}
  \item 正确评价改革开放前后两个不同的历史时期。
\end{enumerate}
\end{mybox}


\newpage
\backgroundsetup{contents={}, center, scale=1, angle=0, opacity=1}
\BgThispage
\pagecolor{c5-med}
\setlist{
    label = {\thesection-\arabic{enumi}.},
    format = {\color{c5-emph}},
    resume,
    itemsep = 0pt,
    parsep = 0pt,
}
\section{中国特色社会主义理论体系的形成、发展}\label{sec:5}
\begin{enumerate}[start=1]
  \item 中国特色社会主义理论体系,是党在(i)洞察国际形势、(ii)判断时代主题、(iii)把握发展规律的基础上形成发展的。
  \item 20 世纪 70 年代,世界发生大变动。最显著的变化就是:\Emph{5}{和平与发展}成为时代主题,世界多极化和经济全球化深入发展,综合国力竞争日趋激烈。
  \item 党的十八大以来,世界之变、时代之变、历史之变正以前所未有的方式展开,世界百年未有之大变局加速演进。
  \item 社会主义建设初步探索时期,犯了不少错误,走了不少弯路。这主要是在经济上急于求成、盲目求纯和急于过渡;在政治上以阶级斗争为纲。原因:
  \begin{enumerate}[label={\roman{enumii})}, start=1]
    \item 偏离了党的实事求是的思想路线;
    \item 对什么是社会主义和如何建设社会主义的问题没有完全搞清楚。
  \end{enumerate}
  \item 党之所以赢得人民的拥护,是因为党作为「两个先锋队」,「三个代表」。
  \item 中国共产党百年奋斗的历史经验,包括坚持以下10个内容:党的领导;人民至上;理论创新;独立自主;中国道路;胸怀天下;开拓创新;敢于斗争;统一战线;自我革命。
  \item 改革开放「十个结合」的宝贵经验\textcolor{dendro!70!black}{$^{\text{(太长了,客观题应该考不了)}}$}
  \item 2012 年党的十八大以来,以习近平同志为核心的党中央创立了习思想。
\end{enumerate}

\begin{mybox}{5}
\textbf{关于中国特色社会主义理论体系的发展…}
\begin{enumerate}
  \item \Emph{5}{1978 年}的\Emph{5}{十一届三中全会},重新确立\Emph{5}{实事求是}思想路线,彻底否定「以阶级斗争为纲」,把全党工作的着重点转移到\Emph{5}{社会主义现代化建设}上来,作出实行\Emph{5}{改革开放}的重大决策,实现伟大转折。
  \item 1982 年党的十二大,邓小平提出「中国特色社会主义」的命题。
  \item 1984 年党的十二届三中全会,作出了《中共中央关于经济体制改革的决定》,提出了社会主义经济是在公有制基础上的有计划的商品经济。
  \item 1987 年党的十三大,系统论述了我国社会主义初级阶段理论,明确提出「一个中心、两个基本点」的基本路线。
  \item 1992 年党的十四大,概括了中国特色社会主义理论的主要内容。
  \item 1997 年党的十五大,正式提出\Emph{5}{「邓小平理论」}这一概念。
  \item 「邓小平理论」在 1997 年十五大写入党章,1999年写入宪法。
  \item 2002 年党的十六大,全面阐述\Emph{5}{「三个代表」}重要思想,并写入党章。
  \item 「三个代表」重要思想加深了建党认识,积累了治党治国经验。\textcolor{dendro!70!black}{$^{\text{(参考第七章)}}$}
  \item 2007 年党的十七大,将\Emph{5}{科学发展观}写入党章。
  \item 2010 年党的十七届五中全会,强调在当代中国,坚持发展是硬道理的本质要求,就是坚持科学发展,更加注重以人为本,更加注重全面协调可持续发展,更加注重统筹兼顾,更加注重保障和改善民生,促进社会公平正义。
  \item 党的十九大、十九届六中全会提出的「十个明确」「十四个坚持」「十三个方面成就」概括了习思想的主要内容。
  \item 2017 年党的十九大,将\Emph{5}{习近平新时代中国特色社会主义思想}写入党章;2018 年写入宪法。
  \item 中国特色社会主义理论体系的历史地位:
  \begin{enumerate}[label={\roman{enumii})}]
    \item 中特理论体系是党长期探索的伟大理论创造,是马克思主义中国化时代化的重大理论成果;
    \item 中特理论体系是思想基础、精神支柱、行动指南、根本指针、精神财富。
  \end{enumerate}
\end{enumerate}
\end{mybox}

\begin{enumerate}[start=23]
  \item 经济全球化使世界经济紧密联系。科技竞争成为综合国力竞争的焦点。
  \item 20 世纪 80 年代末 90 年代初,发生了\Emph{5}{东欧剧变、苏联解体}等重大事件,国际共产主义运动遭受重大挫折。苏联解体后,国际局势发生深刻变化。
  \item 冷战结束后,江泽民提出\Emph{5}{和平与发展}仍然是当今时代主题、科学技术已经成为先进生产力的集中体现和主要标志、社会主义仍然代表人类未来发展方向等重大论断。
  \item 长期以来人类创造了前所未有的经济增长成就,但由于单纯追求经济增长,或者照搬别国发展模式,一些国家发展遇到了种种问题。
\end{enumerate}

\begin{mybox}{5}
\textbf{关于中国特色社会主义理论体系的相关论断…}
\begin{enumerate}
  \item 中特理论体系,凝结了几代中国共产党人带领人民不懈探索实践的智慧和心血,充分汲取了百余年来党领导人民进行伟大奋斗积累的历史经验。
  \item 中特理论体系,是在认真总结我国社会主义建设正反两方面的历史经验、科学判断党和国家发展所处历史方位的基础上形成并不断发展的。
  \item 我们党历经革命、建设和改革,已经
  \begin{enumerate}[label={\roman{enumii})}]
    \item 从为了夺取全国政权而奋斗的党,成为领导人民掌握全国政权并长期执政的党;
    \item 已经从受到外部封锁和实行计划经济条件下领导国家建设的党,成为对外开放和发展社会主义市场经济条件下领导国家建设的党。
  \end{enumerate}
  \item 中特理论体系,是在改革开放、社会主义现代化建设的生动实践中形成发展的。
\end{enumerate}
\end{mybox}

\begin{enumerate}[start=31]
  \item 我国改革从农村$^\text{(家庭联产承包责任制)}$ $\to$ 城市$^\text{(经济体制改革)}$,确立社会主义市场经济的改革方向,坚决推进经济体制改革,探索经济制度。
  \item 我国改革开放和社会主义现代化建设的崭新实践,是人民群众的伟大创造,是理论发展的源泉。
\end{enumerate}

\newpage
\backgroundsetup{contents={}, center, scale=1, angle=0, opacity=1}
\BgThispage
\pagecolor{c6-med}
\setlist{
    label = {\thesection-\arabic{enumi}.},
    format = {\color{c6-emph}},
    resume,
    itemsep = 0pt,
    parsep = 0pt,
}
\section{邓小平理论}\label{sec:6}
\begin{enumerate}[start=1]
  \item 1986 年,邓小平强调:「社会主义财富属于人民,社会主义的致富是全民共同致富。社会主义原则,第一是发展生产,第二是共同致富。」
  \item 1992 年,邓小平在\Emph{6}{南方谈话}中指出\Emph{6}{社会主义的本质}:「就是解放生产力,发展生产力,消灭剥削,消除两极分化,最终达到共同富裕。」
  \item 邓小平关于社会主义本质的概括,把对社会主义的认识提高到新的水平。
  \item \Emph{6}{解放思想、实事求是},是邓小平理论的精髓。邓小平深刻阐明了解放思想和实事求是的辩证统一关系,只有解放思想才能达到实事求是,只有实事求是才是真正的解放思想。
  \item 邓小平发表了《解放思想,实事求是,团结一致向前看》的重要讲话,标志着党重新确立了马克思主义的思想、政治、组织路线。
\end{enumerate}

\textcolor{c6-emph}{(注:邓小平理论由 7 个部分构成,以下用大写罗马数字在框首标出。)}

\begin{mybox}{6}
\textbf{关于(I)社会主义初级阶段和党的基本路线理论…}
\begin{enumerate}
  \item 1987 年\Emph{6}{党的十三大},系统地阐述了社会主义初级阶段的\Emph{6}{科学内涵}。
  \item 我国社会主义的初级阶段,特指我国在生产力落后、商品经济不发达条件下建设社会主义必然要经历的特定阶段。
  \item 1997 年党的十五大,进一步概括了社会主义初级阶段的特征,强调现在处于并将长时期处于社会主义初级阶段是中国最大的实际。
  \item\label{itm:6-社会主义初级阶段基本路线} 1987 年党的十三大,提出了党在\Emph{6}{社会主义初级阶段的基本路线}:「领导和团结全国各族人民,以经济建设为中心,坚持四项基本原则,坚持改革开放,自力更生,艰苦创业,为把我国建设成为富强、民主、文明的社会主义现代化国家而奋斗。」
  \item 党在社会主义初级阶段:
  \begin{enumerate}[label={\roman{enumii})}]
    \item 奋斗目标——建设富强、民主、文明的社会主义现代化国家
    \item 基本途径——以经济建设为中心,坚持四项基本原则,坚持改革开放
    \item 领导力量和依靠力量——领导和团结各族人民
    \item 根本立足点——自力更生,艰苦创业
  \end{enumerate}
  \item 四项基本原则包括:
  \begin{enumerate}[label={\roman{enumii})},start=1]
    \item 坚持社会主义道路
    \item 坚持人民民主专政
    \item 坚持共产党的领导
    \item 坚持马克思列宁主义毛泽东思想
  \end{enumerate}
  \item 「一个中心,两个基本点」包括:
  \begin{enumerate}[label={\roman{enumii})}, start=1]
    \item 以经济建设为中心(新时期\Emph{6}{最根本的拨乱反正})
    \item 坚持四项基本原则(立国之本)
    \item 坚持改革开放(强国之路)
  \end{enumerate}
  \item 关于党的基本路线\textcolor{c6-emph}{(参见 \ref{itm:6-社会主义初级阶段基本路线})}:
  \begin{enumerate}[label={\roman{enumii})}, start=1]
    \item 1987 年党的十三大:$S_{13} = \{\text{富强},\text{民主},\text{文明}\}$
    \item 2007 年党的十七大:$S_{17} = \{\text{富强},\text{民主},\text{文明},\text{和谐}\}$
    \item 2017 年党的十九大:$S_{19} = \{\text{富强},\text{民主},\text{文明},\text{和谐},\text{美丽}\}$
  \end{enumerate}
\end{enumerate}
\end{mybox}

\begin{mybox}{6}
\textbf{关于 (II) 社会主义根本任务和发展战略理论…}
\begin{enumerate}[start=14]
  \item 社会主义的\Emph{6}{根本任务}是发展生产力,中国解决所有问题的关键是靠自己的发展,发展生产力离不开科学技术。1988 年邓小平提出\Emph{6}{科学技术是第一生产力}。
  \item 为了更好地实现「三步走」发展战略,邓小平提出三个战略重点:一是农业,二是能源和交通,三是教育和科学。
  \item 关于中国农业的长远发展战略,邓小平提出了两个飞跃的思想,包含:
  \begin{enumerate}[label={\roman{enumii})}, start=1]
    \item 废除人民公社,实行家庭联产承包为主的责任制。
    \item 适应科学种田和生产社会化的需要,发展适度规模经营,发展集体经济。
  \end{enumerate}
  \item 关于地区的「先富」和「共富」,邓小平提出「两个大局」思想。
  \begin{enumerate}[label={\roman{enumii})}, start=1]
    \item 沿海先发展,带动内地更好地发展,内地要顾全这个大局。
    \item 发展到一定,沿海要拿更多力量帮助内地发展,那时沿海也要服从大局。
  \end{enumerate}
\end{enumerate}
\end{mybox}

\begin{mybox}{6}
\textbf{关于 (III) 改革开放和市场经济理论…}
\begin{enumerate}[start=18]
  \item \Emph{6}{改革}是社会主义社会发展的\Emph{6}{直接动力}。社会主义社会的\Emph{6}{基本矛盾}仍然是生产关系和生产力、上层建筑和经济基础之间的矛盾。
  \item 判断改革和各方面工作是非得失的根本标准:
  \begin{enumerate}[label={\roman{enumii})}, start=1]
    \item 是否有利于发展社会主义社会的生产力;
    \item 是否有利于增强社会主义国家的综合国力;
    \item 是否有利于提高人民的生活水平。
  \end{enumerate}
  \item 开放也是改革,对外开放是建设中国特色社会主义的一项\Emph{6}{基本国策},和改革一起成为新时期最鲜明的特征。
  \item 在实行对外开放过程中,邓小平还特别强调两点:
  \begin{enumerate}[label={\roman{enumii})}, start=1]
    \item 要正确对待资本主义社会创造的现代文明成果;
    \item 要高度珍惜并坚决维护中国人民经过长期奋斗得来的独立自主权利。
  \end{enumerate}
  \item 改革开放是当代中国的鲜明标志和活力源泉,是决定中国命运的关键一招,也是实现中华民族伟大复兴的关键一招。(「\Emph{6}{两个一招}」)
  \item 1992 年党的十四大,明确把建立\Emph{6}{社会主义市场经济体制}作为我国经济体制改革的目标。
  \item 邓小平的社会主义市场经济理论具有丰富的内涵:
  \begin{enumerate}[label={\roman{enumii})}, start=1]
    \item 计划经济和市场经济不是划分社会制度的标志;
    \item 计划和市场都是经济手段;
    \item 市场经济作为资源配置手段本身不具有制度属性。
  \end{enumerate}
  \item 1986 年党的十二届六中全会,初步形成「三位一体」的总体布局:
  \begin{enumerate}[label={\roman{enumii})}, start=1]
    \item 以经济建设为中心(一体);
    \item 坚定不移地进行经济体制改革,坚定不移地进行政治体制改革,坚定不移地加强精神文明建设(三位)。
  \end{enumerate}
\end{enumerate}
\end{mybox}

\begin{mybox}{6}
\textbf{关于 (IV) 两手抓、(V) 祖国统一、(VI) 外交战略、(VII) 党的建设理论…}
\begin{enumerate}[start=26]
  \item 邓小平强调物质、精神文明「两手抓,两手都要硬」,以及其他一系列两手抓。
  \item 「一国两制」作为实现和平统一的重大战略构想,包含丰富的科学内涵。
  \item 维护我国的独立和主权,促进世界的和平与发展,是中国外交政策的基本目标。
  \item \Emph{6}{和平共处五项原则}:(i)互相尊重主权和领土完整、(ii)互不侵犯、(iii)互不干涉内政、(iv)平等互利、(v)和平共处。
  \item 建设中国特色社会主义,关键在于坚持、加强和改善党的领导。
  \item 加强组织建设是党的建设的重要环节。党的基层组织是党的全部工作和战斗力的基础。
  \item 加强领导班子建设,培养和选拔德才兼备的各级领导干部,是加强党的建设,保证党的路线的连续性和国家长治久安的根本大计。
  \item 执政党的党风关系党的生死存亡。
\end{enumerate}
\end{mybox}

\begin{enumerate}[start=34]
  \item 邓小平理论的历史地位:
  \begin{enumerate}[label={\roman{enumii})}, start=1]
    \item 是马克思列宁主义、毛泽东思想的继承和发展;
    \item 是中国特色社会主义理论体系的开篇之作;
    \item 是改革开放和社会主义现代化建设的科学指南。
  \end{enumerate}
\end{enumerate}

\newpage
\backgroundsetup{contents={}, center, scale=1, angle=0, opacity=1}
\BgThispage
\pagecolor{c7-med}
\setlist{
    label = {\thesection-\arabic{enumi}.},
    format = {\color{c7-emph}},
    resume,
    itemsep = 0pt,
    parsep = 0pt,
}
\section{「三个代表」思想}\label{sec:7}
\begin{enumerate}[start=1]
  \item 「三个代表」思想形成的条件:
  \begin{enumerate}[label={\roman{enumii})}, start=1]
    \item 是在对冷战结束后国际局势科学判断的基础上形成的;
    \item 是在科学判断党的历史方位和总结历史经验的基础上提出来的;
    \item 是在建设中国特色社会主义伟大实践的基础上形成的。
  \end{enumerate}
  \item 1997 年党的十五大,江泽民概括了新时期党的建设的总目标,提出「建设一个什么样的党,怎样建设党」的问题。
  \item 2000 年,江泽民首次对「三个代表」进行了比较全面的阐述。
  \item 2000 年,江泽民指出 「三个代表」思想要回答和解决的正是「建设一个什么样的党,怎样建设党」的问题。
  \item 2002 年\Emph{7}{党的十六大},将「三个代表」思想写入党章。
  \item 2004 年,将「三个代表」思想写入宪法。
  \item 2001 年 12 月 11 日,中国正式加入世界贸易组织,成为其第 143 个成员。
\end{enumerate}

\begin{mybox}{7}
\textbf{关于「三个代表」思想的基本信息…}
\begin{enumerate}
  \item 「三个代表」思想的\Emph{7}{核心观点}包括: 中国共产党必须
  \begin{enumerate}[label={\roman{enumii})}, start=1]
    \item 始终代表中国先进生产力的发展要求;
    \item 始终代表中国先进文化的前进方向;
    \item 始终代表中国最广大人民的根本利益。
  \end{enumerate}
  \item 广大\Emph{7}{工人、农民和知识分子}是推动我国先进生产力发展和社会全面进步的根本力量。
  \item \Emph{7}{科学技术是第一生产力},是先进生产力的集中体现和主要标志。
  \item 「三个代表」思想的\Emph{7}{主要内容}包括:
  \begin{enumerate}[label={\roman{enumii})}, start=1]
    \item 发展是党执政兴国的第一要务;
    \item 建立社会主义市场经济体制;
    \item 全面建设小康社会;
    \item 建设社会主义政治文明;
    \item 实施「引进来」和「走出去」相结合的对外开放战略;
    \item 推进党的建设新的伟大工程。
  \end{enumerate}
  \item 「三个代表」思想的历史地位:(i)「三个代表」是中国特色社会主义理论体系的丰富发展;(ii)「三个代表」是加强和改进党的建设,推进中国特色社会主义事业的强大理论武器。
\end{enumerate}
\end{mybox}

\begin{enumerate}[start=13]
  \item 人是生产力中最活跃的因素。开发人力资源,加强人力资源能力建设,是关系我国发展的重大问题。
  \item 发展先进文化,就是发展面向现代化、面向世界、面向未来的,民族的科学的大众的社会主义文化。
  \item 加强社会主义思想道德建设,是发展先进文化的重要内容和中心环节。
  \item 我们党作为执政党,面临的最根本的课题,是能不能始终代表最广大人民的根本利益,始终全心全意为人民服务。
  \item 三个「代表」是统一的整体,相互联系、相互促进。其中:(i)先进生产力是基础条件,(ii)人民群众是创造主体,(iii)实现最广大人民的根本利益是发展目标。
\end{enumerate}

\begin{mybox}{7}
\textbf{关于「三个代表」思想的主要内容…}
\begin{enumerate}
    \item 要正确认识和处理改革、发展、稳定的关系。改革是动力,发展是目的,稳定是前提。
    \item 1997 年党的十五大,初步勾画了实现全面建设小康社会的蓝图:
    \begin{enumerate}[label={\roman{enumii})}, start=1]
        \item $[2000,2010]$:第一个十年实现国民生产总值比 2000 年翻一番;
        \item $[2010,2020]$:到建党一百年时,使国民经济更加发展,各项制度更加完善;
        \item $[2020,2050]$:到世纪中叶建国一百年时,基本实现现代化,建成富强民主文明的社会主义国家。
    \end{enumerate}
    \item 2002 年党的十六大,把社会主义物质文明、政治文明、精神文明一起确立为社会主义现代化全面发展的三大基本目标。
    \item 建设社会主义政治文明,必须坚持和完善中国特色社会主义政治制度。(1根本+3基本)
    \item 坚持中国共产党的领导,核心是\Emph{7}{坚持党的先进性}。推进党的建设新的伟大工程,重点是加强党的\Emph{7}{执政能力建设}。
    \item 党的\Emph{7}{基层组织}是党的全部工作和战斗力的基础。
    \item 党的作风,关系党的形象,关系人心向背,关系党的生命。
    \item 贯彻「三个代表」重要思想,\Emph{7}{关键}在坚持与时俱进,\Emph{7}{核心}在坚持党的先进性,\Emph{7}{本质}在坚持执政为民。
\end{enumerate}
\end{mybox}

\newpage
\backgroundsetup{contents={}, center, scale=1, angle=0, opacity=1}
\BgThispage
\pagecolor{c8-med}
\setlist{
    label = {\thesection-\arabic{enumi}.},
    format = {\color{c8-emph}},
    resume,
    itemsep = 0pt,
    parsep = 0pt,
}
\section{科学发展观}\label{sec:8}
\begin{enumerate}[start=1]
  \item 2003 年党的十六届三中全会,第一次提出科学发展观的概念和内涵:「坚持以人为本,树立全面协调、可持续的发展观,促进经济社会和人的全面发展。」
  \item 科学发展观的\Emph{8}{科学内涵}:
  \begin{enumerate}[label={\roman{enumii})}, start=1]
    \item 推动经济社会发展是科学发展观的\Emph{8}{第一要义};
    \item 以人为本是科学发展观的\Emph{8}{核心立场};
    \item 全面协调可持续是科学发展观的\Emph{8}{基本要求};
    \item 统筹兼顾是科学发展观的\Emph{8}{根本方法}。
  \end{enumerate}
  \item 科学发展观是在深刻把握我国社会主义初级阶段基本国情和新的阶段性特征基础上形成和发展的。
\end{enumerate}

\textcolor{c8-emph}{(注:为厘清逻辑关系,条目 $3\sim 11$ 较原版有顺序上的区别。)}

\begin{mybox}{8}
\textbf{关于科学发展观的具体内容…}
\begin{enumerate}
  \item 坚持科学发展:
  \begin{enumerate}[label={\roman{enumii})}, start=1]
    \item 必须加快转变经济发展方式;
    \item 必须推动科学技术的跨越式发展;
    \item 必须培养高素质创新型人才;
    \item 必须善于抓住和用好机遇。
  \end{enumerate}
  \item 坚持以人为本:就是坚持 (i)发展为了人民,(ii) 发展依靠人民; (iii)发展成果由人民共享。
  \item 坚持全面发展:就是要按照中国特色社会主义事业总体布局,正确认识和把握经济建设、政治建设、文化建设、社会建设、生态文明建设是相互联系、相互促进的有机统一体。
  \item 坚持协调发展,就是保证中国特色社会主义各个领域协调推进。
  \item 坚持可持续发展,必须走生产发展、生活富裕、生态良好的文明发展道路。
  \item 全面协调可持续中的
  \begin{enumerate}[label={\roman{enumii})}, start=1]
    \item 全面指整体性,不仅经济发展,而且各个方面都要发展;
    \item 协调指均衡性,各个方面、各个环节的发展要相互适应、相互促进;
    \item 可持续指持久性,不仅当前要发展,而且要保证长远发展。
  \end{enumerate}
  \item 统筹兼顾是科学发展观的根本方法。
  \item 坚持统筹兼顾,要牢牢掌握统筹兼顾的科学思想方法,努力提高战略思维、创新思维、辩证思维能力,不断增强统筹兼顾的本领,更好地推动科学发展。
\end{enumerate}
\end{mybox}

\begin{enumerate}[start=12]
  \item 科学发展观的\Emph{8}{主要内容}:
  \begin{enumerate}[label={\roman{enumii})}, start=1]
    \item 加快转变经济发展方式;
    \item 发展社会主义民主政治;
    \item 推进社会主义文化强国建设;
    \item 构建社会主义和谐社会;
    \item 推进生态文明建设;
    \item 全面提高党的建设科学化水平。
  \end{enumerate}
  \item 社会主义民主政治的本质和核心是\Emph{8}{人民当家作主}。
  \item 「四大考验」与「四大危险」:
  \begin{enumerate}[label={\roman{enumii})}, start=1]
    \item 「四大考验」指党面临的执政考验、改革开放考验、市场经济考验、外部环境考验;
    \item 「四大危险」指精神懈怠危险、能力不足危险、脱离群众危险、消极腐败危险。
  \end{enumerate}
  \item 科学发展观最鲜明的\Emph{8}{精神实质}是:解放思想、实事求是、与时俱进、求真务实。
  \item 科学发展观的历史地位:中国特色社会主义理论体系的接续发展;全面建设小康社会、加快推进社会主义现代化的根本方针。
\end{enumerate}


\end{document} 