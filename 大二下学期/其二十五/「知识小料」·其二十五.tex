\documentclass[UTF8]{ctexart}
\usepackage{amsmath}
\usepackage{amssymb}
\usepackage{background}
\usepackage{booktabs}
\usepackage{caption,subcaption}
\usepackage{enumitem}
\usepackage{fancyhdr}
\usepackage{float}
\usepackage{fontspec}
\usepackage{fourier}
\usepackage{geometry}
\usepackage{makecell}
\usepackage{tcolorbox}
\tcbuselibrary{breakable, raster}
\usepackage{tikz}
\usetikzlibrary{arrows.meta}
\usepackage{xcolor}

\geometry{a5paper, top=0.1cm, left=1cm, right=1cm, bottom=1cm, footskip=0.6cm, marginparsep=0.1cm}

\setCJKmainfont[BoldFont={汉仪文黑-85W},ItalicFont={方正苏新诗柳楷简体}]{汉仪文黑-55W}
\setfontfamily\Issue{Century Schoolbook}
\setfontfamily\Genshin{Genshin Teyvat Lingua Franca}
\newCJKfontfamily\TitleFont{思源宋体 CN Heavy}
\newfontfamily\timesnewroman{Times New Roman}
\captionsetup{font=small, labelfont=bf}
\setlist[itemize]{itemsep=0pt, parsep=0pt}
\pagestyle{fancy}
\fancyhf{}
\cfoot{\ttfamily\footnotesize{-\ \thepage\ -}}
%\reversemarginpar

%\CTEXsetup[format = {\centering\bfseries\large}, beforeskip = 3pt, afterskip = 3pt]{section}
\CTEXsetup[format = {\color{cyan!50!black}\bfseries\large}]{subsection}

\newtcolorbox{mybox}{colback=cyan!10, colframe=cyan!50!black, boxrule=0.5pt, breakable}

\colorlet{darkcyan}{cyan!50!black}
\newcommand\Black[1]{\textcolor[gray]{0.3}{#1}}
\newcommand\Brown[1]{\textcolor[HTML]{998A4E}{#1}}
\newcommand\Emph[1]{\colorbox{green!10}{\textcolor{green!30!black}{#1}}}
\newcommand\Concept[1]{\textcolor{cyan!70!black}{#1}}
\newcommand\Notes[1]{\textcolor{yellow!50!black}{\small #1}}
\newcommand\Example[1]{\textcolor{cyan!70!black}{\small #1}}
\newcommand\means[1]{\textcolor{cyan!70!black}{#1}}

\newcommand\C{\mathrm{C}}
\newcommand\e{\mathrm{e}}
\renewcommand\d{\mathrm{d}}
\newcommand\Cov{\mathrm{Cov}}
\renewcommand\pi{\text{\timesnewroman π}}

\newcommand\IssueNumber{25}
\newcommand\Date{2024-4-22}
%\newcommand\Contributer{@金光日}
\newcommand\Subject{概率与统计 A}
%\newcommand\Source{2022 考研 408 第 5 题}


\begin{document}
\backgroundsetup{contents=\includegraphics{上半示例.png}, center, scale=1, angle=0, opacity=1}
\BgThispage
\begin{center}
%{\scriptsize\Issue \textcolor[HTML]{C8BA83}{\Genshin WEEKLY TIPS}}
\phantom{...}

{\Large\textcolor{brown!40!white}{\makebox[10cm][s]{\Genshin WEEKLY KNOWLEDGE TIPS}}}

\vspace{-2em}

{\Huge\bfseries\TitleFont \Black{知\ 识\ 小\ 料}}


\vspace{-0.1cm}
{\footnotesize \Brown{「电计 2203 班」周常规知识整理共享}}
\end{center}

\vspace{-0.5cm}


\begin{figure}[H]
\hspace{1cm}
\begin{minipage}[t]{0.3\textwidth}
\centering
    \Brown{\Genshin ISSUE}

    \vspace{-0.6cm}
    \Huge \Issue\slshape\bfseries\Black{\IssueNumber}
\end{minipage}
\hfill
\begin{minipage}[t]{0.35\textwidth}
\small
\centering
    \Brown{日期:\Date} \\
%\vspace{-0.1cm}
%    \Brown{贡献者:\Contributer} \\
\vspace{-0.1cm}
    \Brown{学科:\Subject} \\
%\vspace{-0.1cm}
%    \Brown{来源:\Source}
\end{minipage}
\hspace{0.8cm}
\end{figure}

{\color{cyan!50!black}
本文档用于对《概率与统计A》课程作出简明复习。}

\section{常见的分布}
%1-2
\begin{tcbitemize}[raster columns=2, raster equal height=rows, colframe=cyan!75!black, colback=cyan!5!white, fonttitle=\bfseries]
\tcbitem[squeezed title={二项分布 $X\sim B(n,p)$}]
\phantom{...}
\vspace{-1.2em}
\begin{equation*}
    P(X=k) = \C{}_n^k p^k (1-p)^{n-k}
\end{equation*}

$E(X)=\means{np}$

$D(X)=\means{np(1-p)}$

特殊性质:无
\tcbitem[squeezed title={泊松分布 $X\sim P(\lambda)$}]
\[P(X=k) = \dfrac{\lambda^k}{k!}\e^{-\lambda}\]

$E(X)=\means{\lambda}$

$D(X)=\means{\lambda}$

特殊性质:泊松逼近定理
\end{tcbitemize}

%3-4
\begin{tcbitemize}[raster columns=2, raster equal height=rows, colframe=cyan!75!black, colback=cyan!5!white, fonttitle=\bfseries]
\tcbitem[squeezed title={几何分布 $X\sim G(p)$}]
\[P(X=k) = (1-p)^{k-1}\cdot p\]

$E(X)=\means{\dfrac1p}$

\vspace{0.7em}
$D(X)=\means{\dfrac{1-p}{p^2}}$

特殊性质:无记忆性
\tcbitem[squeezed title={均匀分布 $X\sim U(a,b)$}]
\[f(x)=\dfrac{1}{b-a} (a<x<b)\]

$E(X)=\means{\dfrac{a+b}2}$

\vspace{0.7em}
$D(X)=\means{\dfrac{(b-a)^2}{12}}$

特殊性质:无
\end{tcbitemize}

%5-6
\begin{tcbitemize}[raster columns=2, raster equal height=rows, colframe=cyan!75!black, colback=cyan!5!white, fonttitle=\bfseries]
\tcbitem[squeezed title={指数分布 $X\sim e(\theta)$}]
\[f(x) = \theta \e^{-\theta x} (x>0)\]

$E(X)=\means{\dfrac1{\theta}}$

\vspace{0.7em}
$D(X)=\means{\dfrac1{\theta^2}}$

特殊性质:无记忆性;极小分布 $W=\min\{X_i\}\sim e(n\theta)$
\tcbitem[squeezed title={正态分布 $X\sim N(\mu,\sigma^2)$}]
\[f(x)=\dfrac{1}{\sqrt{2\pi}\sigma}\e^{-\frac{(x-\mu)^2}{2\sigma^2}}\]

$E(X)=\means{\mu}$

$D(X)=\means{\sigma^2}$

特殊性质:太多了
\end{tcbitemize}

\newpage
\backgroundsetup{contents=\includegraphics{空白示例.png}, center, scale=1, angle=0, opacity=1}
\BgThispage

%7-8
\begin{tcbitemize}[raster columns=2, raster equal height=rows, colframe=cyan!75!black, colback=cyan!5!white, fonttitle=\bfseries]
\tcbitem[squeezed title={$\boldsymbol{\chi^2}$ 分布—— $\chi^2\sim \chi^2(n)$}]
$(X_1,X_2,\dots,X_n)\sim N(0,1)$ 独立

\[\means{\chi^2 = X_1^2 + X_2^2 + \cdots + X_n^2}\]

特殊性质:$E(\chi^2)=n$,$D(\chi^2)=2n$,$\chi^2(n)+\chi^2(m)\sim \chi^2(n+m)$

\tcbitem[squeezed title={$\boldsymbol{t}$ 分布—— $t\sim t(n)$}]
$X\sim N(0,1), \ Y\sim \chi^2(n)$ 独立

\[\means{t=\dfrac{X}{\sqrt{\dfrac{Y}{n}}}}\]

特殊性质:$t^2\sim F(1,n)$
\end{tcbitemize}

%9-10
\begin{tcbitemize}[raster columns=2, raster equal height=rows, colframe=cyan!75!black, colback=cyan!5!white, fonttitle=\bfseries]
\tcbitem[squeezed title={$\boldsymbol{F}$ 分布—— $F\sim F(n,m)$}]
$X\sim \chi^2(n), \ Y\sim \chi^2(m)$ 独立

\[\means{F=\dfrac{\quad \dfrac{X}{n}\quad }{\dfrac{Y}{m}}}\]

特殊性质:$\dfrac1F\sim F(m,n)$

\tcbitem[squeezed title={「实例12」}]
$X_1,X_2,\dots,X_n$ 相互独立同分布,$E(X_i)=\mu$,$D(X_i)=\sigma^2$,

\begin{equation*}
\begin{aligned}
   \bar{X} &=\dfrac1n\sum\limits_{i=1}^n X_i \\
   S^2 &=\dfrac1{n-1}\sum\limits_{i=1}^n (X_i-\bar{X})^2
\end{aligned}
\end{equation*}

$E(\bar{X}) = \means{\mu}$,$D(\bar{X}) = \means{\dfrac{\sigma^2}{n}}$

$E(S^2) = \means{\sigma^2}$

\end{tcbitemize}

\section{概率论基本概念}
\subsection{随机事件与运算}
样本空间 $\varOmega$,样本点 $\omega$,随机事件 $A,B\dots$,随机变量 $X,Y\dots$

随机事件关系:
\begin{itemize}[itemsep=0pt,parsep=0pt]
  \item 子事件:$A\subset B$
  \item 并事件:$C=A+B=A\cup B$
  \item 交事件:$C=AB=A\cap B$
  \item 差事件:$C=A-B = A\bar{B}$
  \item 互不相容:$AB=\varnothing$
  \item 分配律、德摩根律
\end{itemize}

概率定义的三个基本方面:非负性、归一性、可列可加性。
\begin{itemize}
  \item 减法公式:$P(A-B) = P(A\bar{B}) = P(A) - P(AB)$
  \item 加法公式:$P(A+B) = P(A) + P(B) - P(AB)$
  \item 单调性:$B\subset A \implies P(B)\leqslant P(A)$
  \item $A=\varnothing\implies P(A)=0$,$B=\varOmega\implies P(B)=1$,但反之不然
\end{itemize}

古典概型:$P(A) = \dfrac{n_A}{n_\varOmega} = \dfrac{A \text{的样本点数}}{\text{样本点总数}}$

\vspace{0.6em}
几何概型:$P(A) = \dfrac{|A|}{|\varOmega|} = \dfrac{A \text{的度量}}{\text{全部度量}}$

相关实例:摸球问题、抽签问题、生日问题等,先定位后挑选;分组法应用

\subsection{条件概率与公式}
条件概率:$P(A|B) = \dfrac{P(AB)}{P(B)}$

乘法公式:$P(AB) = P(B)\cdot P(A|B)$

全概率公式:设 $A_1,A_2,\dots$ 是 $\varOmega$ 的剖分,$P(A_i)>0$,对任意 $B$:
\begin{equation}\label{eq:全概率公式}
    P(B) = \sum\limits_{i} P(A_i)\cdot P(B|A_i)
\end{equation}
贝叶斯公式:
\begin{equation}\label{eq:贝叶斯公式}
    P(A_k|B) = \dfrac{P(A_k)\cdot P(B|A_k)}{\sum\limits_{i} P(A_i)\cdot P(B|A_i)}
\end{equation}

\subsection{事件独立性}
事件独立性的原初定义:$P(AB)=P(A)\cdot P(B)$

互不相容:$P(A+B)=P(A)+P(B)$

多个事件相互独立定义(略)

\begin{mybox}
\textbf{关于事件独立性…}
\begin{enumerate}
    \item $A_1,A_2,\dots$ 独立,则对它们做任意种类运算后得到的事件仍然相互独立。
    \item $X,Y$ 独立:
    \begin{itemize}[itemsep=0pt,parsep=0pt]
        \item $P(X=x_i,Y=y_j) = P(X=x_i)\cdot P(Y=y_j)$
        \item $P(X\leqslant x, Y\leqslant y) = P(X\leqslant x)\cdot P(Y\leqslant y)$
        \item $F(x,y) = F_X(x)\cdot F_Y(y)$
        \item $f(x,y) = f_X(x)\cdot f_Y(y)$
    \end{itemize}
    \item $X,Y$ 独立:
    \begin{itemize}[itemsep=0pt,parsep=0pt]
        \item $(X,Y)\sim N(\mu_1, \mu_2,\sigma_1^2, \sigma_2^2,\rho) \iff \rho=0$
    \end{itemize}
    \item $X,Y$ 独立:
    \begin{itemize}[itemsep=0pt,parsep=0pt]
        \item $g(x),h(y)$ 为连续函数 $\iff g(X),h(Y)$ 独立
    \end{itemize}
    \item $X,Y$ 独立:
    \begin{itemize}[itemsep=0pt,parsep=0pt]
        \item $E(XY)=E(X)E(Y)$
        \item $D(X+Y)=D(X)+D(Y)$
        \item $\Cov(X,Y)=0$
        \item $\rho_{XY}=0$(注:互不相关不一定相互独立)
    \end{itemize}
\end{enumerate}
\end{mybox}

\section{一维、二维随机变量及其分布}
\subsection{一维随机变量}
\Concept{随机变量}:从样本空间 $\varOmega$ 到实数集 $\mathbb{R}$ 的映射,记为 $X=X(\omega)$。

\Concept{分布函数}:$F(x) := P(X\leqslant x)$\textcolor{cyan}{(借用离散数学符号 $:=$ 表示定义)}
\begin{itemize}[itemsep=0pt,parsep=0pt]
  \item 单调递增:$x_1<x_2 \implies F(x_1)\leqslant F(x_2)$
  \item 收敛性:$F(-\infty) = 0$,$F(+\infty) = 1$
  \item 右连续性:$\lim\limits_{h\to 0^+} F(x+h) = F(x)$,对任意 $x$
\end{itemize}

离散型随机变量\Concept{分布列}:$p_i := P(X=x_i)$($i=1,2,\dots$)
\begin{itemize}[itemsep=0pt,parsep=0pt]
  \item 非负性:$P(X=x_i)\geqslant 0$
  \item 归一性:$\sum_{i=1}^\infty  P(X=x_i) = 1$
\end{itemize}

连续性随机变量\Concept{密度函数}:$f(x)$ 满足 $\displaystyle F(x)=\int_{-\infty}^t f(t)\d t$($\forall x\in \mathbb{R}$)
\begin{itemize}[itemsep=0pt,parsep=0pt]
  \item 非负性:$f(x)\geqslant 0$,对任意 $x$
  \item 归一性:$\displaystyle\int_{-\infty}^{+\infty} f(x)\d x=1$
  \item 求导:若 $f(x)$ 连续,则 $F(x)$ 可微且 $f(x)=F'(x)$
\end{itemize}
分布函数 $F(x)$ 须写成左闭右开形式,密度函数 $f(x)$ 不要求。

\Concept{随机变量函数}:随机变量 $X:\varOmega\to\mathbb{R}$ 与函数 $g:\mathbb{R}\to\mathbb{R}$ 的复合,记为 $Y=g(X)$。

\subsection{一维随机变量的其他内容}
常见随机变量的分布,详见第一页。

求随机变量函数的密度函数的方法:写取值范围—列平凡情况—解非平凡情况/转化反解—求导

泊松逼近定理的应用:$X\sim B(n,p)$,$n$ 很大,但 $np$ 不太大,通常 $np\leqslant 5$(有时是 $n(1-p)\leqslant 5$),则有
\begin{equation}\label{eq:泊松逼近定理}
    P(X=k) = \C_n^k p^k (1-p)^{n-k} \approx \dfrac{(np)^k}{k!}\e^{-np}
\end{equation}
即让 $\lambda=np$。

几何分布的无记忆性:$X\sim G(p)$,对任意自然数 $m,n$ 均有 $P(X>n+m|X>n) = P(X>m)$。

指数分布的无记忆性:$X\sim e(\theta)$,对任意 $s,t>0$ 均有 $P(X>s+t|X>s) = P(X>t)$。

正态分布的标准化:$X\sim N(\mu,\sigma^2)$ 经过 $Y=\dfrac{X-\mu}{\sigma}$ 得到 $Y\sim N(0,1)$。

标准正态分布密度函数:
\begin{equation}\label{eq:标准正态分布密度函数}
    \varphi(x) = \dfrac{1}{\sqrt{2\pi}} \e^{-\frac{x^2}{2}}\qquad X\sim N(0,1)
\end{equation}
分布函数 $\varPhi(x)$ 是偶函数,$\varPhi(0)=\frac12$。

若 $X\sim N(\mu,\sigma^2)$,$Y=aX+b$,则 $Y\sim N(a\mu + b, a^2\sigma^2)$。正态分布标准化 $Y=\frac{X-\mu}{\sigma}$ 是特例,取 $a=\frac1{\sigma}$,$b=-\frac{\mu}{\sigma}$。

\subsection{二维随机变量}
\Concept{二维随机变量}:从 $\varOmega$ 到 $\mathbb{R}^2$ 的映射,记为 $(X,Y)=(X(\omega), Y(\omega))$。

\Concept{联合分布函数}:$F(x,y) := P(X\leqslant x, Y\leqslant y)$
\begin{itemize}[itemsep=0pt,parsep=0pt]
\color{gray!80!white}
    \item 单调递增:给定$y$,$x_1<x_2\implies F(x_1,y)\leqslant F(x_2,y)$(给定$x$类似)
    \item 收敛性:$F(-\infty,y) = F(x,-\infty) = F(-\infty,-\infty) = 0$,$F(+\infty,+\infty)=1$。
    \item 右连续性:给定$y$,$\lim\limits_{h\to 0^+} F(x+h,y) = F(x,y)$(给定$x$类似)
    \item 矩形法则:设$x_1<x_2,y_1<y_2$,则 $F(x_2,y_2) - F(x_1,y_2) - F(x_2,y_1) + F(x_1,y_1)\geqslant 0$
\end{itemize}

\Concept{边缘分布函数}:$F_X(x) := F(x,+\infty)$,$F_Y(y) = F(+\infty,y)$

离散型随机变量\Concept{联合分布列}:$p_{ij} := P(X=x_i,Y=y_j)$
\begin{itemize}[itemsep=0pt,parsep=0pt]
\color{gray!80!white}
    \item 非负性:$p_{ij}\geqslant 0$
    \item 归一性:$\sum\limits_{i} \sum\limits_{j} p_{ij} =1$
\end{itemize}

\Concept{边缘分布列}见表 \ref{tab:2-d   discrete}。

\begin{table}[htb]
  \centering
  \begin{tabular}{ccc}
  \toprule
    种类 & 表达式 & 简记 \\
  \midrule
    联合 $(X,Y)$ & $P(X=x_i,\quad Y=y_j)$ & $p_{ij}$ \\
    边缘 $X$ & $P(X=x_i) = P\left(X=x_i,\quad  \bigcup\limits_{j=1}^m \{Y=y_j\} \right)$ & $p_{i\text{\textbullet}}$ \\
    边缘 $Y$ & $P(Y=y_j) = P\left(\bigcup\limits_{i=1}^n\{X=x_i\},\quad  Y=y_j \right)$ & $p_{\text{\textbullet}j}$ \\
  \bottomrule
  \end{tabular}
  \caption{二维离散型随机变量的三种分布列\textcolor{cyan}{(\textbullet 相当于通配符)}}\label{tab:2-d discrete}
\end{table}

\Concept{条件分布列}:给定 $x_i$ 及 $p_{i\text{\textbullet}}>0$,称一列数:$P(Y=y_j|X=x_i) = \dfrac{P(X=x_i,Y=y_j)}{P(X=x_i)}$ 为 $X=x_i$ 发生条件下的 $y$ 的条件分布列。简记:$p_{j|i} = \dfrac{p_{ij}}{p_{i\text{\textbullet}}}$

连续性随机变量\Concept{联合密度函数}:$f(x,y)$ 满足 $\displaystyle F(x,y)=\int_{-\infty}^x \int_{-\infty}^y  f(u,v)\d u\d v$($\forall x,y\in\mathbb{R}$)
\begin{itemize}[itemsep=0pt,parsep=0pt]
\color{gray!80!white}
    \item 非负性:$f(x,y)\geqslant 0$
    \item 归一性:$\displaystyle\iint_{\mathbb{R}^2} f(x,y)\d x\d y = 1$
    \item 求偏导:若 $f$ 连续,则 $F$ 二阶可微且 $f(x,y)=F_{xy}''(x,y)$
\end{itemize}

\Concept{边缘密度函数}见下框。
\begin{mybox}
\textbf{关于联合确定边缘…}

$\displaystyle f_X(x) = f(x,+\infty) = \int_{-\infty}^{+\infty} f(x,y)\d y$

\vspace{1em}
$\displaystyle f_Y(y) = f(+\infty,y) = \int_{-\infty}^{+\infty} f(x,y)\d x$

\end{mybox}

\Concept{条件密度函数}:

对给定 $Y=y$,$f_Y(y)>0$,则 $f_{X|Y}(x|y) = \dfrac{f(x,y)}{f_Y(y)}$ 是 $X$ 的条件密度函数;

对给定 $X=x$,$f_X(x)>0$,则 $f_{Y|X}(y|x) = \dfrac{f(x,y)}{f_X(x)}$ 是 $Y$ 的条件密度函数。

\Concept{二维随机变量函数}:随机变量 $(X,Y):\varOmega\to \mathbb{R}^2$ 与二元函数 $g:\mathbb{R}^2\to\mathbb{R}$ 的复合,记为 $Z=g(X,Y)$。

\subsection{二维随机变量的其他内容}
二维均匀分布:$(X,Y)$,$D$ 是 $\mathbb{R}^2$ 的有界区域,$|D|$ 为面积。若
\begin{equation}\label{eq:二维均匀分布}
    f(x,y) = \begin{cases}
                    \frac{1}{|D|},\quad &(x,y)\in D \\
                    0,&\text{其他}
              \end{cases}
\end{equation}
则称 $(X,Y)$ 服从$D$上的二维均匀分布。

二维正态分布:$(X,Y)$,若
\begin{equation}\label{eq:二维正态分布}
    f(x,y) = \dfrac{1}{2\pi\sigma_1\sigma_2\sqrt{1-\rho^2}} \e^{-\frac{1}{2(1-\rho^2)} \left[\frac{(x-\mu_1)^2}{\sigma_1^2} - \frac{2\rho(x-\mu_1)(y-\mu_2)}{\sigma_1\sigma_2} + \frac{(y-\mu_2)^2}{\sigma_2^2}\right]}
\end{equation}
则称 $(X,Y)$ 服从二维正态分布,记为 $(X,Y)\sim N(\mu_1,\mu_2,\sigma_1^2,\sigma_2^2,\rho)$。

相关系数 $\rho=0$(即 $X,Y$ 独立),且均值 $\mu_1=\mu_2=0$,方差 $\sigma_1^2 = \sigma_2^2 = 1$ 时:
\begin{equation}\label{eq:二维正态分布-rou=0}
    \color{cyan!70!black} f(x,y) = \dfrac{1}{2\pi} \e^{-\frac{x^2+y^2}{2}} \qquad (X,Y)\sim N(0,0,1,1,0)
\end{equation}
这与一维标准正态分布 \textcolor{cyan!70!black}{$f(x) = \dfrac{1}{\sqrt{2\pi}} \e^{-\frac{x^2}{2}}\qquad X\sim N(0,1)$} 很相似。

$X$ 与 $Y$ 有相同分布列:指 $X,Y$ 取值相同,且取到对应值的概率也相同。

极小分布与指数分布:若 $X_i\sim e(\theta)$ 且相互独立,$W=\min\limits_{1\leqslant i\leqslant n} \{X_i\}$,则 $W\sim e(n\theta)$。

泊松分布可加性:$X,Y$独立,$X\sim P(\lambda_1)$,$Y\sim P(\lambda_2)$,则 $Z=X+Y\sim P(\lambda_1+\lambda_2)$。

\begin{mybox}
\textbf{关于正态分布…}
\begin{enumerate}
    \item 一维线性组合:$X\sim N(\mu_1,\sigma_1^2)$,$Y\sim N(\mu_2,\sigma_2^2)$,且 $X,Y$ \Concept{相互独立},则
    \begin{equation*}
        aX+bY+c\sim N(a\mu_1+b\mu_2+c, \quad a^2\sigma_1^2+b^2\sigma_2^2)
    \end{equation*}
    $X_i\sim N(\mu_i, \sigma_i^2)$($i=1,2,\dots,n$),且各个 $X_i$ \Concept{相互独立},则
    \begin{equation*}
        b+\sum\limits_{i=1}^n  a_iX_i \sim N\left(b+\sum\limits_{i=1}^n  a_i\mu_i, \quad \sum\limits_{i=1}^n a_i^2\sigma_i^2\right)
    \end{equation*}
    \item 一维化二维:$X\sim N(\mu_1,\sigma_1^2)$,$Y\sim N(\mu_2,\sigma_2^2)$,且 $X,Y$ \Concept{相互独立}
    \begin{equation*}
        \iff (X,Y)\sim N(\mu_1,\mu_2,\sigma_1^2,\sigma_2^2,\textcolor{cyan!70!black}{0})
    \end{equation*}
    \item 二维化一维:若 $(X,Y)\sim N(\mu_1,\mu_2,\sigma_1^2,\sigma_2^2,\rho)$,则 $X,Y$ 的任意线性组合 $aX+bY+c$ 服从一维正态分布 $N(\hat{\mu},\hat{\sigma}^2)$,其中参数
    \begin{equation*}
    \begin{aligned}
        \hat{\mu} &= E(aX+bY+c) = a\mu_1+b\mu_2+c \\
        \hat{\sigma}^2 &= D(aX+bY+c) = D(aX+bY) = \Cov(aX+bY,aX+bY) \\
        %&= D(aX)+D(bY)+2\Cov(aX+bY) = a^2\sigma_1^2 + b^2\sigma_2^2 + 2\rho\sqrt{D(aX)\cdot D(bY)} \\
         &= a^2\sigma_1^2 + b^2\sigma_2^2 + 2\rho ab\sigma_1\sigma_2
    \end{aligned}
    \end{equation*}
    \item 二维协方差:若 $(X,Y)\sim N(\mu_1,\mu_2,\sigma_1^2,\sigma_2^2,\rho)$,则
    \begin{equation*}
    \begin{aligned}
        \Cov(X,Y) &= \rho_{XY}\cdot \sqrt{D(X)\cdot D(Y)} = \rho \sigma_1 \sigma_2 \\
        E(XY) &= \Cov(X,Y)+E(X)E(Y) = \rho\sigma_1\sigma_2 + \mu_1\mu_2 \\
    \end{aligned}
    \end{equation*}
\end{enumerate}

\end{mybox}

\section{随机变量数字特征}
\subsection{均值/数学期望}
以下当所属求和式或积分式绝对收敛时,\Concept{均值}存在。
\begin{enumerate}
  \item 一维随机变量:
  \begin{itemize}
    \item 离散 $X$,分布列 $p_i=P(X=x_i)$,则 $\displaystyle E(X) = \sum\limits_{i=1}^{\infty} x_ip_i$
    \item 连续 $X$,密度 $f(x)$,则 $\displaystyle E(X) = \int_{-\infty}^{+\infty}  xf(x)\d x$
  \end{itemize}
  \item 随机变量函数 $Y=g(X)$:
  \begin{itemize}
    \item 离散 $X$,分布列 $p_i=P(X=x_i)$,则 $\displaystyle E(Y) = E(g(X)) = \sum\limits_{i=1}^{\infty} g(x_i) \cdot P(X=x_i)$
    \item 连续 $X$,密度 $f(x)$,则 $\displaystyle E(Y) = E(g(X)) = \int_{-\infty}^{+\infty}  g(x) f(x)\d x$
  \end{itemize}
  \item 二维随机变量函数 $Z=g(X,Y)$:
  \begin{itemize}
    \item 离散 $(X,Y)$,分布列 $p_{ij}=P(X=x_i,Y=y_j)$,则 $\displaystyle E(Z) = E(g(X,Y)) = \sum\limits_{i=1}^\infty \sum\limits_{j=1}^\infty  g(x_i,y_j)\cdot p_{ij}$
    \item 连续 $(X,Y)$,密度 $f(x,y)$,则 $\displaystyle E(Z) = E(g(X,Y)) = \iint_{\mathbb{R}^2} g(x,y)f(x,y)\d x\d y$
  \end{itemize}
\end{enumerate}

$k$阶原点矩:$E(X^k)$

$k$阶中心矩:$E\left\{[X-E(X)]^k\right\}$

\subsection{方差}
\Concept{方差}定义为二阶中心矩:$D(X) := E\left\{[X-E(X)]^2\right\}$

\Concept{标准差}:$\sqrt{D(X)}$

计算方差:$D(X) = E(X^2) - [E(X)]^2$

\begin{table}[htb]
  \centering
  \begin{tabular}{ccc}
  \toprule
   & 均值 & 方差 \\
  \midrule
  常数 & $E(c)=c$ & $D(c)=0$ \\
  数乘 & $E(aX)=aE(X)$ & $D(aX)=a^2D(X)$ \\
  加常数 & $E(X+b)=E(X)+b$ & $D(X+b) = D(X)$ \\
  可加性 & $E(X+Y)=E(X)+E(Y)$ & 【独立】$D(X+Y)=D(X)+D(Y)$ \\
  可乘性 & 【独立】$E(XY)=E(X)E(Y)$ & — \\
  \bottomrule
  \end{tabular}
  \caption{均值和方差的性质}\label{tab:characteristic of mean and varience}
\end{table}

\subsection{协方差、相关系数}
\Concept{协方差}:$\Cov(X,Y) = E(XY)-E(X)E(Y)$
\begin{itemize}[itemsep=0pt,parsep=0pt]
  \item 自反性:$\Cov(X,X) = D(X)$
  \item 对称性:$\Cov(X,Y) = \Cov(Y,X)$
  \item 数乘:$\Cov(aX,bY) = ab\Cov(X,Y)$
  \item 加常数:$\Cov(X+a,Y+b) = \Cov(X,Y)$
  \item 分配律:$\Cov(X_1+X_2, Y) = \Cov(X_1,Y) + \Cov(X_2,Y)$
  \item 独立:若 $X,Y$ 独立,则 $\Cov(X,Y)=0$
  \item 与方差的关系:
\end{itemize}
\begin{equation*}
    D(aX+bY) = \Cov(aX+bY,aX+bY) = a^2D(X) + 2ab\Cov(X,Y) + b^2D(Y)
\end{equation*}

\Concept{相关系数}:
\begin{equation}\label{eq:rho}
    \rho_{XY} := \dfrac{\Cov(X,Y)}{\sqrt{D(X)}\cdot \sqrt{D(Y)}}
\end{equation}
它总是 $[-1,1]$ 之间的实数。

\begin{itemize}[itemsep=0pt,parsep=0pt]
  \item $\rho=1$ 为正线性相关,即存在 $a>0,b\in\mathbb{R}$ 使得 $P(Y=aX+b)=1$
  \item $\rho=-1$ 为负线性相关,即存在 $a<0,b\in\mathbb{R}$ 使得 $P(Y=aX+b)=1$
  \item $\rho=0$ 为互不相关,此时 $\Cov(X,Y)=0$。相互独立 $\implies$ 互不相关。
\end{itemize}

\subsection{其他内容}
随机变量\Concept{标准化}:$X$为均值和方差都存在的随机变量,$Y=\dfrac{X-E(X)}{\sqrt{D(X)}}$,则 $E(Y)=0$,$D(Y)=1$。

\Concept{切比雪夫不等式}:对任意 $a>0$,
\begin{equation}\label{eq:切比雪夫不等式}
  P\left(|X-E(X)|\geqslant a\right) \leqslant \dfrac{D(X)}{a^2}
\end{equation}
推论:$D(X)=0 \iff P(X=E(X))=1$。用于估计随机变量偏离中心的程度。

辛钦大数定律:设 $X_1,X_2,\dots,X_n$ 相互独立同分布,均值为 $\mu$,方差可以不存在,则 $\frac1n\sum_{i=1}^n X_i \xrightarrow{P} \mu$。样本均值依概率收敛到总体均值。

\section{数理统计、区间估计、点估计}
\subsection{数理统计、抽样分布}
总体:研究对象的全体。

样本:抽取的部分个体。

简单随机样本:相互独立且与总体 $X$ 同分布的样本 $(X_1,X_2,\dots,X_n)$。

统计量:关于样本的函数 $T(X_1,X_2,\dots,X_n)$,不含未知参数。
\begin{table}[htb]
\renewcommand\arraystretch{1.7}
  \centering
  \begin{tabular}{cc}
  \toprule
  名称 & 表达式 \\
  \midrule
  样本均值 & $\displaystyle \bar{X} = \dfrac1n\sum\limits_{i=1}^n X_i$ \\
  样本方差 & $\displaystyle S^2 = \dfrac{1}{n-1}\sum\limits_{i=1}^n (X_i-\bar{X})^2$ \\
  样本标准差 & $\displaystyle S = \sqrt{\dfrac{1}{n-1}\sum\limits_{i=1}^n (X_i-\bar{X})^2}$ \\
  样本$k$阶原点矩 & $\displaystyle A_k = \dfrac1n\sum\limits_{i=1}^n X_i^k$ \\
  极大次序统计量 & $X_{(n)} = \max\{X_1,X_2,\dots,X_n\}$ \\
  极小次序统计量 & $X_{(1)} = \min\{X_1,X_2,\dots,X_n\}$ \\
  \bottomrule
  \end{tabular}
  \caption{常见的统计量}\label{tab:统计量}
\end{table}

统计量的常用分布:标准正态分布、$\chi^2$分布、$t$分布、$F$分布。详见第二页。

\Concept{单正态总体抽样分布}:总体 $X\sim N(\mu,\sigma^2)$,样本 $(X_1,X_2,\dots,X_n)$,样本均值 $\bar{X} = \frac1n\sum_{i=1}^n X_i$,样本方差 $S^2 = \frac{1}{n-1}\sum_{i=1}^n (X_i-\bar{X})^2$,则有
\begin{mybox}
\begin{enumerate}
  \item $\dfrac{\bar{X}-\mu}{\sigma/\sqrt{n}} \sim N(0,1)$
  \item $\dfrac{n-1}{\sigma^2}S^2 \sim \chi^2(n-1)$,且 $\bar{X}$ 与 $S^2$ 独立
  \item $\dfrac{\bar{X}-\mu}{S/\sqrt{n}} \sim t(n-1)$
\end{enumerate}
\end{mybox}

\Concept{双正态总体抽样分布}:$X,Y$ 相互独立
\begin{table}[htb]
\centering
\renewcommand\arraystretch{1.0}
\begin{tabular}{ccccc}
    总体 & 样本 & 样本均值 & 样本方差 & 样本容量 \\
    $X\sim N(\mu_1,\sigma_1^2)$ & $(X_1,\dots,X_n)$ & $\bar{X} = \frac1n\sum_{i=1}^n X_i$ & $S_1^2 = \frac{1}{n-1}\sum_{i=1}^n (X_i-\bar{X})^2$ & $n$ \\
    $Y\sim N(\mu_2,\sigma_2^2)$ & $(Y_1,\dots,Y_m)$ & $\bar{Y} = \frac1m\sum_{i=1}^m Y_i$ & $S_2^2 = \frac{1}{m-1}\sum_{i=1}^m (Y_i-\bar{Y})^2$ & $m$ \\
\end{tabular}
\end{table}
\begin{mybox}
\begin{enumerate}
  \item $\dfrac{\bar{X}-\bar{Y}-(\mu_1-\mu_2)}{\sqrt{\dfrac{\sigma_1^2}{n} + \dfrac{\sigma_2^2}{m}}} \sim N(0,1)$
  \item $\dfrac{S_1^2}{S_2^2}\cdot \dfrac{\sigma_2^2}{\sigma_1^2} \sim F(n-1,m-1)$
  \item 当 $\sigma_1^2=\sigma_2^2$ 时有 $\dfrac{\bar{X}-\bar{Y}- (\mu_1-\mu_2)}{\sqrt{\dfrac{(n-1)S_1^2 + (m-1)S_2^2}{n+m-2}}\cdot\sqrt{\dfrac1n+\dfrac1m}} \sim t(n+m-2)$
\end{enumerate}
\end{mybox}

\Concept{上 $\alpha$ 分位点}:满足 $P(Y>Y_{\alpha}) = \alpha$ 的分位点 $Y_{\alpha}$。

构造上 $\alpha$ 分位点:
\begin{enumerate}
  \item 大概率区间
  \begin{itemize}[itemsep=0pt,parsep=0pt]
    \item $P(Y_{1-\frac{\alpha}2} < Y < Y_{\frac{\alpha}2}) = 1-\alpha$
    \item $P(Y<Y_{\alpha}) = 1-\alpha$
    \item $P(Y>Y_{1-\alpha}) = 1-\alpha$
  \end{itemize}
  \item 小概率区间
  \begin{itemize}[itemsep=0pt,parsep=0pt]
    \item $P(Y<Y_{1-\frac{\alpha}2} \text{或} Y>Y_{\frac{\alpha}2}) = \alpha$
    \item $P(Y>Y_\alpha)  = \alpha$
    \item $P(Y<Y_{1-\alpha})  = \alpha$
  \end{itemize}
\end{enumerate}

常用的统计量上 $\alpha$ 分位点如表所示。
\begin{table}[htb]
  \centering
  \begin{tabular}{cccc}
  \toprule
  分布类型 & $Y_\alpha$ & 概率 & 特殊性质 \\
  \midrule
  $Z\sim N(0,1)$ & $z_{\alpha}$ & $P(Z>z_\alpha) = \alpha$ & $z_{1-\alpha} = -z_\alpha$ \\
  $t\sim t(n)$ & $t_\alpha(n)$ & $P(t>t_{\alpha}(n)) = \alpha$ & $t_{1-\alpha}(n) = -t_\alpha(n)$ \\
  $\chi^2\sim \chi^2(n)$ & $\chi_\alpha^2(n)$ & $P(\chi^2 > \chi_\alpha^2(n)) = \alpha$ & 无\\
  $F\sim F(n,m)$ & $F_\alpha (n,m)$ & $P(F > F_\alpha(n,m)) = \alpha$ & $F_{1-\alpha}(n,m) = \dfrac{1}{F_{\alpha}(m,n)}$  \\
  \bottomrule
  \end{tabular}
  \caption{常用的统计量上 $\alpha$ 分位点}\label{tab:统计量上alpha分位点}
\end{table}

另外,对于带有对称性的 $Z\sim N(0,1)$ 和 $t\sim t(n)$,还有:
\begin{equation*}
\begin{aligned}
    P(z_{1-\frac{\alpha}{2}} < Z < z_{\frac{\alpha}{2}} ) = 1-\alpha 
    &\iff P(|Z|<z_{\frac{\alpha}{2}}) = 1-\alpha \\
    P(Z<z_{1-\frac{\alpha}{2}} \text{或} Z>z_{\frac{\alpha}{2}}) = \alpha
    &\iff P(|Z|>z_{\frac{\alpha}{2}}) = \alpha 
\end{aligned}
\end{equation*}
$t$ 分布同理。

\subsection{置信区间与假设检验}
总体 $X$,分布 $F(x;\theta)$,其中 $\theta$ 待估,样本 $(X_1,X_2,\dots,X_n)$。$T_1,T_2$ 是关于样本的统计量。
\begin{itemize}[itemsep=0pt,parsep=0pt]
  \item $P(T_1\leqslant \theta\leqslant T_2) = 1-\alpha$,\Concept{置信区间}为 $[T_1,T_2]$
  \item $P(T_1\leqslant \theta)=1-\alpha$,置信下限为 $T_1$
  \item $P(\theta\leqslant T_2)=1-\alpha$,置信上限为 $T_2$
\end{itemize}
其中置信度是 $1-\alpha$。

求置信区间的解法:
\begin{enumerate}[itemsep=0pt,parsep=0pt]
  \item 构造样本函数(单参数)
  \item 构造对应的大概率事件
  \item 解出置信区间
  \item 代值计算
\end{enumerate}

\Concept{假设检验}:认为「一次试验中小概率事件不会发生」。若发生了,就拒绝原假设 $H_0$;否则接受 $H_0$。

两种错误:一为弃真,概率为 $\alpha$;二为采伪,概率为 $\beta$。$\alpha,\beta$ 相互制约,要同时降低它们只能增加样本容量。这里只考虑弃真错误,减小 $\alpha$,称为「显著性检验」。$\alpha$ 称为显著性水平。

假设检验方法:
\begin{enumerate}[itemsep=0pt,parsep=0pt]
  \item 提出假设
  \item 选取检验统计量
  \item 构造对应的小概率事件
  \item 代值计算,判定统计量是否落入拒绝域
\end{enumerate}

\begin{table}[htb]
  \centering
  \begin{tabular}{c|c|c}
  $H_0:\theta=\theta_0$ & $H_0:\theta=\theta_0$ & $H_0:\theta=\theta_0$ \\
  $H_1:\theta\ne\theta_0$ & $H_1:\theta<\theta_0$ & $H_1:\theta>\theta_0$ \\
  $P(Y<Y_{1-\frac{\alpha}2} \text{或} Y>Y_{\frac{\alpha}2}) = \alpha$ & $P(Y<Y_{1-\alpha})  = \alpha$ & $P(Y>Y_\alpha)  = \alpha$ \\
  \end{tabular}
\end{table}

\subsection{点估计与优良性}
总体 $X$,分布 $F(x;\theta)$,其中 $\theta$ 待估,则 $E(X)=h(\theta)$。

\Concept{矩估计}:令样本原点矩等于总体原点矩,如一阶 $\frac1n\sum_{i=1}^n X_i = E(X)= h(\theta)$。解出来 $\hat{\theta} = \hat{\theta}(X_1,X_2,\dots,X_n)$ 是 $\theta$ 的矩估计量,代样本观测值得到的 $\hat{\theta} = \hat{\theta}(x_1,x_2,\dots,x_n)$ 则是 $\theta$ 的矩估计值。(依据:辛钦大数定律)
\begin{equation*}
  \dfrac1n\sum_{i=1}^n X_i^k \xrightarrow[\text{\footnotesize 依概率收敛}]{P} E(X^k)
\end{equation*}

如果是多参数($m$ 个),则让一阶、二阶、……、$m$阶样本原点矩等于总体原点矩,解方程组。 

\Concept{极大似然估计}:样本似然函数 $\displaystyle L(\theta) = \prod_{i=1}^{n} p(x_i;\theta)$ 的最大值点 $\hat{\theta}$ 就是极大似然估计量/估计值。

似然函数中的 $p(x_i,\theta)$ 为分布列或者密度函数:
\begin{itemize}
  \item 对离散型 $X$,$\displaystyle L(\theta) = \prod_{i=1}^{n} p(x_i;\theta) = \prod_{i=1}^n P(X=x_i)$
  \item 对连续型 $X$,$\displaystyle L(\theta) = \prod_{i=1}^{n} p(x_i;\theta) = \prod_{i=1}^{n} f(x_i;\theta)$
\end{itemize}

为了确定最大值点 $\hat{\theta}$,一般采用取对数法或者利用单调性等。

\Concept{无偏性}:若总体参数 $\theta$ 的估计量 $\hat{\theta}$ 满足 $E(\hat{\theta})=\theta$,则称 $\hat{\theta}$ 是无偏估计量。

\Concept{有效性}:在参数 $\theta$ 的所有无偏估计量中,称方差小者更有效。

\newpage
\backgroundsetup{contents=\includegraphics{下半示例.png}, center, scale=1, angle=0, opacity=1}
\BgThispage
\section{可能用到的工数知识}
\begin{enumerate}
  \item 求多元函数极值——拉格朗日乘数法。
  
  在约束条件 $\varphi(x,y,z)=0$ 下,求函数 $f(x,y,z)$ 的极值。令 $L(x,y,z;\lambda) = f(x,y,z) + \lambda\cdot\varphi(x,y,z)$,同时解四元方程组
  \begin{equation}\label{eq:拉格朗日乘数法}
    \dfrac{\partial L}{\partial x} = \dfrac{\partial L}{\partial y} = \dfrac{\partial L}{\partial z} =\dfrac{\partial L}{\partial \lambda} = 0
  \end{equation}
  解得的 $(x_0,y_0,z_0)$ 就是函数 $f(x,y,z)$ 的可能极值点。
  
  \item 直角坐标与极坐标积分互化。
  \begin{equation}\label{eq:直角坐标与极坐标积分互化}
    \iint\limits_{D} f(x,y)\d x\d y = \iint\limits_{D_{r\theta}} f(r\cos\theta,r\sin\theta)\cdot \Concept{r}\d r\d \theta
  \end{equation}
  
  \item 分部积分。
  \begin{equation}\label{eq:分部积分}
    \int f'(x)g(x)\d x = f(x)g(x) - \int g'(x)f(x)\d x
  \end{equation}
  
  \item 变限积分求导。当 $f(x)$ 连续时,
  \begin{equation}\label{eq:变上限积分}
    \left(\int_{\alpha(x)}^{\beta(x)} f(t)\d t\right)' = f(\beta(x))\cdot \beta'(x) - f(\alpha(x))\cdot \alpha'(x)
  \end{equation}
\end{enumerate}


\end{document} 