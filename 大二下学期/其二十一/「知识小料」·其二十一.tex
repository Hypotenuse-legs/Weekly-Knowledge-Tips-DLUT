\documentclass[UTF8]{ctexart}
\usepackage{amsmath}
\usepackage{amssymb}
\usepackage{booktabs}
\usepackage{background}
\usepackage{caption,subcaption}
\usepackage{CJKfntef}
\usepackage{enumitem}
\usepackage{float}
\usepackage{fontspec}
%\usepackage{fourier}
\usepackage{geometry}
\usepackage{listings}
\usepackage{tikz}
\usetikzlibrary{arrows.meta}
\usepackage{xcolor}

\geometry{a5paper, top=0.1cm, left=1cm, right=1cm, bottom=0.3cm, footskip=0.1cm}
\setCJKmainfont[BoldFont={汉仪文黑-85W},ItalicFont={方正苏新诗柳楷简体}]{汉仪文黑-55W}
\setfontfamily\Issue{Century Schoolbook}
\setfontfamily\Genshin{Genshin Teyvat Lingua Franca}
\newCJKfontfamily\TitleFont{思源宋体 CN Heavy}
\newfontfamily\timesnewroman{Times New Roman}
%\reversemarginpar

%\CTEXsetup[format = {\centering\bfseries\large}, beforeskip = 3pt, afterskip = 3pt]{section}
\CTEXsetup[format = {\color{cyan!50!black}\bfseries\large}]{subsection}

\colorlet{darkcyan}{cyan!50!black}
\newcommand\Black[1]{\textcolor[gray]{0.3}{#1}}
\newcommand\Brown[1]{\textcolor[HTML]{998A4E}{#1}}
\newcommand\Emph[1]{\colorbox{green!10}{\textcolor{green!30!black}{#1}}}
\newcommand\Notes[1]{\textcolor{yellow!50!black}{\small #1}}
\newcommand\Example[1]{\textcolor{cyan!70!black}{\small #1}}

\renewcommand\d{\mathrm{d}}

\lstset{
    basicstyle=\ttfamily, %注意行末有逗号!
    keywordstyle=\ttfamily\bfseries\color{blue!70!black},
    commentstyle=\ttfamily\color{cyan!90!black},
    stringstyle=\ttfamily\color{green!40!black},
    columns=flexible,
    numbers=left,
    numberstyle=\footnotesize,
    escapechar=`,
    frame=shadowbox,
    %rulesepcolor=\color{red!20!blue!20!green!20}
    backgroundcolor=\color{cyan!5!white},
    language=C,
}

\newcommand\IssueNumber{21}
\newcommand\Date{2024-3-30}
%\newcommand\Contributer{@金光日}
\newcommand\Subject{数据结构与算法}


\begin{document}
\backgroundsetup{contents=\includegraphics{上半示例.png}, center, scale=1, angle=0, opacity=1}
\BgThispage
\begin{center}
%{\scriptsize\Issue \textcolor[HTML]{C8BA83}{\Genshin WEEKLY TIPS}}
\phantom{...}

{\Large\textcolor{brown!40!white}{\makebox[10cm][s]{\Genshin WEEKLY KNOWLEDGE TIPS}}}

\vspace{-2em}

{\Huge\bfseries\TitleFont \Black{知\ 识\ 小\ 料}}


\vspace{-0.1cm}
{\footnotesize \Brown{「电计 2203 班」周常规知识整理共享}}
\end{center}

\vspace{-0.5cm}


\begin{figure}[H]
\hspace{1cm}
\begin{minipage}[t]{0.3\textwidth}
\centering
    \Brown{\Genshin ISSUE}

    \vspace{-0.6cm}
    \Huge \Issue\slshape\bfseries\Black{\IssueNumber}
\end{minipage}
\hfill
\begin{minipage}[t]{0.35\textwidth}
\centering
    \Brown{日期:\Date} \\
%\vspace{-0.1cm}
%    \Brown{贡献者:\Contributer} \\
\vspace{-0.1cm}
    \Brown{学科:\Subject} \\
\end{minipage}
\hspace{0.8cm}
\end{figure}

{\color{cyan!50!black} 阅读以下算法,回答下列问题。 }
\begin{lstlisting}
LinkList mynote(LinkList L)
{ //L是不带头结点的单链表头指针
    if(L&&L->next)
    {
        q=L; L=L->next; p=L;
`\textrm{S1:}`     while(p->next) p=p->next;
`\textrm{S2:}`     p->next=q; q->next=NULL;
    }
    return L;
}
\end{lstlisting}

\begin{enumerate}[itemsep=0pt,parsep=0pt]
\color{cyan!50!black}
    \item 说明语句 S1 的功能;
    \item 说明语句组 S2 的功能;
    \item 设链表表示的线性表为 $(a_1,a_2,\dots,a_n)$,写出算法执行后的返回值所表示的线性表。
\end{enumerate}

这是一个典型的链表操作算法。\textcolor{cyan}{\CJKsout{很容易看出它根本无法编译。}}

单链表 $L$ 不带表头结点,因此表头位置实实在在存储了元素;$p,q$ 是两个指针。我们用赛跑的情景描述这个算法,逐行阅读它:

\begin{itemize}
    \item 第 3 行,如果表中至少有 2 个元素,就可以进入条件。\textcolor{cyan}{(不是循环)}
    \item 第 5 行,$p$ 跑在 $q$ 的「前面」\textcolor{cyan}{(即下一个位置)},$q$ 在「起点」。
    \item 第 6 行(S1),我们让 $p$ 不断地探路:$p$ 会「向前跑」,直到到达「终点」。
    \item 第 7 行(S2),我们把 $q$ 所指的位置成为 $p$ 的下一个元素,并让 $q$ 的下一个为空。这也就意味着,程序把 $q$ 所指的「起点」位置挪到了「终点」的下一个位置,原来的「起点」变成了新的「终点」,如图 \ref{fig} 所示。
    \item 第 9 行,一轮过程结束后,算法结束,返回整个链表。
\end{itemize}

\newpage
\backgroundsetup{contents=\includegraphics{下半示例.png}, center, scale=1, angle=0, opacity=1}
\BgThispage

\begin{figure}[htb]
    \centering
    \begin{tikzpicture}[>=Stealth,scale=0.7]
        %\draw[lightgray] (0,0) grid (11,1);
        \foreach \i in {1,2,3,4,5}{
            \draw (2*\i,0) rectangle (2*\i+1,1);
            \ifnum \i<5 \draw[->] (2*\i+1, 0.5) -- (2*\i+2, 0.5); \fi
            \node at (2*\i+0.5, 0.5) {\i};
        }
        \node[font=\footnotesize] at (2.5, -0.5) {起点};
        \node[font=\footnotesize] at (10.5, -0.5) {终点};
        \draw[->, font=\footnotesize] (5, 2) -- (7, 2) node[right] {「前面」}; 
        \draw[->, cyan!70!black] (2.5, 2) node[above] {$q$} -- (2.5, 1);
        \draw[->, cyan!70!black] (10.5, 2) node[above] {$p$} -- (10.5, 1);
        
        \foreach \i in {2,3,4,5}{
            \draw (2*\i,-4) rectangle (2*\i+1,-3);
            \draw[->] (2*\i+1, -3.5) -- (2*\i+2, -3.5); 
            \node at (2*\i+0.5, -3.5) {\i};
        }
        \draw (12,-4) rectangle (13,-3);
        \node at (12.5,-3.5) {1};
        \draw[->, cyan!70!black] (12.5, -2) node[above] {$q$} -- (12.5, -3);
        \draw[->, cyan!70!black] (10.5, -2) node[above] {$p$} -- (10.5, -3);
        \node[font=\footnotesize] at (12.5, -4.5) {新的终点};
    \end{tikzpicture}
    \caption{执行第 7 行前后的图解(\texttt{p->next=q; q->next=NULL;})}\label{fig}
\end{figure}

把刚才的思考过程转化成书面语:「前面」=下一个结点,「起点」=第一个结点,「终点」=最后一个结点。返回的线性表在图中有所体现,就是 $(a_2,a_3,\dots,a_n,a_1)$。这样我们就得到以下答案。

\vspace{1em}
{\color{cyan!80!black} 【结论】
\begin{enumerate}
    \item 查询链表的尾节点(遍历链表)
    \item 将第一个结点链接到链表的尾部,作为新的尾结点
    \item 返回的线性表为 $(a_2,\dots,a_n,a_1)$
\end{enumerate}

【点评】本题是链表算法的阅读理解问题,对链表元素的移动和拼接(插入)作出了考察。理解「后继结点」的赋值操作的真正含义是解决此题的关键。
}

\end{document}