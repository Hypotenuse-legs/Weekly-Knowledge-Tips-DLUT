\documentclass[UTF8]{ctexart}
\usepackage{amsmath}
\usepackage{amssymb}
\usepackage{booktabs}
\usepackage{background}
\usepackage{caption,subcaption}
\usepackage{diagbox}
\usepackage{enumitem}
\usepackage{float}
\usepackage{fontspec}
%\usepackage{fourier}
\usepackage{geometry}
\usepackage{makecell}
\usepackage{tikz}
\usetikzlibrary{arrows.meta}
\usepackage{xcolor}

\geometry{a5paper, top=0.1cm, left=1cm, right=1cm, bottom=0.3cm, footskip=0.1cm}
\setCJKmainfont[BoldFont={汉仪文黑-85W},ItalicFont={方正苏新诗柳楷简体}]{汉仪文黑-55W}
\setfontfamily\Issue{Century Schoolbook}
\setfontfamily\Genshin{Genshin Teyvat Lingua Franca}
\newCJKfontfamily\TitleFont{思源宋体 CN Heavy}
\newfontfamily\timesnewroman{Times New Roman}
\captionsetup{font=small, labelfont=bf}
\setlist[itemize]{itemsep=0pt, parsep=0pt}
%\reversemarginpar

%\CTEXsetup[format = {\centering\bfseries\large}, beforeskip = 3pt, afterskip = 3pt]{section}
\CTEXsetup[format = {\color{cyan!50!black}\bfseries\large}]{subsection}

\colorlet{darkcyan}{cyan!50!black}
\newcommand\Black[1]{\textcolor[gray]{0.3}{#1}}
\newcommand\Brown[1]{\textcolor[HTML]{998A4E}{#1}}
\newcommand\Emph[1]{\colorbox{green!10}{\textcolor{green!30!black}{#1}}}
\newcommand\Notes[1]{\textcolor{yellow!50!black}{\small #1}}
\newcommand\Example[1]{\textcolor{cyan!70!black}{\small #1}}

\renewcommand\d{\mathrm{d}}
\newcommand\Cov{\mathrm{Cov}}

\newcommand\IssueNumber{22}
\newcommand\Date{2024-4-6}
%\newcommand\Contributer{@金光日}
\newcommand\Subject{概率与统计 A}
\newcommand\Source{2021数 I 考研第 16 题}


\begin{document}
\backgroundsetup{contents=\includegraphics{上半示例.png}, center, scale=1, angle=0, opacity=1}
\BgThispage
\begin{center}
%{\scriptsize\Issue \textcolor[HTML]{C8BA83}{\Genshin WEEKLY TIPS}}
\phantom{...}

{\Large\textcolor{brown!40!white}{\makebox[10cm][s]{\Genshin WEEKLY KNOWLEDGE TIPS}}}

\vspace{-2em}

{\Huge\bfseries\TitleFont \Black{知\ 识\ 小\ 料}}


\vspace{-0.1cm}
{\footnotesize \Brown{「电计 2203 班」周常规知识整理共享}}
\end{center}

\vspace{-0.5cm}


\begin{figure}[H]
\hspace{1cm}
\begin{minipage}[t]{0.3\textwidth}
\centering
    \Brown{\Genshin ISSUE}

    \vspace{-0.6cm}
    \Huge \Issue\slshape\bfseries\Black{\IssueNumber}
\end{minipage}
\hfill
\begin{minipage}[t]{0.35\textwidth}
\small
\centering
    \Brown{日期:\Date} \\
%\vspace{-0.1cm}
%    \Brown{贡献者:\Contributer} \\
\vspace{-0.1cm}
    \Brown{学科:\Subject} \\
\vspace{-0.1cm}
    \Brown{来源:\Source}
\end{minipage}
\hspace{0.8cm}
\end{figure}

{\color{cyan!50!black} 甲、乙两个盒子中有 2 个红球和 2 个白球,选取甲盒中任意一球,观察颜色后放入乙盒,再从乙盒中任取一球,令 $X,Y$ 分别表示从甲盒和乙盒中取到的红球的个数,则 $X$ 与 $Y$ 的相关系数为\underline{\hspace{2cm}}。

\vspace{1em}
}

首先列出 $X,Y$ 的\textbf{取值},均为 $0,1$。由于是离散型随机变量,所以可以考虑列分布列,直接得到数字特征。

要计算 $P(X=0,Y=0)$,可以把取球过程拆成\textbf{两个步骤}:
\begin{itemize}%[itemsep=0pt,parsep=0pt]
  \item 在甲盒中取一白球;
  \item 在已知乙盒中多了一个白球的条件下,在乙盒中取一白球。
\end{itemize}
因此我们可以借用\textbf{概率乘法公式}\textcolor{cyan}{(也就是分步乘法)}:
\begin{equation}\label{eq:1}
    P(X=0,Y=0) = P(X=0)\cdot P(Y=0|X=0) = \dfrac12\times\dfrac35 = \dfrac{3}{10}
\end{equation}

类似地,可以求出另外的概率值,列出\textbf{分布列}:
\begin{equation}\label{eq:2}
\begin{split}
   P(X=0,Y=1) &= P(X=0)\cdot P(Y=1|X=0) = \dfrac12\times\dfrac25 = \dfrac15 \\
   P(X=1,Y=0) &= P(X=1)\cdot P(Y=0|X=1) = \dfrac12\times\dfrac25 = \dfrac15 \\
   P(X=1,Y=1) &= P(X=1)\cdot P(Y=1|X=1) = \dfrac12\times\dfrac35 = \dfrac{3}{10} \\
\end{split}
\end{equation}

\begin{table}[htb]
    \centering
    \begin{tabular}{c|cc|c}
    \Xhline{1pt}
    \diagbox{$Y$}{$X$} & 0 & 1 & $p_{i\textbullet}$ \\
    \Xhline{.5pt}
    0 & $\frac{3}{10}$ & $\frac{1}{5}$ & $\frac{1}{2}$ \\
    1 & $\frac15$ & $\frac3{10}$ & $\frac12$ \\
    \Xhline{.5pt}
    $p_{\textbullet j}$ & $\frac12$ & $\frac12$ & 1\\
    \Xhline{1pt}
    \end{tabular}
    \caption{二维随机变量 $(X,Y)$ 的分布列}
\end{table}

有了分布列,就可以计算 $(X,Y)$ 的\textbf{相关系数}了。

\newpage
\backgroundsetup{contents=\includegraphics{下半示例.png}, center, scale=1, angle=0, opacity=1}
\BgThispage

\begin{equation}\label{eq:3}
    \rho_{XY} = \dfrac{\Cov(X,Y)}{\sqrt{D(X)}\sqrt{D(Y)}} = \dfrac{E(XY) - E(X)E(Y)}{\sqrt{D(X)}\sqrt{
    D(Y)}}
\end{equation}

这样就转化为计算 $E(X),E(Y),E(XY),D(X),D(Y)$ 了。

由「\textbf{联合确定边缘}」的思想,我们可以写出相关一维随机变量的分布列。其中注意 $P(XY=1) = P(X=1,Y=1) = \dfrac3{10}$。

\begin{figure}[htb]
\begin{minipage}[t]{.32\textwidth}
    \centering
    \begin{tabular}{ccc}
    \toprule
        $X$ & 0 & 1 \\
    \midrule
        $P$ & $\frac12$ & $\frac12$ \\
    \bottomrule
    \end{tabular}
    \subcaption{$X$ 的分布列}
\end{minipage}
\begin{minipage}[t]{.32\textwidth}
    \centering
    \begin{tabular}{ccc}
    \toprule
        $Y$ & 0 & 1 \\
    \midrule
        $P$ & $\frac12$ & $\frac12$ \\
    \bottomrule
    \end{tabular}
    \subcaption{$Y$ 的分布列}
\end{minipage}
\begin{minipage}[t]{.32\textwidth}
    \centering
    \begin{tabular}{ccc}
    \toprule
        $XY$ & 0 & 1 \\
    \midrule
        $P$ & $\frac7{10}$ & $\frac3{10}$ \\
    \bottomrule
    \end{tabular}
    \subcaption{$XY$ 的分布列}
\end{minipage}
\caption{相关一维随机变量的分布列}
\end{figure}

可以看到,$X,Y,XY$ 三个随机变量均服从二项分布,$X\sim B(1,p)$,$E(X)=p$,$D(X)=p(1-p)$,因此:
\begin{itemize}
  \item $E(X) = \frac12$,$D(X) = \frac12\times(1-\frac12) = \frac14$
  \item $E(Y) = \frac12$,$D(Y) = \frac12\times(1-\frac12) = \frac14$
  \item $E(XY) = \frac3{10}$
\end{itemize}
\begin{equation*}
    \rho_{XY} = \dfrac{E(XY) - E(X)E(Y)}{\sqrt{D(X)}\sqrt{
    D(Y)}} = \dfrac{\frac{3}{10} - \frac12\times\frac12}{\sqrt{\frac14}\times\sqrt{\frac14}} = \dfrac{1}{5}
\end{equation*}

\vspace{1em}
{\color{cyan!80!black} 【结论】$\dfrac{1}{5}$

\vspace{1em}
【点评】本题以相关系数为切入点,考察了相关系数的公式、协方差的公式、二维离散型随机变量分布列的求解、条件概率与乘法公式的应用等,算是一道综合性较强的问题。要解决此问题,关键是会运用「联合确定边缘」的思想,以及会把抽样过程分成两个步骤求解。
}


\end{document}