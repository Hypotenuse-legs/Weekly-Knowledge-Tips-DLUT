\documentclass[UTF8]{ctexart}
\usepackage{amsmath}
\usepackage{amssymb}
\usepackage{background}
\usepackage{caption, subcaption}
\usepackage{circuitikz}
\usepackage{float}
\usepackage{fourier}
\usepackage{fontspec}
\usepackage{geometry}
\usepackage{pifont}
\usepackage{tikz}
\usepackage{xcolor}
\usetikzlibrary{arrows.meta}



\backgroundsetup{contents=\includegraphics{示例.png}, center, scale=1, angle=0, opacity=1}
\geometry{a5paper, top=0.1cm, left=1cm, right=1cm, bottom=1.1cm}
\setCJKmainfont[BoldFont={汉仪文黑-85W},ItalicFont={方正苏新诗柳楷简体}]{汉仪文黑-55W}
\setfontfamily\Issue{Century Schoolbook}
\newCJKfontfamily\TitleFont{思源宋体 CN Heavy}
\newfontfamily\timesnewroman{Times New Roman}
\captionsetup{font=small, labelfont=bf}


\newcommand\Black[1]{\textcolor[gray]{0.3}{#1}}
\newcommand\Brown[1]{\textcolor[HTML]{998A4E}{#1}}
\newcommand\Emph[1]{\colorbox{red!20!}{\textcolor{red!80!black}{#1}}}
\newcommand\Correct[1]{\colorbox{green!20}{\textcolor{green!50!black}{#1}}}
\newcommand\mypi{\text{\timesnewroman π}}
%这4个信息随“刊号”更新
\newcommand\IssueNumber{02}
\newcommand\Date{2023-9-18}
%\newcommand\Contributer{@金光日}
\newcommand\Subject{模拟电子线路}
\renewcommand\thefootnote{\ding{\numexpr171 + \value{footnote}}}

\ctikzset{resistors/scale=0.6, % smaller R
    capacitors/scale=0.7, % even smaller C
    diodes/scale=0.6, % small diodes
    transistors/scale=1.3,
    sources/scale=0.6,
}


\begin{document}
\begin{center}
{\scriptsize\Issue \textcolor[HTML]{C8BA83}{WEEKLY TIPS}}

{\Huge\bfseries\TitleFont \Black{知\ 识\ 小\ 料}}

\vspace{-0.1cm}
{\footnotesize \Brown{「电计 2203 班」周常规知识整理共享}}
\end{center}

\vspace{-0.5cm}

\begin{figure}[H]
\hspace{1cm}
\begin{minipage}[t]{0.3\textwidth}
\centering
    \Brown{ISSUE.}

    \vspace{-0.6cm}
    \Huge \Issue\slshape\bfseries\Black{\IssueNumber}
\end{minipage}
\hfill
\begin{minipage}[t]{0.3\textwidth}
\centering
    \Brown{日期:\Date} \\
%\vspace{-0.1cm}
%    \Brown{贡献者:\Contributer} \\
\vspace{-0.1cm}
    \Brown{学科:\Subject} \\
\end{minipage}
\hspace{0.8cm}
\end{figure}

%此处以后填写正文

{\color{cyan!50!black}
如图 \ref{fig:title} 所示,已知电源 $v_\mathrm{s}$ 为正弦波电压,若二极管采用理想模型,试定性绘出负载 $R_\mathrm{L}$ 两端的电压 $v_\mathrm{L}$ 波形。

\begin{figure}[htb]
    \centering
    \captionsetup{font+={color={cyan!50!black}}}
    \begin{circuitikz}[european, scale=0.4]
        \draw (0,-3.5) to[vsource] (0,3.5) -- (6,3.5) to[short, -*] (6,3) to[short, stroke diode, -*] (9,0) -- (11,0) to[R] (11,-4) -- (2,-4) -- (2,0) to[short,-*] (3,0) to [short, stroke diode, -*] (6,-3) -- (6,-3.5) -- (0,-3.5) -- (0,-3);
        \draw (3,0) to[stroke diode] (6,3);
        \draw (6,-3) to[stroke diode] (9,0);
        \node[left, outer sep=10pt] at (0,0) {$v_\mathrm{s}$};
        \node[above left] at (4.5,1.5) {$\mathrm{D_1}$};
        \node[above right] at (7.5,1.5) {$\mathrm{D_2}$};
        \node[below right] at (7.5,-1.5) {$\mathrm{D_3}$};
        \node[below left] at (4.5,-1.5) {$\mathrm{D_4}$};
        \node[left, outer sep=5pt] at (11,-2) {$R_\mathrm{L}$};
        \node[right, outer sep=5pt] at (11,-2) {$v_\mathrm{L}$};
        \node at (-0.5, 1.2) {$+$};
        \node at (-0.5,-1.2) {$-$};
        \node at (12, -0.5) {$+$};
        \node at (12, -3.5) {$-$};
    \end{circuitikz}
    \caption{原题电路图}\label{fig:title}
\end{figure}
}

这题乍一看有好多交错排列的二极管,实际上可以用惯常的分析方式——把二极管断路,观察两端电位孰高孰低。

可以看到,当 $v_\mathrm{s}>0$ 时,$\mathrm{D_1}$ 阳极电位为 0,阴极电位为正,阴极电位更高些,所以它截止;同理,$\mathrm{D_2}$ 导通,$\mathrm{D_3}$ 截止,$\mathrm{D_4}$ 导通。再将电阻 $R_\mathrm{L}$ 稍微移动位置,得到当 $v_\mathrm{s}>0$ 时,等效电路如图 \ref{subfig:vs>0} 所示。

类似地,当 $v_\mathrm{s}<0$ 时(相当于对题图电压源正负对调),$\mathrm{D_1}$ 阳极电位为 0,阴极电位为负,阳极电位更高些,所以它导通;同理,$\mathrm{D_2}$ 截止,$\mathrm{D_3}$ 导通,$\mathrm{D_4}$ 截止。因此当 $v_\mathrm{s}<0$ 时,等效电路如图 \ref{subfig:vs<0} 所示。\footnote{需要注意的是,电压为正值表示参考方向与实际方向相同。图 \ref{subfig:vs<0} 中 $v_\mathrm{s}<0$,而由于已经将电压源的正负对调了,因此参考方向和实际方向仍旧相同,即图中标出来的电压为正值,也就是 $-v_\mathrm{s}$。}

\begin{figure}[htb]
\begin{minipage}[b]{.5\textwidth}
    \centering
    \begin{circuitikz}[european, scale=0.4]
        \draw (0,-3) to[vsource] (0,3) -- (6,3) -- (9,0);
        \draw (3,0) -- (6,-3) -- (0,-3);
        \draw (3,0) to[R] (9,0);
        \node[left, outer sep=10pt] at (0,0) {$v_\mathrm{s}$};
        \node[yellow!80!black, above right] at (7.5,1.5) {$\mathrm{D_2}$};
        \node[yellow!80!black, below left] at (4.5,-1.5) {$\mathrm{D_4}$};
        \node[above, outer sep=5pt] at (6,0) {$R_\mathrm{L}$};
        \node[below, outer sep=5pt] at (6,0) {$v_\mathrm{L}$};
        \node at (-0.5, 1.2) {$+$};
        \node at (-0.5,-1.2) {$-$};
        \node at (7.5, -0.5) {$+$};
        \node at (4.5, -0.5) {$-$};
    \end{circuitikz}

    \subcaption{$v_\mathrm{s}>0$ 的等效电路}\label{subfig:vs>0}
\end{minipage}
\begin{minipage}[b]{.5\textwidth}
    \centering
    \begin{circuitikz}[european, scale=0.4]
        \draw (0,-3) to[vsource] (0,3) -- (6,3) -- (3,0);
        \draw (9,0) -- (6,-3) -- (0,-3);
        \draw (3,0) to[R] (9,0);
        \node[left, outer sep=10pt] at (0,0) {$-v_\mathrm{s}$};
        \node[yellow!80!black, above left] at (4.5,1.5) {$\mathrm{D_1}$};
        \node[yellow!80!black, below right] at (7.5,-1.5) {$\mathrm{D_3}$};
        \node[above, outer sep=5pt] at (6,0) {$R_\mathrm{L}$};
        \node[below, outer sep=5pt] at (6,0) {$v_\mathrm{L}$};
        \node at (-0.5, 1.2) {$-$};
        \node at (-0.5,-1.2) {$+$};
        \node at (7.5, -0.5) {$+$};
        \node at (4.5, -0.5) {$-$};
    \end{circuitikz}

    \subcaption{$v_\mathrm{s}<0$ 的等效电路}\label{subfig:vs<0}
\end{minipage}
\caption{等效电路图}\label{fig:dengxiao}
\end{figure}

\newpage
\phantom{...}
\vspace{3.2cm}

因此当 $v_\mathrm{s}>0$ 时,由 KVL 得到 $v_\mathrm{s}-v_\mathrm{L}=0$,即 $v_\mathrm{L}=v_\mathrm{s}>0$;当 $v_\mathrm{s}<0$ 时,同样由 KVL 得到 $-v_\mathrm{s}-v_\mathrm{L}=0$,即 $v_\mathrm{L}=-v_\mathrm{s}>0$。

经由上述推导,可以定性地作出 $v_\mathrm{L}$ 的波形,如图 \ref{fig:wave} 所示。

\begin{figure}[htbp]
    \centering
    \begin{tikzpicture}[>=Stealth]
    \begin{scope}[yshift=4cm]
        \draw [->] (0,0) -- (7.5,0) node[below] {$\omega t$};
        \draw [->] (0,-1.5) -- (0,1.5) node[left] {$v_\mathrm{s}$};
       % \draw[domain = 0 : 4*pi ] plot (\x, {sin(2*(\x r))});
        \draw [thick, cyan!50!black, domain=0:pi] plot[smooth] (\x, {sin (2*(\x r))});
        \draw [thick, cyan!50!black, domain=pi:2*pi] plot[smooth] (\x, {sin (2*(\x r))});
        \coordinate[label=left:$O$] (O) at (0,0);
        \coordinate[label={below left, outer sep=-2pt}:$\mypi$] (A) at (pi/2,0);
        \coordinate[label={below right, outer sep=-2pt}:$2\mypi$] (B) at (pi,0);
        \coordinate[label={below left, outer sep=-2pt}:$3\mypi$] (C) at (3*pi/2,0);
        \coordinate[label={below right, outer sep=-2pt}:$4\mypi$] (D) at (2*pi,0);
    \end{scope}
    \begin{scope}
        \draw [->] (0,0) -- (7.5,0) node[below] {$\omega t$};
        \draw [->] (0,-1.5) -- (0,1.5) node[left] {$v_\mathrm{L}$};
        \draw [thick,cyan,domain=0:pi] plot[smooth] (\x, {abs(sin (2*(\x r)))});
        \draw [thick,cyan,domain=pi:2*pi] plot[smooth] (\x, {abs(sin (2*(\x r)))});
        \coordinate[label=left:$O$] (O1) at (0,0);
        \coordinate[label={below, outer sep=-2pt}:$\mypi$] (A1) at (pi/2,0);
        \coordinate[label={below, outer sep=-2pt}:$2\mypi$] (B1) at (pi,0);
        \coordinate[label={below, outer sep=-2pt}:$3\mypi$] (C1) at (3*pi/2,0);
        \coordinate[label={below, outer sep=-2pt}:$4\mypi$] (D1) at (2*pi,0);
        \draw[dashed] (A) -- (A1);
        \draw[dashed] (B) -- (B1);
        \draw[dashed] (C) -- (C1);
        \draw[dashed] (D) -- (D1);
    \end{scope}
    \end{tikzpicture}
    \caption{$v_\mathrm{s}$ 和 $v_\mathrm{L}$ 的波形图} \label{fig:wave}
\end{figure}

\textcolor{cyan!80!black}{【结论】如图 \ref{fig:wave} 下半部分所示。}

\textcolor{cyan!80!black}{【点评】这道题用到了二极管重要的性质——单向导电性,以及其理想模型的分析。抓住「断开二极管,比两端电位」的方法是解决本题的关键。}

\end{document} 